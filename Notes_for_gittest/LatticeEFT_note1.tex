%\documentclass[10pt]{article}
\documentclass[10pt]{book}
\def\baselinestretch{1.1}
\usepackage[bookmarks]{hyperref}
\usepackage{kotex} % korean tex
\usepackage[utf8]{inputenc} % set input encoding (not needed with XeLaTeX)
\usepackage{framed}
\usepackage{geometry} % to change the page dimensions
\geometry{letterpaper} % or letterpaper (US) or a5paper or....
% \geometry{margins=2in} % for example, change the margins 
%to 2 inches all round
% \geometry{landscape} % set up the page for landscape
%   read geometry.pdf for detailed page layout information

\usepackage{graphicx} % support the \includegraphics command and options

% \usepackage[parfill]{parskip} % Activate to begin paragraphs 
%with an empty line rather than an indent

\usepackage{booktabs} % for much better looking tables
\usepackage{array} % for better arrays (eg matrices) in maths
\usepackage{paralist} % very flexible & customisable lists 
%(eg. enumerate/itemize, etc.)
\usepackage{verbatim} % adds environment for commenting 
% out blocks of text & for better verbatim
\usepackage{subfig} % make it possible to include 
%more than one captioned figure/table in a single float
% These packages are all incorporated in the memoir class 
%to one degree or another...

%%% HEADERS & FOOTERS
\usepackage{fancyhdr} % This should be set 
% AFTER setting up the page geometry
\pagestyle{fancy} % options: empty , plain , fancy
\renewcommand{\headrulewidth}{0pt} % customise the layout...
\lhead{}\chead{}\rhead{}
\lfoot{}\cfoot{\thepage}\rfoot{}

%%% SECTION TITLE APPEARANCE
\usepackage{sectsty}
\allsectionsfont{\sffamily\mdseries\upshape} 
% (See the fntguide.pdf for font help)
% (This matches ConTeXt defaults)

%%% ToC (table of contents) APPEARANCE
\usepackage[nottoc,notlof,notlot]{tocbibind} 
% Put the bibliography in the ToC
\usepackage[titles,subfigure]{tocloft} 
% Alter the style of the Table of Contents
\renewcommand{\cftsecfont}{\rmfamily\mdseries\upshape}
\renewcommand{\cftsecpagefont}{\rmfamily\mdseries\upshape} % No bold!

\usepackage{amsmath}
\usepackage{amssymb}
\usepackage{epsfig}
\usepackage{color}

\usepackage{empheq}
% make possible to box equations, For example
%\begin{empheq}[box=\fbox]{align*}
%a&=b \tag{test}\\
%E&=mc^2 + \int_a^a x\, dx
%\end{empheq}

\parindent 10pt\textheight 9in\topmargin -0.4in\textwidth 6in
\oddsidemargin .25in\evensidemargin 0in
\def\bm{\boldsymbol}
\newcommand{\bea}{\begin{eqnarray}}
\newcommand{\eea}{\end{eqnarray}}
\newcommand{\be}{\begin{eqnarray}}
\newcommand{\ee}{\end{eqnarray}}
\newcommand{\no}{\nonumber \\}
\newcommand{\nnb}{\nonumber}
\newcommand{\etal}{{\it et al.}~}
\newcommand{\eg}{{\it e.g.}}
\newcommand{\ie}{{\it i.e.}}
\newcommand{\sll}[1]{#1\hspace{-0.5em}/}
\newcommand{\del}{\partial}
\def\vs{{\bm \sigma}}

\def\vh{{\bm h}}
\def\vk{{\bm k}}
\def\vl{{\bm l}}
\def\vm{{\bm m}}
\def\vn{{\bm n}}
\def\vp{{\bm p}}
\def\vq{{\bm q}}
\def\vr{{\bm r}}
\def\vR{{\bm R}}
\def\vv{{\bm v}}
\def\vx{{\bm x}}
\def\vy{{\bm y}}

\def\la{\langle}
\def\ra{\rangle}
\newcommand{\colmtwo}[2]
{\left(\begin{tabular}{c} {$#1$}\\{$#2$}\end{tabular}\right)}

\newcommand{\colmthr}[3]
{\left(\begin{tabular}{c} {$#1$}\\{$#2$}\\{$#3$}\end{tabular}\right)}
\newcommand{\twodmat}[4]
{\left(\begin{tabular}{cc} {$#1$}&{$#2$}\\
		{$#3$}&{$#4$}\end{tabular}\right)}
\newcommand{\threedmat}[9]
{\left(\begin{tabular}{ccc} {$#1$}&{$#2$}&{$#3$} \\
		{$#4$}&{$#5$}&{$#6$}\\
		{$#7$}&{$#8$}&{$#9$}\\
	\end{tabular}\right)}
\newcommand{\threejsymbol}[6]
{\left(\begin{tabular}{ccc} {$#1$}&{$#2$}&{$#3$}\\
                             {$#4$}&{$#5$}&{$#6$}\end{tabular}\right)}
\newcommand{\sixjsymbol}[6]
{\left\{\begin{tabular}{ccc} {$#1$}&{$#2$}&{$#3$}\\
                             {$#4$}&{$#5$}&{$#6$} \end{tabular}\right\}}


%%% The "real" document content comes below...

\title{Lattice EFT note 1}
\author{Young-Ho Song}
\date{\today}
%\date{} % Activate to display a given date or no date (if empty),
         % otherwise the current date is printed 

\begin{document}
\maketitle
\tableofcontents
\newpage

\chapter{Introduction}

\section{Euclidean Time}
Analytic continuation to Eucliean time ,$it\to t_E$, enables 
the numerical calculation of path integral.

\subsection{Euclidean action for Boson} 
Consider a Quantum mechanical path integral,
\bea 
& &\la x_f|U(t_f,t_i)|x_i\ra=\int {\cal D}x(t) e^{i S[x]} ,\quad  S[x]=\int_{t_i}^{t_f} dt L(x,\dot{x}), \no  
& &L(x,\dot{x})=\frac{m}{2}\dot{x}(t)^2 - V(x(t))=\dot{x}p -\left(\frac{p^2}{2m}+V(x) \right) , \quad p=m\dot{x} 
\eea 
Let us transform into a Euclidean time by $it\to t_E$, then (Note overall $-$ sign comes from the Lagrangian itself.)
\bea 
& &e^{i\int dt (\frac{m}{2}\dot{x}(t)^2 - V(x(t)))}\to e^{\int dt_E (-\frac{m}{2}\dot{x}(t_E)^2 - V(x(t_E))) }
= e^{-S_{E}[x]} ,\no 
& &  S_{E}[x]=\int dt_E L_E =\int dt_E \left[\frac{m}{2}\dot{x}(t_E)^2 + V(x(t_E))\right]. 
\eea 
Then, the following path integral gives the complete information for the quantum theory,(in Euclidean time)
\bea 
\la x_f|e^{-\tilde{H}(t_f-t_i)}|x_i\ra =\int_{x_i}^{x_f} {\cal D}x(t) e^{-S[x]}.
\eea 
{\color{red}  Be careful for the sign. In other words, it goes like $e^{-S_E}=e^{-H_E t}$}

Let us consider for a pion action for example. (interaction with nucleon will be considered later
and also the time derivative of pion will not be considered because we will not treat 
dynamic pion. $\pi_I$ where $I$ is an iso-spin index.)
\bea 
& &i S[\pi_I]=i\int dt \int d^3 x \left[\frac{1}{2}\del_\mu\pi_I \del^\mu \pi_I -\frac{1}{2}m_\pi^2\pi_I^2
                                  -V(\pi_I)\right] \no 
&\to& \int d\tau \int d^3 x \left[-\frac{1}{2}(\del_\tau \pi_I)^2 -\frac{1}{2}(\nabla \pi_I)^2 
                                  -\frac{1}{2}m_\pi^2\pi_I^2 -V(\pi_I)\right] \no 
&=& -\int d\tau \int d^3 x \left[\frac{1}{2}(\del_\tau \pi_I)^2 +\frac{1}{2}(\nabla \pi_I)^2 
+\frac{1}{2}m_\pi^2\pi_I^2 +V(\pi_I)\right]                                                                  
\eea 

\subsection{Euclidean action for Fermion}
Let us consider a non-relativistic Fermion action,
\bea 
i\int dt d^3x \psi^\dagger(i\del_t+\frac{\hbar^2\nabla^2}{2m})\psi-V
&\to& \int d\tau d^3x \psi^\dagger(-\del_\tau+\frac{\hbar^2\nabla^2}{2m})\psi-V \no 
  &=& -\int d\tau d^3x \psi^\dagger(\del_\tau-\frac{\hbar^2\nabla^2}{2m})\psi+V
\eea 

\subsection{Path integral in Euclidean time}
In a similar way, let us consider action for the QFT. 
Once Wick rotate to Euclidean time, the partition function can be written as
\bea
Z_T={\rm Tr}(e^{-TH})=\sum_n \la n| e^{-TH}|n\ra=\sum_n e^{-TE_n}. 
\eea
Here, it is important to remind that $|n\ra$ is a multi particle states.

Euclidean correlator is defined
\bea 
\la O_1(t) O_2(0)\ra_{T}=\frac{1}{Z_T}{\rm Tr}
  \left[e^{-(T-t)H}\hat{O}_2e^{-tH}\hat{O}_1\right] 
\eea 

Inserting complete states between operators, 
we have the first important relation,
\begin{framed}
\bea 
\lim_{T\to \infty} \frac{1}{Z_T}{\rm Tr}\left[
       e^{-(T-t)H}\hat{O}_2e^{-tH}\hat{O}_1\right]
       =\sum_{n}\la 0|\hat{O}_2|n\ra \la n|\hat{O}_1|0\ra e^{-t (E_n-E_0)}
\eea 
\end{framed}
This means, if we can compute the left hand-side by path integral,
we can extract the information on excited states, energy $E_n$ and wave function
or observables $\la 0|\hat{O}_2|n\ra \la n|\hat{O}_1|0\ra$. 

We can obtain information on quantum system by computing 
average of operators. Now the second key equation is
\begin{framed}
	\bea
	\frac{1}{Z_T}{\rm Tr}\left[e^{-(T-t)\hat{H}}\hat{O}_2 e^{-t\hat{H}}\hat{O}_1
	\right] 
	=\frac{1}{Z_T}\int{\cal D}[\Phi] e^{-S_E[\Phi]} O_2[\Phi(.,t)]O_1[\Phi(.,0)]. 
	\eea 
\end{framed}
This means that if we compute the path integral numerically, 
we can extract physical information we want from the first relation.

{\bf From now on}, let us represent any physical quantity as power of lattice size $a$,
so that all quantoty can be written as dimensionless.

\section{Example: Numerical Path Integral for scalar field}
{\color{blue} In this example, some details on how to construct the path integral for a scalar field
is shown. }

In Minkowski space, Lagrangian of scalar field is
\bea 
L(\Phi,\del_\mu\Phi)
=\frac{1}{2}\dot{\Phi}^2-\frac{1}{2}(\nabla\Phi)^2-\frac{m^2}{2}\Phi^2-V(\Phi).
\eea 
The Hamiltonian of scalar field becomes
\bea
\hat{H}=\int d^3 x\left(\frac{1}{2}\hat{\Pi}(\vx)^2+\frac{1}{2}(\nabla\hat{\Phi}(\vx))^2
      +\frac{m^2}{2}\hat{\Phi}(\vx)^2+V(\hat{\Phi}(\vx))
 \right), 
\eea 
with equal time commutation relation,
\bea 
\left[ \hat{\Phi}(\vx), \hat{\Pi}(\vy)\right]=i\delta(\vx-\vy),
\quad 
\left[ \hat{\Phi}(\vx),\hat{\Phi}(\vy)\right]
=\left[\hat{\Pi}(\vx),\hat{\Pi}(\vy)\right]=0,
\eea 
where,
\bea
\hat{\Pi}(t,\vx)=\frac{\del}{\del \dot{\Phi}(t,\vx)}L(\Phi,\del_\mu\Phi)
=\dot{\Phi}(t,\vx),
\eea 
or equivalently, from the commutation relation, we can 
considering $\hat{\Pi}$ as an operator in $\Phi$ representation,
\bea 
 \hat{\Pi}(\vx)=i \frac{\del}{\del\Phi(\vx)}.
\eea 
By introducing 3-D lattice $\Lambda_3$,
\bea 
\vx\to a{\bm n},\quad n_i=0,1,\dots, N-1\quad i=1,2,3.
\eea 
We change Hamiltonian into lattice form,
\bea 
\hat{H}=a^3 \sum_{ {\bm n}\in\Lambda_3} \left(\frac{1}{2}\hat{\Pi}({\bm n} )^2
   +\frac{1}{2}\sum_{j=1}^3 \left(\frac{\hat{\Phi}({\bm n}+j)-\hat{\Phi}({\bm n}-j)}{2a} \right)^2 
      +\frac{m^2}{2}\hat{\Phi}({\bm n})^2+V(\hat{\Phi}({\bm n}) )\right) 
\eea 
With,
\bea 
\left[ \hat{\Phi}({\bm n}), \hat{\Pi}({\bm m})\right]=i a^{-3}\delta_{{\bm n},{\bm m}},
\quad 
\left[ \hat{\Phi}({\bm n}),\hat{\Phi}({\bm m})\right]
=\left[\hat{\Pi}({\bm n}),\hat{\Pi}({\bm m})\right]=0,
\eea 
And, $\hat{\Pi}({\bm n})$ can be considered as a derivative operator according to the commutation relation,
\bea 
\hat{\Pi}({\bm n})=-\frac{i}{a^3}\frac{\del}{\del\Phi({\bm n})}.
\eea 
Then considering action of operators,
\bea 
& &\hat{\Phi}({\bm n})|\Phi\ra = \Phi({\bm n})|\Phi\ra,\no  
& &\la \Phi'|\Phi\ra=\delta(\Phi'-\Phi) \equiv \prod_{{\bm n}\in\Lambda_3} 
   \delta(\Phi'({\bm n})-\Phi({\bm n})),\no 
& & 1=\int D\Phi |\Phi\ra \la \Phi|,\quad D\Phi=\prod_{{\bm n}\in\Lambda_3} d\Phi({\bm n}). 
\eea 
One can write the eigen state of $\hat{\Pi}(\vn)$ operator as $\la \Phi|\Pi\ra$,
\bea 
\la \Phi|\Pi\ra=\prod_{{\bm n}\in\Lambda_3} \sqrt{\frac{a^3}{2\pi}} 
    e^{i a^3 \Pi({\bm n})\Phi({\bm n})},\quad \hat{\Pi}(\vn)\la \Phi|\Pi\ra= \Pi(\vn)\la \Phi|\Pi\ra.
\eea 
Then, free Hamiltonian evolves as
\bea 
\la \Phi|\hat{H}_0|\Pi\ra
&=&-\frac{1}{2a^3}\sum_{{\bm n}\in\Lambda_3} \frac{\del^2}{\del \Phi({\bm n})^2}\la \Phi|\Pi\ra
=\frac{a^3}{2}\sum_{{\bm n}\in\Lambda_3} \Pi({\bm n})^2 \la \Phi|\Pi\ra,\no   
\la \Phi'|e^{-t\hat{H}_0}|\Phi\ra 
&=&\int D\Pi \la \Phi'|e^{-t\hat{H}_0}|\Pi\ra\la \Pi|\Phi\ra 
 =\int D\Pi \la \Phi'|\Pi\ra\la \Pi|\Phi\ra e^{-t\frac{a^3}{2}\sum_{n} \Pi({\bm n})^2}\no 
 &=&\prod_{{\bm n}\in\Lambda_3} \frac{a^3}{2\pi}\int d\Pi({\bm n}) 
    e^{i a^3\Pi({\bm n})(\Phi'({\bm n})-\Phi({\bm n}))}e^{-ta^3\Pi({\bm n})^2/2}  \no 
 &=&\prod_{{\bm n}\in\Lambda_3}\sqrt{\frac{a^3}{2\pi t}} e^{-a^3/(2t)(\Phi'({\bm n})-\Phi({\bm n}))^2}
 =  \left(\frac{a^3}{2\pi t}\right)^{\frac{N^3}{2}} e^{-\frac{a^3}{2t}\sum_n (\Phi'({\bm n})-\Phi({\bm n}))^2}  
\eea 

With interaction, 
\bea
\hat{W}_\epsilon&=&
e^{-\epsilon \hat{U}/2}e^{-\epsilon\hat{H}_0}e^{-\epsilon\hat{U}/2},\no 
\la\Phi'|e^{-t\hat{H}}|\Phi\ra
&=&\lim_{n_t\to\infty}\la \Phi'|\hat{W}_\epsilon^{n_t}|\Phi\ra
=\lim_{n_t\to\infty}\int D\Phi_1 \dots D\Phi_{n_t-1}
 \la \Phi'|\hat{W}_\epsilon|\Phi_{n_t-1}\ra\dots
 \la \Phi_1|\hat{W}_\epsilon|\Phi\ra.
\eea 
And the transfer matrix elements
\bea
\la \Phi_{i+1}|\hat{W}_\epsilon|\Phi_i\ra
&=& \la \Phi_{i+1}|e^{-\epsilon \hat{U}/2}e^{-\epsilon\hat{H}_0}e^{-\epsilon\hat{U}/2}|\Phi_i\ra
=e^{-\epsilon U[\Phi_{i+1}]/2}
 \la \Phi_{i+1}|e^{-\epsilon\hat{H}_0}|\Phi_i\ra
 e^{-\epsilon U[\Phi_i]/2} \no 
&=& C^{N^3}e^{-\epsilon U[\Phi_i]/2
           -a^3/(2\epsilon)\sum_n (\Phi({\bm n})_i-\Phi({\bm n})_{i+1})^2
           -\epsilon U[\Phi_{i+1}]/2}               
\eea

Finally, the partition function can be written as a path integral,
\bea 
Z_T^\epsilon&=&\int D\Phi_0\la\Phi_0|\widehat{W}_\epsilon^{N_T}|\Phi_0\ra 
 = C^{N^3 N_T} \int D\Phi_0\dots D\Phi_{N_T-1}
    e^{-S_E[\Phi]}\no 
 &=& C^{N^3 N_T}\int {\cal D}[\Phi] e^{-S_E[\Phi]},\quad 
      {\cal D}[\Phi]=\prod_{({\bm n},n_4)\in\Lambda}d\Phi({\bm n},n_4),
\eea 
where 
\bea          
S_E[\Phi]&=& \epsilon a^3\sum_{({\bm n}n_4)\in \Lambda_4}\left[ 
   \frac{1}{2}\left( \frac{\Phi({\bm n},n_4+1)-\Phi({\bm n},n_4)}{\epsilon}   \right)^2
   \right. \no & & \left.  
   +\frac{1}{2}\sum_{j=1}^3\left(\frac{\Phi({\bm n}+\hat{j},n_4)-\Phi({\bm n}-\hat{j},n_4)}{2a} \right)^2  
   +\frac{m^2}{2}\Phi({\bm n},n_4)^2+V(\Phi({\bm n},n_4)) \right] 
\eea 


\section{Example: non-relativistic quantum mechanics of 1-body in a potential.}
{\color{blue} This example shows how to construct path integral and how to compute it numerically.
Also how to extract the information for ground state or excited state.}

As an example of numerical calculation, let us evaluate 
\bea 
\la x| e^{-H(t_f-t_i)}|x\ra. 
\eea 
for a particle in an external potential in quantum mechanics.

Discretize the time as 
\bea 
t_j=t_i+j a,\quad \mbox{for }j=0,1,\dots N,\quad a_t=\frac{t_f-t_i}{N}
\eea 
A path(or a configuration) is described by a vector
\bea 
x=\{ x(t_0),x(t_1),\dots, x(t_N)\}.
\eea 
Then, in a notation $x_j=x(t_j)$ and b.c. $x_0=x_N=x$,
\bea 
\int {\cal D}x(t)\to A\int_{-\infty}^{+\infty} dx_1 dx_2\dots dx_{N-1},
\eea 
where endpoints($x_0$ and $x_N$) are fixed (not integrated).\footnote{
Normalization factor $A=(\frac{m}{2\pi a})^{N/2}$.} 
And, in a simple approximation
\bea 
\int_{t_j}^{t_{j+1}} dt L\simeq a\left[\frac{m}{2}\left(\frac{x_{j+1}-x_j}{a}\right)^2+\frac{1}{2}(V(x_{j+1})+V(x_j))\right]  
\eea 

Thus, 
\bea 
& &\la x|e^{-\tilde{H} T}|x\ra \simeq A\int_{-\infty}^\infty dx_1\dots d x_{N-1} e^{-S_{lat}[x]},\no 
& &S_{lat}[x]\equiv \sum_{j=0}^{N-1} \left[\frac{m}{2a}(x_{j+1}-x_j)^2+a V(x_j) \right],
\eea 
we can evaluate the path integral numerically. 

One way to do this integration is using Monte Carlo integration for all (N-1) variables by choosing 
random numbers. 

Let us consider now excited states by sum over end points. 
To analyze excited states using path integral, 
interrupt the propagation of the ground state by new operators. 
\bea 
G(t)\equiv \sum_{x} \la x|x(t_2)x(t_1)|x\ra =
    \la \la x(t_2)x(t_1)\ra \ra 
    \equiv \frac{\int {\cal D}x(t) x(t_2)x(t_1)e^{-S[x]}}{\int {\cal D}x(t)e^{-S[x]}  }
\eea 
where now we integrate over all $x_i=x_f=x$ as well as the intermediate $x(t)$'s. 
In a very large time limit $T=t_f-t_i$ , the ground state contribution will be dominant,
and large $t=t_2-t_1$ limit,
\bea 
\int {\cal D}x(t) x(t_2)x(t_1)e^{-S[x]}&=&
   \int dx \la x| e^{-\tilde{H}(t_f-t_2)}\tilde{x} e^{-\tilde{H}(t_2-t_1)}\tilde{x} e^{-\tilde{H}(t_1-t_i)}|x\ra 
   ,\no 
\la \la x(t_2)x(t_1)\ra \ra&=&\frac{\sum e^{-E_n T}\la E_n|\tilde{x} e^{-(\tilde{H}-E_n)t}\tilde{x}|E_n\ra}
                         {\sum e^{-E_n T}} \no 
                         &\to& \la E_0| \tilde{x} e^{-(\tilde{H}-E_0)t}\tilde{x}|E_0\ra
                         \to |\la E_0|\tilde{x}|E_1\ra|^2 e^{-(E_1-E_0)t} 
\eea 
Thus, one can extract the excited state by 
\bea 
\log (G(t) /G(t+a))\to (E_1-E_0) a
\eea 
and also matrix element $|\la E_0|\tilde{x}|E_1\ra|^2$. 

For example, we may analyze the time dependence
\bea 
G(t)=\frac{1}{N}\sum_j \la \la x(t_j+t)x(t_j)\ra\ra
\to G_n = \frac{1}{N}\sum_j \la \la x_{(j+n)mod N}  x_j\ra\ra 
\eea 
for all $t=0,a_t,2a_t,\dots (N-1) a_t$. And get energy difference,
\bea 
\Delta E_n\equiv \log(G_n/G_{n+1})\to_{large\ n} (E_1-E_0)a
\eea 


One can extract any information on quantum mechanical
system by using some operator,
\bea 
\la \la \Gamma[x]\ra\ra = \frac{\int {\cal D}x(t) \Gamma[x] e^{-S[x]}}{\int {\cal D}x(t)e^{-S[x]}  },
\eea 
where it is a weighted average with weight $e^{-S[x]}$, over large $N_{cf}$ configurations.
If we choose particular paths(configuration $(\alpha)$) with probability 
\bea 
P[x^{(\alpha)}]\propto e^{-S[x^{(\alpha)}]},
\eea 
we get 
\bea 
\la \la \Gamma[x]\ra\ra\simeq \bar{\Gamma}\equiv \frac{1}{N_{cf}}\sum_{\alpha=1}^{N_{cf}} \Gamma[x^{(\alpha)}].
\eea 

Monte Carlo uncertainty in this approximation is estimated as
\bea 
\sigma^2_{\bar{\Gamma}}\simeq \frac{\la\la \Gamma^2\ra \ra -\la\la\Gamma\ra\ra^2}{N_{cf}}
\eea 

One way to update $x^{(\alpha)}$ to $x^{(\alpha+1)}$ configuration is Metropolis algorithm. 

More details on the algorithm is given in later chapter. 
But one have to take into account (1) avoid correlation between updates. (This can be done by keeping only $N_{cor}$-th path) (2) when starting the procedure requires "thermalizing the lattice"(This can be done
by discarding some initial configurations). 

\section{Example: 1D QFT of scalar particle}
{\color{blue} This is a continuation from previous example for the case of scalar particle.} 
	

For QFT, we change $x(t)\to \phi(t)\to \phi(t,x)$.
\bea 
\la\la \Gamma[\phi]\ra\ra \equiv \frac{1}{\cal Z}\int \prod_{x_j\in grid}d\phi(x_j) e^{-S[\phi]}\Gamma[\phi], 
\eea 
\bea 
{\cal Z}\equiv \int \prod_{x_j\in grid}d\phi(x_j) e^{-S[\phi]}
\eea 

For example, one may use operators like
\bea 
\Gamma(t)\equiv \frac{1}{\sqrt{N}}\sum_{x_j}\phi(x_j,t),
\eea 
which corresponds to zero momentum state $\phi(p=0,t)$. Then, in large time limit, 
only the lowest energy excitation state ,which is one particle state, 
will contribute to the average  
\bea 
\la \la \Gamma(t)\Gamma(0)\ra\ra \to |\la 0|\Gamma(0)|\phi: p=0\ra |^2 e^{-m_\phi t}
\eea 


\section{Example: Non-relativistic free particle action in lattice}
{\color{blue} This example shows how to write the approximate discretized lattice action for free particle.
}

Relativistic single particle energy spectrum in lattice have a doubling problem because of
the relation $E=\pm\sqrt{p^2+m^2}$.  But, non-relativistic case does not have doubling problem. 

After discretization, 
\bea 
\int_0^L dx\to a \sum_{n=0}^{N-1},\quad a(x)\to a(n) 
\eea 
We may approximate the free Hamiltonian as( note that this is just one possibility.)
\bea
H_{free}&=&-\frac{1}{2m}\int_0^L dx a^\dagger(x)\frac{\del^2}{\del x^2} a(x)
          =\frac{1}{2m}\int_0^L dx (\frac{\del}{\del x} a^\dagger(x))(\frac{\del}{\del x} a(x) )\no 
        &\to& -\frac{a}{a^2}\frac{1}{2m}\sum_{n=0}^{N-1}[ a^\dagger(n+1) a(n)+  a^\dagger(n) a(n+1)
                                            -2 a^\dagger(n)a(n)]\no 
        &=&-\frac{a}{a^2}\frac{1}{2m}\sum_{n=0}^{N-1}[ a^\dagger(n+1) a(n)+  a^\dagger(n-1) a(n)
                      -2 a^\dagger(n)a(n)]                                     
\eea
where, $a(N)=a(0)$ with periodic boundary condition.
We expect the energy of one particle momentum eigenstate $|\vp\ra$ to be $\frac{\vp^2}{2m}$
in continuum limit. However, as shown below, the energy eigen value for this approximation 
does not give the exact continuum limit value.

A momentum eigen state on 1-D lattice can be defined( as a translation operator eigen state) as
\bea 
|p\ra =\frac{1}{\sqrt{N}}\sum_{n=0}^{N-1} e^{ipn} a^\dagger(n)|0\ra ,
 \quad p=\frac{2\pi k}{N} \mbox{ with k integer}
\eea 
Then the energy of this state for approximate $H^{free}$ is 
\bea 
H^{free}|p\ra&=&-\frac{1}{2m}\sum_{n'=0}^{N-1}[ a^\dagger(n'+1) a(n')+a^\dagger(n') a(n'+1) -2a^\dagger(n') a(n') ] 
\left( \frac{1}{\sqrt{N}}\sum_{n=0}^{N-1} e^{ipn}a^\dagger(n)|0\ra \right) \no 
&=&-\frac{1}{2m}\frac{1}{\sqrt{N}}\sum_{n=0}^{N-1} e^{ipn}
[ a^\dagger(n+1)+a^\dagger(n-1)-2a^\dagger(n)]|0\ra  \no 
&=&-\frac{1}{2m}\frac{1}{\sqrt{N}}\sum_{n=0}^{N-1} 
[ e^{-ip}+e^{ip}-2]e^{ipn} a^\dagger(n)|0\ra \no 
&=& -\frac{1}{2m} (e^{-ip}+e^{ip}-2) |p\ra =\frac{1}{m}(1-\cos p)|p\ra   
\eea 
Thus, the dispersion relation for this approximate Hamiltonian is 
\bea 
 E(p)=\frac{1}{m}(1-\cos p)=\frac{p^2}{2m}+{\cal O}(p^4)
\eea 
and only approximately the same with continuum limit. 

\subsection{Improved action} 
We may ``improve" the Hamiltonian so that it gives better agreement in
continuum limit. 
Suppose we discretize the Hamiltonian by 
\bea 
\boxed{ 
H_{free}=\frac{1}{2m}\sum_{n=0}^{N-1}\sum_{\Delta=0}^{\Delta_{max}}
         (-1)^\Delta \omega_\Delta[a^\dagger(n+\Delta) a(n)+a^\dagger(n)a(n+\Delta)].}
\eea 
Then the spectrum of single particle state is
\bea
H_{free}|p\ra&=&\left( \frac{1}{2 m}\sum_{\Delta=0}^{\Delta_{max}} 
              (-1)^\Delta \omega_\Delta(e^{i\Delta p}+e^{-i\Delta p})\right) |p\ra 
             =Q(p)|p\ra  \no 
Q(p)&=&\frac{1}{m}\sum_{\Delta=0}^{\Delta_{max}}
             (-1)^\Delta \omega_\Delta\cos(\Delta\cdot p)\no 
    &=&\frac{1}{m}(\omega_0-\omega_1 \cos(p)+\omega_2 \cos(2p)+\dots )                     
\eea 
Then by expanding in power of $p$, 
\bea
Q(p)=\frac{1}{m} \sum_{\Delta=0}^{\Delta_{max}}\sum_{\nu=0}^\infty
    \frac{(-1)^\nu}{(2\nu)!}\omega_\Delta  \Delta^{2\nu}p^{2\nu} 
\eea 
By choosing appropriate $\omega_\Delta$, one can set $Q(p)=\frac{p^2}{2m}+{\cal O}(p^{n})$, 
and approximate Hamiltonian which eliminates lattice artifact.

In 3-d lattice dispersion relation of free particle,
$Q^{(n)}(p)=\frac{1}{2m}\sum_{l=1,2,3} p_l^2[1+O(a^{2n+2})]$, 
\bea 
\boxed{ 
	H_{free}=\frac{1}{2m}\sum_{l=1,2,3}\sum_{n=0}^{N-1}\sum_{\Delta=0}^{\Delta_{max}}
	(-1)^\Delta \omega_\Delta[a^\dagger(n+\Delta\hat{l}) a(n)+a^\dagger(n)a(n+\Delta\hat{l})].}
\eea 

lowest order approximation is 
\bea 
\omega_0=1,\quad  \omega_1=1.
\eea 
At ${\cal O}(a^2)$ improved,
\bea 
\omega_0=\frac{5}{4},\quad 
\omega_1=\frac{4}{3},\quad 
\omega_2=\frac{1}{12}.
\eea 

At ${\cal O}(a^4)$ improved,
\bea
\omega_0=\frac{49}{36},\quad 
\omega_1=\frac{3}{2},\quad 
\omega_2=\frac{3}{20},\quad 
\omega_3=\frac{1}{90}.
\eea 
and for $Q(p)=\frac{p^2}{2m}+{\cal O}(p^{10})$, we get({\color{red} sign error?})
\bea
\omega_0=\frac{205}{144},\quad \omega_1=-\frac{8}{5},\quad \omega_2=\frac{1}{5},
    \quad \omega_3=-\frac{8}{315},\quad \omega_4=\frac{1}{560}.
\eea 

Sometimes, "well-tempered(wt) action" can be used. They are defined implicitly 
in terms of their dispersion relation,
\bea 
\omega^{(wtn)}(\vp)=\omega^{(n-1)}(\vp)+c[\omega^{(n)}(\vp)-\omega^{(n-1)}(\vp)],
\eea 
where the unknown constant $c$ was determined by the integral constraint,
\bea 
\int_{-\pi}^{\pi}\int_{-\pi}^{\pi}\int_{-\pi}^{\pi} dp_1 dp_2 dp_3
\left[\omega^{(wtn)}(\vp)-\frac{1}{2m}\sum_{l=1,2,3} p_l^2 \right] =0.
\eea 
Note that this does not mean that dispersion relation is exact for $|{\bm p}\ra$.  

\section{Example: lattice QCD action}
{\color{blue} This section discuss about the QCD lattice action. One may skip this section.}


In case of QCD, before making path integral formula, we need lattice action for quarks and
gluons. Because of discretization, normal covariant derivative form does not satisfy 
gauge invariance exactly. Thus, it is difficult to use gluon field as variable.
Instead, we introduce link variable as a gauge transporter. 
Then, fermion action becomes,
\bea 
S_F[\psi,\bar{\psi},U]&=&a^4\sum_{f=1}^{N_f}\sum_{{\bm n}\in \Lambda}
  \left( \bar{\psi}^{(f)}({\bm n})\sum_{\mu=1}^{4}\gamma_\mu 
  \frac{U_\mu(n) \psi^{(f)}(n+\hat{\mu})-U_{-\mu}(n) \psi^{(f)}(n-\hat{\mu})}{2a}
  \right.\no & &\left.
  +m^{(f)} \bar{\psi}^{(f)}(n)\psi^{(f)}(n)
  \right) +\mbox{ additional terms }
\eea  
where, link variable, $U_\mu(n)$, links between  site ${\bm n}$ and ${\bm n}+\mu$
and make the term gauge invariant,
\bea
& &\bar{\psi}'(n)U'_\mu(n)\psi'(n+\mu)=\bar{\psi}(n)\Omega(n)^\dagger U'_\mu(n)\Omega(n+\mu)
  \psi(n+\mu)=\bar{\psi}(n)U_\mu(n)\psi(n+\mu),\no 
& &U_\mu(n)\to U'_\mu(n)=\Omega(n)U_\mu(n)\Omega(n+\mu)^\dagger, \quad 
   U_{-\mu}(n)=U_\mu(n-\mu)^\dagger  
\eea 
Explicit form of link and gauge field can be thought as
\bea 
U_\mu(n)=\exp(ia A_\mu(n)),
\eea 
which corresponds to the covariant derivative in continuum limit
and make sure fermion bilinear becomes gauge invariant.

Another gauge invariant term is a closed loop, $L[U]$,
\bea
L[U]={\rm tr}\left[U_{\mu_{0}}(n_0)U_{\mu_1}(n_0+\mu_0)\dots U_{-\mu_{N}}(n_0) \right]
    ={\rm tr}\left[\prod_{(n,\mu)\in \mbox{ loop}} U_\mu(n)\right] 
\eea 

We can construct lattice gauge action in terms of plaquette such that
\bea 
S_G[U]=\frac{2}{g^2}\sum_{n\in\Lambda}\sum_{\mu<\nu}{\rm Re}\left[ {\rm tr}[I-U_{\mu\nu}(n)]\right] 
 =\frac{a^4}{2g^2}\sum_{n\in\Lambda}\sum_{\mu,\nu}{\rm tr}[F_{\mu\nu}(n)^2]+{\cal O}(a^2).
\eea 
where, plaquette is a product of 4 links,
\bea 
U_{\mu\nu}(n)=U_\mu(n)U_\nu(n+\mu)U_\mu(n+\nu)^\dagger U_\nu(n)^\dagger
 =\exp(ia^2 F_{\mu\nu}(n)+{\cal O}(a^3)). 
\eea 

\subsubsection{Gauge invariance/fixing/Integration}
Because the guage link is a matrix, it is necessary to specify how to integrate over 
elements,$\int d U_\mu({\bm n})$ ,  in Lie algebra. 
Also, it is important to make the measure gauge invariant.
Such measure is known as Harr measure. 

The Harr measure satisfies the properies,
\bea
d U&=&d (UV)=d(VU),\quad V \in G,\no 
\int dU\ 1 &=& 1.
\eea 
It can be defined as
\bea
dU&=&c \sqrt{\mbox{det}[g(\omega)]}\prod_{k} d\omega^{(k)},\no 
g(\omega)_{nm}&=&{\rm tr}\left[
   \frac{\del U(\omega)}{\del\omega^{(n)}}
   \frac{\del U(\omega)^\dagger}{\del\omega^{(m)}}     \right].
\eea
where, $\omega$ are real parameters for compact Lie group. 
\footnote{
some of basic integrals for SU(3) are
\bea
& &\int_{SU(3)} dU U_{ab}=0,\no
& &\int_{SU(3)} dU U_{ab} U_{cd}=0,\no
& &\int_{SU(3)} dU U_{ab} (U^\dagger)_{cd}=\frac{1}{3}\delta_{ad}\delta_{bc},\no
& &\int_{SU(3)} dU U_{ab}U_{cd} U_{ef}=\frac{1}{6}\epsilon_{ace}\epsilon_{bdf}.
\eea
}
We can fix gauge in many ways in lattice. Simpliest might be the temporal gauge,
\bea
U_4(n)=1,\quad \forall n .
\eea

\section{Gaussian integration in D-dimension}
The most basic integral in field theory is the Gaussian integration,
\bea
\int_{-\infty}^\infty dv ^{-\frac{1}{2}a v^2}=\left(\frac{2\pi}{a}\right)^{\frac{1}{2}}.
\eea 

For a D-dimensional vectors $v,\rho$ and $D\times D$ real symmetric matrix A, 
\bea 
\int d^D v e^{-\frac{1}{2} v^T A v}=(2\pi)^{D/2} e^{-\frac{1}{2}{\rm Tr} \ln A}
                                  =(2\pi)^{D/2} (\det A)^{-\frac{1}{2}}
\eea
\begin{framed}
\bea
\int d^D v e^{-\frac{1}{2}v^T A v+\rho^T v}
  =(2\pi)^{D/2} e^{-\frac{1}{2}{\rm Tr} \ln A}e^{\frac{1}{2}\rho^T A^{-1} \rho}
\label{eq:gaussian_rhov}  
\eea 
\end{framed} 
a slight modification gives
\footnote{ 	This is an analogy of
	\bea
	\int_{-\infty}^{+\infty} dx e^{-c x^2\pm i b x}=\sqrt{\frac{\pi}{c}}e^{-b^2/4c},
	\eea
}
\bea 
\int d^D v\ e^{-\frac{1}{2} v^T A v} v_{k_1}\dots v_{k_n} 
=(2\pi)^{D/2} e^{-\frac{1}{2}{\rm Tr} \ln A}
\left( A_{k_1 k_2}^{-1}\dots A_{k_{n-1}k_n}^{-1}+\mbox{permut.}\right) 
\eea 

This relation can be used later for the introduction of auxiliary particle field. 

Another useful form may be 
\bea 
\int {\cal D} v e^{-\frac{1}{2} v^T A v +\rho^T B v} =
 (2\pi)^{D/2} e^{-\frac{1}{2}{\rm Tr} \ln A} e^{\frac{1}{2} \rho^T X^{-1} \rho },
\quad X^{-1}=B A^{-1} B^T .
\eea 



When A is a positive definite hermitian matrix and z is a complex vector,
\bea
\int dz_1 d z_1^*\dots dz_D d z_D^* e^{-z^\dagger A z}
  =(2\pi)^D e^{-{\rm Tr}\ln A}
\eea 
where complex measure is defined as
\bea
\int dz dz^*\ f(z,z^*)=2\int d{\rm Re}z \int d {\rm Im}z\ f(z,z^*)
\eea 

In many case, we need the integration of continuous fields. Let us define, 
scalar products for continuum.
\bea 
(\phi,\psi)&=& \int d^4 x \phi(x)\psi(x),\no 
(\phi,A\psi)&=& \int d^4 x \int d^4 x' \phi(x) A(x,x')\psi(x')
\eea 
Then, we can use similar expressions for integration over continuous fields by replacing,
$v^T A v \to (\phi^\dagger,A\phi)$





\subsubsection{an example of auxiliary field}
Let us consider a simple case of auxiliary particle. 
We will show 
\begin{framed}
\bea
:\exp\left(-\frac{c\alpha_t}{2}\rho^2\right):=\sqrt{\frac{1}{2\pi}}\int_{-\infty}^\infty ds : \exp\left(-\frac{1}{2}s^2+\sqrt{-c\alpha_t} s\rho\right):
\eea 
\end{framed} 
This can be proved from eq. \eqref{eq:gaussian_rhov}, with $D=1$, $A^{-1}=-c\alpha_t$
and rename $v\to \sqrt{-c\alpha_t} s$,
\bea 
& &\int dv e^{-\frac{1}{2}v^2\frac{1}{-c\alpha_t}+\rho v}=
    \sqrt{-c\alpha_t}\int ds e^{-\frac{1}{2}s^2+\sqrt{-c\alpha_t}\rho s} \no 
&=&(2\pi)^{1/2}e^{-\frac{1}{2}\ln(\frac{1}{-c\alpha_t})} e^{-\frac{c\alpha_t}{2}\rho^2}
  =(2\pi)^{1/2}\sqrt{-c\alpha_t}e^{-\frac{c\alpha_t}{2}\rho^2}
\eea 
where $e^{-\frac{1}{2}\ln(\frac{1}{-c\alpha_t})}=\sqrt{-c\alpha_t}$. 


\subsubsection{an example of auxiliary field for Non-local interaction} 

Let us consider a non-local interaction, with  $f_{s_L}(\vn'-\vn)=f_{s_L}(|\vn'-\vn|)  $,
\bea 
& &\exp(-\frac{c_0\alpha_t}{2}\sum_{n',n,n''} :\rho_{NL}(n')f_{s_L}(n'-n)f_{s_L}(n-n'')\rho_{NL}(n''):) \no 
&=&:\exp(- \frac{c_0\alpha_t}{2}\sum_{n}\left(\sum_{n'} \rho_{NL}(n')f_{s_L}(n'-n)\right)\left(\sum_{n''}f_{s_L}(n-n'')\rho_{NL}(n'')\right) ): \no 
&=& :\exp(-\frac{c_0\alpha_t}{2}\sum_{n}\tilde{\rho}_{NL}(n)\tilde{\rho}_{NL}(n) ):
\eea 
From eq. \eqref{eq:gaussian_rhov} , with $A^{-1}(n,n')=-c_0\delta_{nn'}$, 
\bea 
& &\exp(-\frac{c_0\alpha_t}{2}\sum_{n',n,n''} :\rho_{NL}(n')f_{s_L}(n'-n)f_{s_L}(n-n'')\rho_{NL}(n''):) \no 
&\propto&\int [d s] \exp\left(-\frac{1}{2}\sum_{n} s(n)^2+\sqrt{-c_0\alpha_t}\sum_{n n'}\rho_{NL}(n)f_{s_L}(n-n')s(n')\right)
\eea 

Note here that there is no $\alpha_t$ in kinetic part of auxiliary action.(
With scale $s(\vn)\to \sqrt{\alpha_t}s(\vn)$, we get $\alpha_t$ factor in all action.) 

Also note the sign of the coupling gives potential,
\bea 
V_{s}={\color{red}-}\sqrt{-c_0\alpha_t}\sum_{n n'}\rho_{NL}(n)f_{s_L}(n-n')s(n').
\eea 
This has opposite sign with the paper , PRL119,222505. I think PRL119,222505 have a typo.   

\subsubsection{an example of auxiliary field for one-pion exchange}

Consider a pion field and its derivative $\nabla_S \pi_I(\vn)$ using DFT,
\bea 
\pi_I(\vn)&=&\frac{1}{L^3}\sum_\vq e^{i\vq\cdot\vn}\hat{\pi}(\vq), \quad 
\hat{\pi}(\vq)=\sum_{\vn} e^{-i\vq\cdot\vn}\pi_I(\vn), \no 
\nabla_S \pi_I(\vn) &=& \frac{1}{L^3}\sum_\vq e^{i\vq\cdot\vn}(i\vq)_S \hat{\pi}(\vq)
\eea 

One may introduce a function (``qhop" function $\Delta_S(\vn)$ or $f_S^\pi(\vn)$   ) such that 
\bea 
\nabla_S \pi_I(\vn) &=& \frac{1}{L^3}\sum_\vq e^{i\vq\cdot\vn}(i\vq)_S 
                        \sum_{\vn'} e^{-i\vq\cdot\vn'}\pi_I(\vn') 
                    =  \frac{1}{L^3}\sum_{\vn'} \pi_I(\vn') 
                    \sum_\vq e^{i\vq\cdot(\vn-\vn')}(i\vq)_S \no 
                    &=& \sum_{\vn'} \pi_I(\vn')\Delta_S(\vn'-\vn)=\sum_{\vn'} \pi_I(\vn+\vn')\Delta_S(\vn'),\no 
\Delta_S(\vn-\vn')&=&  \frac{1}{L^3} \sum_\vq e^{-i\vq\cdot(\vn-\vn')}(i\vq)_S 
                   =  -\Delta_S(\vn'-\vn).
\eea 
This function can be further simplified in cubic lattice as 1-D function,
\bea 
\Delta_S(\vn-\vn')=\delta_{n_i,n'_i}\delta_{n_j,n'_j}  \Delta_S(n_S-n'_S),\quad (i,j)\neq S.
\eea 


Thus, we can write the $\pi N$ coupling term,
\bea 
V_{\pi N}&=&\frac{g_A}{2f_\pi}\sum_\vn\sum_{SI} \nabla_S \pi_I(\vn)\rho_{SI}(\vn) \no 
         &=& \frac{g_A}{2f_\pi}\sum_{\vn,\vn'}\sum_{SI} \pi_I(\vn')f_S(\vn'-\vn)\rho_{SI}(\vn) \no 
         &=& {\color{red}-}\frac{g_A}{2f_\pi}\sum_{\vn,\vn'}\sum_{SI} \rho_{SI}(\vn) f_S(\vn-\vn') \pi_I(\vn')
\eea 

For the free pion action, we may introduce $f_{\pi\pi}(\vn)$,
\bea 
f_{\pi\pi}(\vn,\vn')=\frac{1}{L^3}\sum_\vq e^{i\vq\cdot(\vn-\vn')} \hat{f}_{\pi\pi}(\vq)=
                \frac{1}{L^3}\sum_\vq e^{i\vq\cdot(\vn-\vn')} \exp(b_\pi \vq^2)(\vq^2+m_\pi^2)
\eea 
and its inverse matrix
\bea 
f^{-1}_{\pi\pi}(\vn,\vn')=\frac{1}{L^3}\sum_\vq e^{i\vq\cdot(\vn-\vn')} \frac{\exp(-b_\pi \vq^2)}{\vq^2+m_\pi^2}.
\eea 
Thus, the Path integral with pion and coupling becomes, (Note that the sign change according to the 
arguments of pion and qhop function.)\footnote{
The numerical code uses the first form when computing pion derivative using qhop function.}
\bea  
I&\equiv&\int [d\pi] \exp\left(-S_{\pi\pi}-S_{\pi N}\right) \no 
 &=&  \int [d\pi] \exp\left(-\frac{1}{2}\sum_{\vn,\vn'}\pi_I(\vn)f_{\pi\pi}(\vn-\vn')\pi_I(\vn')
{\color{red} -}\frac{g_A}{2f_\pi}\sum_{\vn,\vn'}\sum_{SI} \pi_I(\vn') f_S(\vn'-\vn)  \rho_{SI}(\vn) 
\right) \no 
  &=& \int [d\pi] \exp\left(-\frac{1}{2}\sum_{\vn,\vn'}\pi_I(\vn)f_{\pi\pi}(\vn-\vn')\pi_I(\vn')
{\color{red} +}\frac{g_A}{2f_\pi}\sum_{\vn,\vn'}\sum_{SI}\rho_{SI}(\vn)f_S(\vn-\vn')\pi_I(\vn')   
\right)
\eea 

Then from analogy with eq. \eqref{eq:gaussian_rhov}, $v\to \pi_I$, $A\to f_{\pi\pi}$,
\bea 
& &-\frac{1}{2}  v^T A v\to -\frac{1}{2} \sum_{\vn,\vn'} \pi_I(\vn) f_{\pi\pi}(\vn,\vn') \pi_I(\vn'),\no 
& &+\rho^T v \to +\sum_{\vn,\vn'} \frac{g_A}{2f_\pi} \rho_{SI}(\vn) f_S(\vn-\vn')\pi_I(\vn') 
\eea 
Then,
\bea 
I &=& (2\pi)^{L^3/2} \exp\left(-\frac{1}{2}{\rm Tr}\ln f_{\pi\pi}\right) 
       \no & &\quad \times 
      	  \exp\left( \frac{1}{2} \left(\frac{g_A}{2f_\pi}\right)^2 \sum_{\vn_1\vn_2,\vn',\vn''}
      	     \sum_{S_1,S_2,I}     	    
      	        \rho_{S_1 I}(\vn')f_{S_1}(\vn',\vn_1)f^{-1}_{\pi\pi}(\vn_1-\vn_2)
      	  	   f^T_{S_2}(\vn_2,\vn'')\rho_{S_2 I}(\vn'') \right) \nonumber 
\eea 

By doing sum over $\vn_1,\vn_2$, we can define
\bea 
f_{S_1 S_2}(\vn'-\vn'')&\equiv &\sum_{\vn_1,\vn_2} f_{S_1}(\vn',\vn_1)f^{-1}_{\pi\pi}(\vn_1-\vn_2)
f^T_{S_2}(\vn_2,\vn'') \no 
&=& \left(\frac{1}{L^3}\right)^3 \sum_{\vn_1,\vn_2}\sum_{\vq_1\vq \vq_2} 
    e^{i\vq_1\cdot(\vn'-\vn_1)} (-i\vq_1)_{S_1}
    e^{i\vq\cdot(\vn_1-\vn_2)} \frac{e^{-b_\pi q^2}}{(\vq^2+m_\pi^2)} 
    e^{-i\vq_2\cdot(\vn_2-\vn'')} (-i\vq_2)_{S_2}
\no 
&=& \frac{1}{L^3}\sum_{\vq} e^{i\vq\cdot(\vn'-\vn'')} \frac{q_{S_1}q_{S_2} e^{-b_\pi q^2}}{(\vq^2+m_\pi^2)}
    =\frac{1}{L^3}\sum_{\vq} e^{-i\vq\cdot(\vn'-\vn'')} \frac{q_{S_1}q_{S_2} e^{-b_\pi q^2}}{(\vq^2+m_\pi^2)}
\eea 
Thus, we get
\bea 
I\propto \exp\left(+\frac{1}{2}\left(\frac{g_A}{2f_\pi}\right)^2\sum_{\vn',\vn,S,S',I} 
              \rho_{SI}(\vn') f_{SS'}(\vn'-\vn) \rho_{S' I}(\vn)\right)
\eea 


Thus, we get the OPE potential 
\bea 
V_{OPE}&=&-\frac{g_A^2}{8f_\pi^2}\sum_{n'n,S',S,I} : \rho_{S',I}(n')f_{SS'}(n'-n)\rho_{S,I}(n):
\eea  
Note that this has different sign with paper, PRL119,222505. 
I think PRL119,222505 have typo by mistaking order of $\pi_I$ field and qhop function.   

\subsection{Example: analytic calculation of quadratic functional  }  
For free scalar field, we have the functional as,
\bea 
W^E_0[J]&=&{\cal N}_E \int {\cal D}\phi e^{-\frac{1}{2}(\phi,A\phi)+(\rho,\phi)}\no 
        &=&{\cal N}_E(\det A)^{-\frac{1}{2}} e^{\frac{1}{2}(\rho,A^{-1} \rho)}
\eea 
where,
\footnote{In a similar way, Grassmann integration
\bea 
\int {\cal D}v^* {\cal D}v e^{ (v^*, A v)+(v^*,\eta)+(\eta^*,v)}
  =[\det A]e^{-(\eta^*, A^{-1}\eta)}
\eea 

}
\bea
{\cal L}_E&=&-\frac{1}{2}\left(\del^E_\mu\phi\del^E_\mu\phi+m^2\phi^2\right)
           =-\frac{1}{2}\phi\left(-\del^2_E+m^2\right)\phi
 ,\no 
A(x_E,x'_E)&=&(-\del_E^2 +m^2)\delta^4(x'_E-x_E), \quad \rho(x_E)=J(x_E).
\eea 

If matrix $A(x,x')$ only depends on the difference, we can Fourier transform,
\bea 
A(x,x')=\int \frac{dk}{2\pi} e^{-ik(x'-x)}A(k).
\eea 
In a similar way, any function of matrix A, only depending on the difference,
 also can be F.T. ,
\bea 
f(A)(x,x')=\int \frac{dk}{2\pi} e^{-ik(x'-x)} f(A(k))
\eea 
In other words,
with $A(p_E)=(p_E^2+m^2)$,(here $p_E$ are 4-dimensional)
\bea 
{\rm Tr}\ {\ln A(x,x')}&=&\int d^4 x_E (\ln A)(x_E,x_E)
       =\int d^4 x_E \int \frac{d^4 p_E}{(2\pi)^4}(\ln A(p_E)) \no 
       &=&\int d^4 x_E \int \frac{d^4 p_E}{(2\pi)^4}\ln (p_E^2+m^2) 
\eea 
and 
\bea 
A^{-1}(x_E,x'_E)&=&\int \frac{d^4 p_E}{(2\pi)^4} e^{-ip_E\cdot(x'_E-x_E)} A^{-1}(p_E)
                =\int \frac{d^4 p_E}{(2\pi)^4} e^{-ip_E\cdot(x'_E-x_E)}\frac{1}{p_E^2+m^2} \no 
               &=&\Delta_F^E(x'_E-x_E)              
\eea 
This implies that free two point Green's function, in Minkowski space,
\bea 
G_0^{(2)}(x_1,x_2)&=&(\frac{\hbar}{i})^2 \frac{\delta^2 W_0[J]}{\delta J(x_1)\delta J(x_2)}|_{J=0}
   =i\hbar\Delta_F(x_1-x_2)\no 
   &=&i\hbar A^{-1}(x_1,x_2)=\int \frac{d^4 p}{(2\pi)^4} e^{-ip\cdot(x'-x)}\frac{i}{p^2-m^2+i\epsilon}
\eea 
Also, we can change 
\bea 
e^{\frac{1}{2}(\rho,A^{-1}\rho)}=(\det A)^{\frac{1}{2}}
 \int {\cal D}\phi\ e^{-\frac{1}{2}(\phi,A\phi)+(\rho,\phi)}
\eea 
For example, we can rewrite the original action into auxiliary formalism form as,
\bea 
& & Z\propto \int{\cal D}\bar{N}{\cal D}N e^{\int d^4x {\cal L}_E(\bar{N},N)},\no 
& &{\cal L}_E(\bar{N},N)=-\bar{N}[\del_4-\frac{\nabla^2}{2m}+(m-\mu)]N
           -\frac{1}{2}C_1 \bar{N}N\bar{N}N
           -\frac{1}{2}C_2\bar{N}{\vec \tau} N\cdot\bar{N}{\vec \tau} N,\no 
 &\rightarrow& 
     Z \propto \int{\cal D}\bar{N}{\cal D}N{\cal D}f{\cal D}{\vec\phi} 
      e^{\int d^4x {\cal L}(\bar{N},N,f,{\vec{\phi}})},\no
& & {\cal L}_E(\bar{N},N,f,{\vec{\phi}})
     =-\bar{N}[\del_4-\frac{\nabla^2}{2m}+(m-\mu)]N
      -\frac{1}{2}f\frac{1}{-C_1}f
      +\bar{N}N f
      -\frac{1}{2}{\vec{\phi}}\frac{1}{-C_2}{\vec{\phi}}
      +\bar{N}{\vec \tau} N\cdot{\vec{\phi}}\no 
  & &\simeq -\bar{N}[\del_4-\frac{\nabla^2}{2m}+(m-\mu)]N
        -\frac{1}{2}f' f'    +\sqrt{-C_1} \bar{N}N f'
        -\frac{1}{2}{\vec{\phi}'}\cdot{\vec{\phi}'}
        +\sqrt{-C_2}\bar{N}{\vec \tau} N\cdot{\vec{\phi}'}
\eea 
\footnote{{\bf ( ref. nucl-th/0407088)}
Note here, if $C_1<0$, the interaction between f and Nucleon is real
and if $C_1>0$, the interaction between f and Nucleon is complex.
On the other hand, 
if $C_{2}>0$, the matrix $M=\exp({\cal L}_E\Delta_t)$ 
becomes real because $\tau_2 M\tau_2=M^*$ despite the appearance of $i$
. However, $C_2<0$ makes the 
M to be complex. 

Thus, to avoid sign problem it is better to use a form of
$C_2>0$. 
Thus, though formally equivalent,
$:\bar{N}N\bar{N}N:=-\frac{1}{2}:\bar{N}{\vec \sigma} N\cdot\bar{N}{\vec \sigma} N:
-\frac{1}{2}\bar{N}{\vec \tau} N\cdot\bar{N}{\vec \tau} N: $
, it is better to use a form
of 
\bea 
-\frac{1}{2}C_1 \bar{N}N\bar{N}N
           -\frac{1}{2}C_2\bar{N}{\vec \tau} N\cdot\bar{N}{\vec \tau} N
\eea 
instead of 
\bea 
-\frac{1}{2}C'_1 \bar{N}N\bar{N}N
           -\frac{1}{2}C'_2\bar{N}{\vec \sigma} N\cdot\bar{N}{\vec \sigma} N
\eea            
because $C'_1<0,C'_2<0$ but $C_2=-C'_2>0, C_1=C'_1-2 C'_2<0$.
} 

\chapter{Numerical method: Basic }

In case of fermion integration, the basic steps to compute the lattice action
and compute observables are as follows.
\begin{itemize}
	\item Euclidean action is converted into auxiliary action
	\bea
	Z=\int {\cal D}\psi^\dagger {\cal D}\psi e^{-S[\psi^\dagger,\psi]} 
	\to \int {\cal D}\phi {\cal D}\psi^\dagger {\cal D}\psi  \rho[\phi] e^{-S[\phi,\psi^\dagger,\psi]} 
	\eea 
	\item fermion action is integrated analytically,
	\bea 
	Z=\int {\cal D}\phi {\cal D}\psi^\dagger {\cal D}\psi \rho[\phi] e^{-S[\phi,\psi^\dagger,\psi]} 
	\to \int {\cal D}\phi \rho[\phi]\det K[\phi] =\int {\cal D}\phi P[\phi]
    \eea 
    	where, kinetic and interaction terms in action may be written as
    \bea 
    S[\phi,\psi^\dagger,\psi]=\frac{1}{b_\tau}\sum_{\tau,\tau'} \psi^\dagger_{\tau'}[K(\phi)]_{\tau',\tau}\psi_\tau 
    \eea 
    
    \item observable can be approximated as
    \bea 
    \la {\cal O}\ra =\frac{1}{Z}\int {\cal D}\phi P[\phi] {\cal O}[\phi]
    \to \frac{1}{N_{cf}}\sum_{n}^{N_{cf}} {\cal O}(\phi_n)
    \eea 
    where, configurations of $\phi_n$ is selected according to the probability $P[\phi]$.
    This selection or update of $\phi_n$ can be done by {\bf determinantal Monte Carlo}(Metropolis sampling),
    or {\bf Hybrid Monte Carlo} method.
    \item one way to compute the determinant or compute integration over Grassman number 
          is to use transfer matrix.  
\end{itemize}



Basic numerical technique to compute path integral is Importance sampling Monte 
Carlo integration, or Metropolis algorithm, 
\bea
\la O\ra&=&\int D[U] P[U] O[U],\mbox{ with } 
         P[U]=\frac{e^{-S[U]}}{Z},\quad 
        Z=\int D[U] e^{-S[U]}\no
        &\simeq& \frac{1}{N}\sum_{U_n \mbox{ with probability } P[U_n] }^N O[U_n]  
\eea

In general,
\bea 
\la\la \Gamma[x]\ra\ra=\frac{\int {\cal D}x(t)\ \Gamma[x] e^{-S[x]}}{\int {\cal D}x(t)\ e^{-S[x]}}
\eea 
With {\bf $N_{cf}$ number of random paths or configurations}, $x^{(\alpha)}$,  
if the configuration is chosen with probability $P[x^{(\alpha)}]\propto e^{-S[x^{(\alpha)}]}$,
above integral can be expressed as unweighted average over path,
\bea 
\la\la \Gamma[x]\ra\ra\simeq 
\bar{\Gamma}\equiv \frac{1}{N_{cf}}\sum_{\alpha=1}^{N_{cf}}\Gamma[x^{(\alpha)}].
\eea 
$\bar{\Gamma}$ is a MC estimation of integral. Uncertainty in MC estimation can be thought as
\bea 
\sigma^2_{\bar{\Gamma}}\simeq 
\frac{1}{N_{cf}}\left\{\frac{1}{N_{cf}}\sum_{\alpha=1}^{N_{cf}}\Gamma^2[x^{(\alpha)}]
         -\bar{\Gamma}^2     \right\}
\eea 

Updating path can be done by using Markov process in metropolis algorithm. 
Instead of updating path as a whole, it updates paths at each sites 
by the difference in action at each sites. Algorithm for randomizing $x_j$ at j-th site is :
\begin{itemize}
\item  generate a random number $\zeta $ with probability uniformly distributed between $-\epsilon$
      and $+\epsilon$.   
\item  replace $x_j\to x_j+\zeta$ and compute the change $\Delta S$ in the action
      caused by this replacement. $\Delta S=S_{new}-S_{old}$
\item  if $\Delta S<0$, retain new value for $x_j$ and proceed to next site.
\item if $\Delta S>0$, generate random number $\eta$ uniformly distributed between 0 and 1;
      retain the new value $x_j$ if $\exp(-\Delta S)>\eta $, otherwise restore
      old value; 
\item proceed to the next site. If this is done for all sites, it is called as one
      {\bf sweep}. A random walk parameter $\epsilon$ have to be tuned so that 
      the acceptance rate around $50\%$ for one sweep.
\item However, not every sweep will be used in the calculation. 
     To remove correlation between sweeps, only one path per every $N_{cor}$ sweep
     will be used in the calculation. The optimal value of $N_{cor}$ depends on 
     lattice spacing, $N_{cor}\propto 1/a^2$.             
\end{itemize} 

The procedure to start the algorithm, we need thermalization. 

\begin{itemize}
\item Initialization Path
\item Update $5 N_{corr}-10N_{corr}$ times to thermalize it.
\item Update the path $N_{corr}$ times, then compute $\Gamma[x]$ and save it;
      repeat $N_{cf}$ times.
\item average $N_{cf}$ values of $\Gamma[x]$ saved in the previous step
      to obtain Monte Carlo estimate $\bar{\Gamma}$ for $\la\la\Gamma[x]\ra\ra$. 
      Compute statistical error.       
\end{itemize}

To compute {\bf statistical error} without too large $N_{cf}$, we can use 'boot strap' 
and 'binning' method. 
Suppose we obtained $N_{cf}$ values of $\{\Gamma^{(\alpha)},\alpha=1\dots N_{cf} \}$.
Then, 
\begin{itemize}
\item bootstrap: generate $N_{bts}$ copies of $\Gamma^{(\alpha)}$'s by selecting random
      configurations while allowing duplication and omission. Then compute statistical error 
      from each bootstrap copies to compute uncertainty. This is only for statistical error estimates,
      not average estimates. 
            
\item Binning: When $N_{cf}$ is very large, it is difficult to store all results, $\Gamma^{(\alpha)}$'s
      Instead, we can store average of  $\Gamma^{(\alpha)}$'s with several number of configurations,
      $\bar{\Gamma}^{(\beta)}=\frac{1}{N_{bin}}\sum_{n=1}^{N_{bin}} \Gamma^{(\alpha)_n}$. The final 
      results can be stored as $50-100\bar{\Gamma}^{(\beta)}$ results. 
       
\end{itemize}

\section{A simple example of MC calculation}
{\color{blue} This example shows how the world line method can be used to compute path integral.}

{\bf Problem:} Suppose there exists three possible states $|1\ra,|2\ra,|3\ra$ and the transition probability 
between them at finite time interval is known, $M_{ij}$ as
\bea 
M=\threedmat{1.1}{0.1}{0.1}{0.1}{0.8}{0.1}{0.1}{0.1}{0.8},
\eea  
$f(n)\equiv\la 1| M^n|1\ra$ is a probability of getting state $|1\ra$ after n-time interval started from $|1\ra$ state. Compute $\frac{f(20)}{f(19)}$ by using Metropolis Monte Carlo.

Here $M$ corresponds to the transfer matrix $e^{-H \Delta t}$.
Note that this example does not involve lattice representation of states.
The exact solution can be obtained easily by doing matrix product.
But, how can we do this with Monte Carlo? 

We may consider many possible way way to reach state $|1\ra$.
For example, series of matrix elements $M_{12} M_{22}\dots M_{11}M_{12} M_{21}$ 
represent one possible path(or configuration). 
If we sum over every possible configuration, we get exact result. 
In MC, we sample only several possible configuration randomly. 
\bea 
f(n)&=&\sum_{i_1 i_2\dots i_{n}=1} M_{1 i_{n-1}} M_{i_{n-1} i_{n-2}}\dots M_{i_1 1} \no 
    &=&\sum_{\mbox{all path}} [M\cdots M]_{path} 
\eea 
Here total number of possible path is $3^{(n-1)}$.
Let us denote product for a path $c$ for n-steps as $W^{(n)}[c]$ then,
\bea 
f(n)=\sum_{c} W^{(n)}[c]
\eea 

If we choose path $c$ randomly every time, it is a simple Monte Carlo. 
If we update new path $c_{i+1}$ from previous path $c_{i}$ according,
with a certain probability, it is a Markov chain Monte Carlo. 

Suppose old path gives $W^{(20)}(old)$ and new choice of path gives $W^{(20)}(new)$.
By choosing random number $r$ and accept the update when $r \leq \frac{W^{(20)}(new)}{W^{20}(old)}$.
While doing this, one also can keep track of $W^{(19)}$.

If the state at step 19 is $k(k=1,2,3)$ for one path, then
\bea 
\frac{W^{(19)}}{W^{(20)}}=\left\{ \begin{array}{cr} 1/M_{11} & \mbox{if k=1} \\
                                                       0     & \mbox{if k=2} \\ 
                                                       0     & \mbox{if k=3} \end{array} \right.  
\eea 
Because $W^{19}=0$ if state at step 19 is not $k=1$. 
This acts as an observable(or operator) to be averaged.

Average $\frac{W^{(19)}}{W^{(20)}}$ over all twenty-step paths
gives $\frac{f(19)}{f(20)}$. If we sample path according to $ W^{(20)}$, 
\bea 
\frac{f(19)}{f(20)}=\frac{\sum_{paths}  W^{(20)} \left(\frac{W^{(19)}}{W^{(20)}}\right)}{\sum_{paths} W^{(20)}} 
   = \frac{\sum_{sampling} \left(\frac{W^{(19)}}{W^{(20)}}\right)}{\sum_{sampling} 1}  
\eea 

\chapter{Fermion Path integral: Transfer Matrix formalism}
How one can treat Fermion in path integral? As we can see later,
the path integral of fermion field requires Grassmann variables.
But, it would be difficult to realize Grassmann number integration numerically.
Then how one can proceed? Let us separate the average of any operator into
fermion part and boson part.
\bea 
\la O \ra =\la \ \la O\ra_F \ra_G.
\eea 
And fermion part 
\bea 
\la A\ra_F&=&\frac{1}{Z_F[U]}\int {\cal D}[\psi,\bar{\psi}]\ e^{-S_F[\psi,\bar{\psi},U]}
                    A[\psi,\bar{\psi},U] \no 
Z_F[U]&=&\int {\cal D}[\psi,\bar{\psi}]\ e^{-S_F[\psi,\bar{\psi},U]}
\eea 
boson part
\bea 
\la B\ra_G=\frac{1}{Z}\int {\cal D}[U]\ e^{-S_G[U]} Z_F[U] B[U]
\eea 
In practice, computation of $Z_F[U]$ or $\la O\ra _F$ requires analytic methods
or transfer matrix formalism. Once they are calculated, one can numerically 
compute bosonic path integral for $\la \la O\ra_F \ra_G$. 

\section{Grassmann integral}
The Grassmann variables anti-commutes with each other,
fermion field operators satisfy the equal time anti-commutation relations.
Thus, the correspondence between fermion field and Grassmann variables
are not obvious. We might introduce path integral over Grasmann variables
in several different ways

Fermion operators satisfy equaltime anti-commutation relation,
\bea
\{\hat{a}_n,\hat{a}^\dagger_m \}=\delta_{nm},
 \quad \{\hat{a}_n,\hat{a}_m\}=\{\hat{a}_n^\dagger,\hat{a}_m^\dagger\}=0. 
\eea
On the other hand, we introduce Grasmann numbers such that
that Grassmann number $\chi$, $\chi^\dagger$ anti-commutes with each others.
\bea
\{ \chi_i,\chi_j^\dagger \}_{+}=\{ \chi_i,\chi_j \}_{+}=
\{ \chi_i^\dagger,\chi_j^\dagger \}_{+}=0.
\eea
Then, Grasman integration is defined as
\bea
\int d\chi\ 1=0,\quad \int d\chi\ \chi =1,\quad d\eta_i d\eta_j=-d\eta_j d\eta_i
\eea
Several basic integrals are
\bea 
\eta'_i=\sum_{j=1}^N M_{ij} \eta_j \quad \to\quad  d^N \eta=\det[M] d^N\eta'
\eea 

\bea 
\int d\eta_N d\bar{\eta}_N\dots d\eta_1 d\bar{\eta}_1 \exp\left(
   \sum_{i,j=1}^N \bar{\eta}_i M_{ij}\eta_j\right) =\det[M].
\eea
\bea 
& &\int \prod_{i=1}^N d\eta_i d\bar{\eta}_i
   \exp\left(\sum_{k,l=1}^N \bar{\eta}_k M_{kl}\eta_l
    +\sum_{k=1}^N \bar{\theta}_k\eta_k+\sum_{k=1}^N \bar{\eta}_k\theta_k\right) \no 
&=&\det[M]\exp\left(-\sum_{n,m=1}^N \bar{\theta}_n(M^{-1})_{nm}\theta_m   \right).   
\eea 
\bea 
\la \eta_{i_1}\bar{\eta}_{j_1}\dots\eta_{i_n}\bar{\eta}_{j_n}\ra_F
&=&\frac{1}{Z_F}\int [\prod_{k=1}^N d\eta_k d\bar{\eta}_k]\ 
  \eta_{i_1}\bar{\eta}_{j_1}\dots\eta_{i_n}\bar{\eta}_{j_n}
  \exp\left(\sum_{l,m=1}^N \bar{\eta}_l M_{lm}\eta_m\right) \no 
&=&(-1)^n\sum_{P(1,2,\dots,n)} {\rm sign}(P)  
 (M^{-1})_{i_1 j_{P_1}}(M^{-1})_{i_2 j_{P_2}}\dots (M^{-1})_{i_n j_{P_n}}
\eea 

As we can see, we often need the inverse of matrix $M$, $M^{-1}$. 
When $M_{nm}$ is a position space matrix, it is convenient to use its 
F.T. expression , $M_{pq}$ in momentum space because this matrix is diagonal 
in many case. So, $M_{pq}=\delta_{pq}D(p)$ and inverse can be easily
obtainable $M^{-1}_{pq}=\delta_{pq}D^{-1}(p)$. Then we can obtain
$(M^{-1})_{nm}$ by Fourier transformation of $M^{-1}_{pq}$.
(Fermion doubling occurs when the $D^{-1}(p)$ have additional poles 
because of lattice Fourier transformation. For example, 
$D^{-1}(p)\sim 1/\sin(pa)^2$ have physical pole $p=0$ and unphysical poles
$p=n\pi a$.   )



\section{Anti-periodicity of Fermion field}
The periodic boundary condition in time is necessary in path integral,
because we are interested in the trace of operators.
In case of scalar field, 
it is easy to see why we impose periodic boundary condition,
$\phi(T)=\phi(0)$. However, it is not easy to see why anti-periodic boundary condition 
is necessary for fermion.
On the other hand, in path integral formalism, for fermion states,
$|0\ra $ and $|1\ra=a^\dagger|0\ra$, trace means 
\bea 
\mbox{Tr } \hat{O}=\la 0|\hat{O}|0\ra+\la 1|\hat{O}|1\ra
\eea  

One way to see is to compute thermal average of correlators. Thermal average 
is defined as
\bea
\la \hat{A}\ra_\beta ={\rm Tr }\ e^{-\beta \hat{H}}\hat{A}, \quad 
\mbox{ with nomarlization } {\rm Tr}\ e^{-\beta\hat{H}}=1.  
\eea
Then, 
\bea
\la \hat{\psi}(x,\tau_1)\hat{\psi}(y,0)\ra_\beta
&=&{\rm Tr}[e^{-\beta\hat{H}}\hat{\psi}(x,\tau_1)\hat{\psi}(y,0)]
  ={\rm Tr}[\hat{\psi}(y,0) e^{-\beta\hat{H}}\hat{\psi}(x,\tau_1)] \no 
&=&{\rm Tr}[e^{-\beta\hat{H}}e^{\beta\hat{H}}\hat{\psi}(y,0) 
            e^{-\beta\hat{H}}\hat{\psi}(x,\tau_1)] \no
&=&{\rm Tr}[e^{-\beta\hat{H}}\hat{\psi}(y,\beta) 
            \hat{\psi}(x,\tau_1)] \no
&=&\la \hat{\psi}(y,\beta)\hat{\psi}(x,\tau_1)\ra_\beta \no 
&=&\pm \la \hat{\psi}(x,\tau_1)\hat{\psi}(y,\beta)\ra_\beta               
\eea
Thus, we impose $\hat{\psi}(y,\beta)=\pm \hat{\psi}(y,0)$ for $bosonic$
and fermionic field operators. This is also the reason 
for introducing Matsubara frequency in thermal theory because the
two particle Green's function is periodic(anti-periodic).
So the frequency have to be 
$\omega_n=\frac{ 2n\pi}{\beta}$ for boson and
$\omega_n=\frac{(2n+1)\pi}{\beta}$ for fermion. 




\section{Coherent state formalism}
First, let us define coherent states
\bea
|\psi\ra&\equiv& e^{-\psi \hat{a}^\dagger} |0\ra
         =|0\ra -\psi |1\ra.
\eea
We may assume Grasman number and fermion operators anti-commutes,
\bea
\{\chi, \hat{a}\}_{+}&=&\{\chi, \hat{a}^\dagger\}_{+}=0,\no   
(\psi \hat{a}^\dagger)^\dagger&=& \hat{a}\psi^*,\no 
|1\ra \psi &=& \hat{a}^\dagger|0\ra \psi= -\psi|1\ra,\no 
\hat{a}|\psi\ra&=& \psi|\psi\ra. 
\eea
Then, we define bra-states as
\bea                      
\la \phi|&\equiv& \la 0| e^{-\hat{a}\phi^*}
         =\la 0|+\phi^*\la 1|=(|\phi\ra)^\dagger             
\eea 
Then, the scalar product of coherent states becomes
\bea
\la \psi|\phi\ra &=&(\la 0|+\psi^*\la 1|)(|0\ra -\phi |1\ra)
                  =1-\psi^*\la 1|\phi |1\ra
                  =1+\psi^*\phi=e^{\psi^*\phi}  
\eea
We can see coherent states are not normlaized. 
So completeness
relation requires compensation factor,
\bea
\int d\psi^* d\psi\ |\psi\ra e^{-\psi^*\psi} \la\psi|=I.
\eea
where factors $e^{-\psi^*\psi}$ 
enters as a kind of measure in Grasmann integrations. We can check above 
completeness relation,
\bea 
\int d\psi^* d\psi e^{-\psi^*\psi}|\psi\ra \la \psi|\psi'\ra 
&=& \int d\psi^* d\psi e^{-\psi^*\psi}e^{\psi^*\psi'} |\psi\ra 
  = \int d\psi^* d\psi e^{-\psi^*(\psi-\psi')} |\psi\ra \no 
  &=& \int d\psi \delta(\psi-\psi')|\psi\ra
  =|\psi'\ra 
\eea 
where the delta function is defined as
\bea
\delta(\psi-\psi')
= \int d\chi e^{\chi(\psi-\psi')} 
\eea
\footnote{
Careful that, in case of many particle states, 
\bea
e^{-\sum \chi_i}=\sum_n \frac{1}{n!}(-\sum\chi_i)^n \neq 1-\sum_i \chi_i
\eea 
}
Then, let us consider matrix elements for bosonic operator $\hat{O}$,
\bea
\la \eta|\hat{O}|\chi\ra
&=&(\la 0|+\eta^*\la 1|)\hat{O}(|0\ra-\chi|1\ra)
=\la 0|\hat{O}|0\ra+\eta^*\la 1|\hat{O}|0\ra
       -\la 0|\hat{O}\chi|1\ra -\eta^*\la 1|\hat{O}\chi|1\ra \no 
&=&\la 0|\hat{O}|0\ra+\eta^* \chi\la 1|\hat{O}|1\ra.                      
\eea
\footnote{
If $\hat{O}=c_0+c_2 \hat{a}^\dagger \hat{a}$, then
$\la 0|\hat{O}|0\ra=c_0$ and $\la 1|\hat{O}|1\ra=c_0+c_2$ and
\bea
\la \eta|\hat{O}|\eta'\ra=c_0+\eta^*\eta'(c_0+c_2)
      = (1+\eta^*\eta')c_0+\eta^*\eta' c_2
      =e^{+\eta^*\eta'}(c_0+\eta^*\eta' c_2)
\eea
}
Note that $\la \eta|\hat{O}|\chi\ra$ is not a c-number.
Then, we can see that
\bea
\int d\eta^* d\eta \la \eta|\hat{O}|\eta\ra e^{-\eta^*\eta}
    &=& \la 0|\hat{O}|0\ra -\la 1|\hat{O}|1\ra,\no 
\int d\eta^* d\eta \la -\eta|\hat{O}|\eta\ra e^{-\eta^*\eta}
    &=& \la 0|\hat{O}|0\ra +\la 1|\hat{O}|1\ra    
\eea
Thus, the Euclidean partition function in fermion path integral
have to be defined as
\bea
\la \hat{O}\ra_T={\rm Tr}\ e^{-T\hat{H}}\hat{O}
=\int d\eta^* d\eta \la -\eta| e^{-T\hat{H}} \hat{O}|\eta\ra e^{-\eta^*\eta}. 
\eea
This implies that we need anti-periodic-time boundary condition
for Grasmann variable. $\eta(T)=-\eta(0)$.

\section{Transfer matrix formalism 1}
Ref: M.L\"{u}scher, Commun.math.Phys. 54(1977)283. (Is this reference correct? 
Probably, it have to be replaced with M.Creutz, Phys.Rev.D38(1988)1228 .)

As similar as coherent states, we can use isomorphism between
Fock space vectors 
$$|k_1,k_2,\dots k_j\ra
=\hat{a}^\dagger_{k_1}\hat{a}^\dagger_{k_2}\dots\hat{a}^\dagger_{k_j}|0\ra \leftrightarrow \chi_{k_1}^* \chi_{k_2}^*\dots \chi_{k_j}^*.$$
From now on,
let us use $\chi$ represent set of Grassman numbers $\{\chi_1,\chi_2\dots\}$.
And represent general Fock space vector 
$|X\ra=X(\hat{a}^\dagger)|0\ra \leftrightarrow X(\chi^*)$.
\footnote{for example, 
$|0\ra\leftrightarrow 0,\ |k_1\ra\leftrightarrow \chi^*_{k_1}$.
}

Also use notation for short, for n-dimensional Grassmann numbers,
\bea 
\chi^*\eta &\equiv& \sum_{j}\chi_j^*\eta_j ,\no 
\int [d\eta]&\equiv& \int d\eta_n\dots d\eta_1,\quad  [d\eta^*][d\eta]=(-1)^n [d\eta][d\eta^*] \no 
\int [d\eta^* d\eta]&\equiv& \int d\eta^*_n d\eta_n\dots d\eta_1^* d\eta_1
  =(-1)^n \int [d\eta d\eta^*]
  =(-1)^{n-1}\int [d\eta^*][d\eta]
\eea 



Then, we wiil assoicitaes the scalar product $\la X|Y\ra$ as
\footnote{
\bea
\la k_1| k_1\ra&=&\int d\eta^*_1 d \eta_1  
   e^{-\eta^*_1 \eta_1 } (\eta_1^*)^* \eta_1^*,\no
\la k_1,k_2|k_1 k_2\ra &=&
   \int d\eta^*_1 d\eta^*_2 d\eta_1 d\eta_2 
   e^{-\eta^*_1\eta_1-\eta_2^*\eta_2} \eta_2\eta_1 \eta^*_1\eta^*_2      
\eea
}
\bea
\la X|Y\ra =\int d\eta^*_n d\eta_n \dots  d\eta^*_1 d\eta_1
            e^{-\sum_{j=1}^n \eta_j^* \eta_j}
            (X(\eta^*))^\dagger Y(\eta^*) 
\eea
Here, note that the integration with $e^{-\sum_{j=1}^n \eta_j^* \eta_j} $ factor 
is introduced to be consistent with normalization.(Unlike previous 
coherent formalism.)  



Representation of operators in Grassmann number can be associated from 
matrix elements of operators $\hat{A}(\hat{a}^\dagger,\hat{a})$ as
\bea 
A(k_1,\dots,k_j|l_1,\dots l_i)\equiv 
\la k_1,\dots,k_j|\hat{A}|l_1,\dots l_i\ra
=\la 0|\hat{a}_{k_j}\dots\hat{a}_{k_1}\hat{A}\hat{a}^\dagger_{l_1}\dots\hat{a}^\dagger_{l_i}|0\ra  
\eea 
\bea 
\bar{A}(\chi^*,\chi)\equiv 
 \sum_{\{k\},\{l\} } \frac{1}{j! i!} \chi_{k_1}^*\dots\chi_{k_j}^*
                     A(k_1,\dots,k_j|l_1,\dots l_i)\chi_{l_1}\dots\chi_{l_i}
\eea 

This definition of $A(\chi^*,\chi)$ satisfies
\bea 
(\bar{A}X)(\chi^*)=\int db_n^* db_n\dots db_1^* db_1 e^{-\sum_{j=1}^n b_j^* b_j}
             \bar{A}(\chi^*,b)X(b^*)  
\eea 
similar to integral kernel.

Product of two operators,
\bea 
(\bar{A}\cdot \bar{B})(\chi^*,\chi)
=\int db_n^* db_n\dots db_1^* db_1 e^{-\sum_{j=1}^n b_j^* b_j}
      \bar{A}(\chi^*,b)\bar{B}(b^*,\chi)
\eea 
And the trace of operator is given by
\bea
{\rm Tr}\ \bar{A} &=&\int d\chi^*_n\dots d\chi_1 e^{-\sum_{j=1}^n \chi^*_j\chi_j}
             \bar{A}(\chi^*,-\chi)\no  
{\rm Tr}\ \overline{AB}&=&
 \int [d\chi^* d\chi] e^{-\chi^*\chi} \overline{AB}(\chi^*,-\chi) \no 
            &=&\int[d\chi^* d\chi] e^{-\chi^*\chi}
               \int[d\eta^* d\eta] e^{-\eta^*\eta} \bar{A}(\chi^*,\eta)\bar{B}(\eta^*,-\chi)              
\eea

For example, 
(1) if there is only two states, $|0\ra$ and $|1\ra$,
\bea 
& &:\hat{A}:=c_0+c_2\hat{a}^\dagger\hat{a} \no 
&\leftrightarrow& 
\bar{A}(\chi^*,\chi)=c_0+(c_0+c_2)\chi^*\chi=e^{\chi^*\chi}(c_0+c_2\chi^*\chi)
\eea 
One can verify the equivalence ${\rm Tr}:\hat{A}:=\int d\chi^*_n\dots d\chi_1 e^{-\sum_{j=1}^n \chi^*_j\chi_j}
\bar{A}(\chi^*,-\chi)$.
In a similar way, (2) if there are four states $|0\ra,|1_1\ra,|1_2\ra,|1_1 1_2\ra$,
\bea 
& &:\hat{A}:=c_{00}+c_{20}\hat{a}_1^\dagger\hat{a}_1
+c_{02}\hat{a}_2^\dagger\hat{a}_2+c_{22}\hat{a}_1^\dagger\hat{a}_1\hat{a}_2^\dagger\hat{a}_2\no 
&\leftrightarrow& 
\bar{A}(\chi^*,\chi)=e^{\chi^*_1\chi_1+\chi^*_2\chi_2}
(c_{00}+c_{20}\chi^*_1\chi_1+c_{02}\chi^*_2\chi_2+c_{22}\chi^*_1\chi_1\chi^*_2\chi_2)
\eea

Another examples are  
\bea 
\hat{A}=e^{\sum_{k,l} \hat{a}^\dagger_k M_{kl} \hat{a}_l}  \quad  
\leftrightarrow \quad \bar{A}(\chi^*,\chi)=e^{\sum_{kl} \chi_k^* (e^M)_{kl} \chi_l},
\eea
\bea 
\hat{A}=e^{\sum_{k,l} \hat{a}^\dagger_k M_{kl} \hat{a}_l}
          e^{\sum_{k,l} \hat{a}^\dagger_k N_{kl} \hat{a}_l} \quad
\leftrightarrow \quad \bar{A}(\chi^*,\chi)=e^{\sum_{kl} \chi_k^* (e^M e^N)_{kl} \chi_l}.
\eea 

Under change of variables,
$\eta_k=\sum_l A_{kl} \chi_l$ and $\eta^*_k=\sum_l A^*_{kl} \chi^*_l$,
\bea
\int d\chi^*_n\dots d\chi_1 f(\chi,\chi^*)
    =|{\rm det} A|^2 \int d\eta^*_n\dots d\eta_1 f(A^{-1}\eta,(A^*)^{-1}\eta^*)
\eea  

\section{Transfer matrix formalism 2}
However, previous definition of $A$ operator is not convenient to use
because of additional factor. If we use convention 
Grassmann functional corresponding to 
operator as
\bea 
\hat{A}(\hat{a}^\dagger,\hat{a}) \leftrightarrow A(\chi^*,\chi)
\eea
Previous Grassmann functional becomes
\bea 
:\hat{A}(\hat{a}^\dagger,\hat{a}):\to 
\bar{A}(\chi^*,\chi)\to e^{\chi^*\chi} A(\chi^*,\chi)
\eea 
  
Then we can write 
\bea 
{\rm Tr}\ :A: &=&\int [d\chi^* d\chi]\ e^{-2\chi^*\chi}A(\chi^*,-\chi) \no  
            &=&\int [d\chi d\chi^*]\ e^{2 \chi^*\chi} A(\chi^*,\chi)      
\eea 
where, we changed variable and orders of $d\chi^* d\chi $ in second line. 

In other way, if there is a periodic condition $c(1)=-c(0)$, we can rewrite as
\begin{framed}
\bea 
{\rm Tr}\ :f(a,a^\dagger) : =\int [dc(0) dc^*(0)]\ e^{c^*(0)(c(0)-c(1))} A(c(0),c^* (0)) 
\eea 
\end{framed}

In a similar way, product of operators can be rewritten in a similar form .
\bea 
{\rm Tr}\ :A::B:&=&\int[d\chi^* d\chi] e^{-\chi^*\chi}
               \int[d\eta^* d\eta] e^{-\eta^*\eta} 
                                   e^{\chi^*\eta} e^{-\eta^*\chi} 
               A(\chi^*,\eta)B(\eta^*,-\chi) \no 
          &=&\int[d\chi^*] [d\chi] 
                 [d\eta^*] [d\eta] e^{-\chi^*\chi} e^{-\eta^*\eta} 
                         e^{\chi^*\eta} e^{-\eta^*\chi} 
                         A(\chi^*,\eta)B(\eta^*,-\chi) \no
          &=&\int [d\eta] [d\chi^*][d\chi] 
                 [d\eta^*] e^{\chi^*\chi} e^{-\eta^*\eta} 
                            e^{\chi^*\eta} e^{\eta^*\chi} 
                             A(\chi^*,\eta)B(\eta^*,\chi).                       
\eea 
Let us rename $\chi^*\to \chi_1^*$, $\eta\to \chi_1$, $\eta^*\to\chi_2^*$
and $\chi\to \chi_2$. Then
\bea 
{\rm Tr}\ :A::B:&=&\int [d\chi_1 d\chi^*_1][d\chi_2d\chi^*_2] 
                    e^{\chi^*_1\chi_2} e^{-\chi^*_2\chi_1} 
                            e^{\chi^*_1\chi_1} e^{\chi_2^*\chi_2} 
              A(\chi^*_1,\chi_1)B(\chi^*_2,\chi_2)\no 
            &=&\int [d\chi_1 d\chi^*_1][d\chi_2d\chi^*_2] 
                   e^{\chi^*_1(\chi_1+ \chi_2)} 
                   e^{\chi^*_2(\chi_2-\chi_1)}
                          A(\chi^*_1,\chi_1)B(\chi^*_2,\chi_2)\no 
           &=&\int [d\chi_1 d\chi^*_1][d\chi_0d\chi^*_0] 
                              e^{\chi^*_1(\chi_1+ \chi_0)} 
                              e^{\chi^*_0(\chi_0-\chi_1)}
                                     A(\chi^*_1,\chi_1)B(\chi^*_0,\chi_0)                 
\eea 
Now the general relation between trace of operators
and Grassman path integral is 
\begin{framed}
\bea
& &{\rm Tr}\left\{ : F_{L_t-1}[a^\dagger({\bm n}'),a({\bm n}) ]:\times \cdots \times 
   :F_{0}[a^\dagger({\bm n}'),a({\bm n}) ]:\right\}  \no 
&=&\int Dc Dc^* \exp\left[\sum_{n_t=0}^{L_t-1}\sum_{\vn, i}
    c^*_i(\vn,n_t)[c_i(\vn,n_t)-c_i(\vn,n_t+1)] \right]
    \prod_{n_t=0}^{L_t-1} F_{n_t}[c^*_{i'}(\vn',n_t),c_i(\vn,n_t)].\no  
& & \mbox{ with } c_i(\vn,L_t)=-c_i(\vn,0)
\eea 
\end{framed}
There are several characteristic and implication of this equation.
\begin{itemize}
	\item This implies that one can replace the path integral over Grassman variable
	      into an operator relation. 
	\item Though it is possible to compute right-hand side  analytically in a perturbation theory,
	      the left-hand side can be done even in non-perturvative case
	      by explicitly computing action of creation and annihilation operators.
	\item Note that the Trace implies 
	the sum of results of actions of operators on all possible states.
	\item In right side, Grassman numbers have time dependence. In left side, operator themselves
	have no time dependence.	
\end{itemize}




If we define new Matrix ${\cal F}$, above product of normal ordered operators can be considered as
a standard form,
\bea 
\int [Dc Dc^*]\ e^{c^* {\cal F} c}=\det[{\cal F}],
\eea  
with ${\cal F}$ have matrix elements
\bea 
c^* {\cal F} c= \sum_{n_t=0}^{L_t-1} \sum_{\vn, i}
    c^*(\vn,n_t)[c_i(\vn,n_t)-c_i(\vn,n_t+1)]+\log\left(  
    \prod_{n_t=0}^{L_t-1} F_{n_t}[c^*_{i'}(\vn',n_t),c_i(\vn,n_t)]\right)
\eea 
In other words, the standard form of path integral can be recovered if we choose
\bea 
\log\left(  
\prod_{n_t=0}^{L_t-1} F_{n_t}[c^*_{i'}(\vn',n_t),c_i(\vn,n_t)]\right)
\to \sum_{n_t=0}^{L_t-1} \hat{H}[c^*(n,n_t),c(n,n_t)]
\eea 
Thus, if we choose transfer matrix, we can write partition function as
\bea 
Z&=&{\rm Tr}(M^{L_t}) ,\no 
M&=& :\exp\left[-\alpha_t \hat{H}(a^\dagger,a)\right]: 
\eea 
Note here the trace is for any multi particle states and M is an operator acting on states.
This is equivalent to the ordinary form
\bea 
Z=\int D c Dc^* \exp[-S(c,c^*)]
\eea 

Note that though the action $S(c,c^*)$ in path integral have time derivative 
but the action $M=e^{-H\alpha_t}$ in transfer matrix does not have time derivative.

In a more general case of auxiliary field or bosons, only fermion part of path integral is
converted into transfer matrix form and path integral over auxiliary particle or bosons
is done numerically. Roughly speaking, the actual numerical calculation is done with 
\bea 
Z=\int Ds e^{-S(s)}{\rm Tr}[M^{L_t}_A(s)]
\eea  


\subsection{Another derivation of Creutz(?) formula }
Consider one component fermion $|0\ra$ and $|1\ra=a^\dagger|0\ra$.
For $|i\ra=|0\ra$ or $|1\ra$,
we can prove following identity,
\bea 
\la i|:f(a,a^\dagger):|j\ra = f_{ij} =\overrightarrow{ \left(\frac{\del}{\del c^*}\right)^i}
                   \left[ e^{c^* c} f(c,c^*) \right] \overleftarrow{ \left(\frac{\del}{\del c}\right)^j}
                   \Big|_{c=0,c^*=0}.
\eea 
by considering four possible cases, $f(a,a^\dagger)=1$, $a$, $a^\dagger$, $a^\dagger a$. 
(For example, if $f=a$, $f_{ij}=\delta_{i0}\delta_{j1}$ in both expression.)

Then, we can express trace of operators as
\bea 
{\rm Tr}:f(a,a^\dagger): &=& \sum_{i=0,1} \la i|:f(a,a^\dagger):|i\ra 
=\sum_{i=0,1} \overrightarrow{ \left(\frac{\del}{\del c^*}\right)^i}
\left[ e^{c^* c} f(c,c^*) \right] \overleftarrow{ \left(\frac{\del}{\del c}\right)^i}  
\Big|_{c=0,c^*=0} \no 
 &=& \int d c d c^* e^{c^* c} f(c,c^*)
\eea 
Because of the equivalence between derivative and integration of Grassman number,
\bea 
\int d c \leftrightarrow \frac{\del}{\del c}, \quad 
\int d c^* \leftrightarrow \frac{\del}{\del c^*}.
\eea 
\bea 
\int d c \ 1 = 0 = \int d c^* \ 1
\eea 
\bea 
\int d c_a\ c_{b} =\delta_{ab}=\int d c^*_a\ c^*_{b} 
\eea 


On the other hand , matrix product is 
\bea 
\sum_{i=0,1} f(a)|i\ra \la i|g(a^\dagger) = f(c')\overleftarrow{ \left(\frac{\del}{\del c'}\right)^i}
                        \overrightarrow{ \left(\frac{\del}{\del c^*}\right)^i} g(c^*) \Big|_{c=0,c^*=0}
                      =\int (-1) d c' dc^* e^{-c^* c} f(c') g(c^*)
\eea 

Then,
\bea 
& &{\rm Tr} :f_{L_t-1}(a,a^\dagger)::f_{L_t-2}(a,a^\dagger):\dots :f_{0}(a,a^\dagger):\no 
&=&\sum_{i_{L_t-1},i_{L_t-2}\dots i_{0}}
   \la i_0|:f_{L_t-1}(a,a^\dagger):|i_{L_t-1}\ra \la i_{L_t-1}| :f_{L_t-2}(a,a^\dagger):|i_{L_t-2}\ra \dots 
   \la i_1 |:f_{0}(a,a^\dagger):|i_0\ra \no 
&=&\int d c(0) d c^*(L_t-1) e^{c^*(L_t-1) c(0)} e^{ c^*(L_t-1) c(L_t-1)} f_{L_t-1}(c,c^*) \no 
   & &  \times (-1) d c(L_t-1) d c^*(L_t-2) e^{-c^*(L_t-2)c(L_t-1)}\times \dots \no 
   & &  \times (-1) d c(1) d c^*(0) e^{-c^*(0) c(1)} e^{c^*(0)c(0)} f_0(c(0),c^*(0))  
\eea 

change order in measure,
\bea 
& &dc(0) dc^*(L_t-1)(-1) d c(L_t-1) dc^*(L_t-2)\dots (-1) dc(1) dc^*(0) \no 
&=& dc(L_t-1) dc^*(L_t-1)\dots d c(0) dc^*(0)
  = Dc Dc^*
\eea 
Thus,
\bea 
& &{\rm Tr}:f_{L_t-1}(a,a^\dagger)::f_{L_t-2}(a,a^\dagger):\dots :f_{0}(a,a^\dagger): \no 
&=&\int Dc Dc^* \exp \left( \sum_{n_t=0}^{L_t-1} c^*(n_t)[c(n_t)-c(n_{t}+1)] \right) 
           f_{L_t-1}(c(L_t-1),c^*(L_t-1))\dots f_0(c(0),c^*(0))\no 
& & \quad\mbox{with}\quad  c(L_t)=-c(0)            
\eea 

\section{Auxiliary Field}
Let us consider the simple action,
\bea 
Z=\int Dc Dc^* \exp[-S(c,c^*)],\quad 
S(c,c^*)=S_{free}(c,c^*)+C\alpha_t\sum_{\vn,n_t}
 \rho_\uparrow(\vn,n_t)\rho_\downarrow(\vn,n_t).
\eea 
with
\bea
S_{free}(c,c^*)&=&\sum_{\vn,n_t,i=\uparrow,\downarrow}
 [c_i^*(\vn,n_t)c_i(\vn,n_t+1)-c_i^*(\vn,n_t)c_i(\vn,n_t)]
 \no & &
 +h \sum_{\vn,n_t,i=\uparrow,\downarrow} 3\cdot 2 c_i^*(\vn,n_t)c_i(\vn,n_t)
 \no & &
 -h \sum_{\vn,n_t,i=\uparrow,\downarrow}\sum_{l=1,2,3}
 [c_i^*(\vn,n_t)c_i(\vn+\hat{l},n_t)+c_i^*(\vn,n_t)c_i(\vn-\hat{l},n_t)]
\eea
When the interaction involves more than two fermion field, 
we can not do integration manually because it is not in Gaussian integral form. 
In that case, we can use auxiliary field to make the interaction to be 
linear in two fermion field. 
\bea 
Z&=&\prod_{\vn,n_t}\left[\int d_A s(\vn,n_t) \right]
     \int Dc Dc^* \exp[-S_A(c,c^*,s)],\no 
S_A(c,c^*,s)&=& S_{free}(c,c^*)-\sum_{\vn,n_t} A[s(\vn,n_t)]\rho(\vn,n_t)
\eea 
such that\footnote{ Exact form of auxiliary field measure
varies as long as they satisfy conditions.  
Some example is
\bea 
\int d_A s(\vn,n_t)
&=&\frac{1}{\sqrt{2\pi}}\int_{-\infty}^{\infty} ds(\vn,n_t) e^{-\frac{1}{2}s^2(\vn,n_t)},\no 
A[s(\vn,n_t)]&=&\sqrt{-C\alpha_t} s(\vn,n_t)
\eea 

}
\bea 
\int d_A s(\vn,n_t)\ 1 &=& 1,\no 
\int d_A s(\vn,n_t) A[s(\vn,n_t)]&=&0, \no 
\int d_A s(\vn,n_t) A^2(s(\vn,n_t))&=& -C \alpha_t.
\eea 

This is equivalent to previous fermion partition function after integration over $s(\vn,n_t)$.

In a similar way, we can introduce auxiliary field in transfer matrix form,
\begin{framed}	
\bea 
Z=\prod_{\vn,n_t}[\int d_A s(\vn,n_t)]{\rm Tr}\{ M_A(s,L_t-1)\dots M_A(s,0)   \}, 
\eea 
\bea 
M_A(s,n_t)=:\exp\left[-\alpha_t \hat{H}_{free}+\sum_{\vn} A[s(\vn,n_t)] \rho(\vn) \right]:
\eea 
\end{framed}
where $\hat{H}_{free}$ have kinetic energy part of fermion and all interactions are
described by the coupling of auxiliary field and nucleon density. 
(Time derivatives are already included in the formalism 
and $\hat{H}_{free}$ only have spatial derivatives) 
And $d_A s(n,n_t)$ includes actions of auxiliary particles. 

As we can see now the transfer matrix is a function of auxiliary field, but
it is linear in fermion density. Thus, the transfer matrix acts each nucleon
separately linked by auxiliary field. 

Now the numerical calculation is done roughly
\begin{itemize}
	\item setup auxiliary filed configuration $s(n_t,{\bm n})$ for every time and lattice points.
	\item compute the fermion trace for transfer matrix {\bf (How to compute this is explained later)   }
	\item update auxiliary field configuration according to the 
	      probability distribution(which is calculated using previous 
	      transfer matrix elements) {\bf (How to update the configuration is explained in 
	      section about Metropolis or hybrid Monte carlo algorithm.)}
	\item repeat the above procedure with many configurations 
	\item take average over all configurations
	\item In many case, we have to test dependence on $L_t$. Thus, repeat the above steps 
	      to obtain ${\cal O}(L_t)$ functions and extrapolate to get physical values.       
\end{itemize}

{\color{red} {\bf This part requires further explanation. More detailed 
treatment of auxiliary field is already introduce in the first chapter.} }

\section{chemical potential}

For the case of chemical potential or any operators,
we can use following relation to make it normal ordered.
\bea 
e^{\alpha a^\dagger a}&= & 1+(e^{\alpha}-1)a^\dagger a=: e^{(e^\alpha-1)a^\dagger a}:,
\eea 
Because, for any parameter $\alpha$, 
\bea 
e^{\alpha a^\dagger a}|0\ra =1|0\ra,\quad e^{\alpha a^\dagger a}|1\ra =e^{\alpha}|1\ra.
\eea 
Thus, any operator should be written as normal ordered as
\bea 
\exp[a_i^\dagger X_{ij} a_j]&=& :\exp[a_i^\dagger(e^X -1)_{ij} a_j]: 
\eea 

\section{Long range interaction in auxiliary field formalism}
In previous subsection we rewrote the 4-fermion contact interaction 
in terms of auxiliary field. Then how can we represent the long range interaction
among fermions or smeared contact interaction in terms of auxiliary field?
\bea 
\Delta S=\sum_{\vn,\vn', n_t}\rho(\vn)V(\vn-\vn')\rho(\vn',n_t)
\eea 

{\color{red} {\bf One example can be found in the first chapter.} }


\section{Fermion transfer amplitude calculation}
Let us consider fermion transfer matrix with auxiliary field. 
By the introduction of auxiliary field, all operators in the action becomes
one-body operators.
\bea 
M_A(s,n_t)=:\exp\left[-\alpha_t \hat{H}_{free}+\sum_{\vn} A[s(\vn,n_t)] \rho(\vn) \right]:
\eea 
Then, physical spectrum of N-body fermion system can be obtained by computing,
\bea 
Z_{\Psi}=\prod_{\vn,n_t}[\int d_A s(\vn,n_t)]\la \Psi| \{ M_A(s,L_t-1)\dots M_A(s,0)   \} |\Psi\ra , 
\eea 
with proper N-body state $|\Psi\ra$ which have quantum number of system of interests. 
We will use following identity,

\begin{framed}
For N free fermion states with momentum $\vk$'s($\vk_{i=1,N}$), $|\Psi\ra $
is a slater determinant ground state wave function on lattice.

For $\Psi=|\phi_{\vk 1},\dots\phi_{\vk N}\ra$,
\bea 
\la \Psi| M(s,L_t-1)\dots M(s,0)|\Psi\ra =\det X(s)
\eea 
where
\bea 
X_{\vk,\vk'}(s)=\la \phi_\vk|M(s,L_t-1)\dots M(s,0)|\phi_\vk'\ra 
\eea 
with single particle state $\phi_\vk$ have momentum $\vk$. 
\end{framed}

This means that we can compute the amplitude as
\bea 
Z_\Psi(t)= \int D\pi D s\exp[-S_{\pi\pi}-S_{ss}]\det X(\pi,s,;t).
\eea 
and $X_{ij}$ matrix ,$i,j=1,\dots A$ can be
calculated by only computing single particle operators.

In other words, we don't even need to construct slater determinant wave function $|\Psi\ra$ itself.
Only prepare N different single particle wave functions and compute all
$X_{k,k'}(s)$ separately. Then, one can compute the action by $\det{X}$. 
Thus actual numerical calculation only involves computing $X_{kk'}(s)$ using
operators. 

\subsection{proof}

We can create a single-nucleon state using creation operators acting on the vacuum with
coefficient function $f({\bm n})$. $f({\bm n})$ is written as a column vector 
in the space of nucleon spin and isospin components, and the single-nucleon state can be
\bea 
|f\ra =\sum_{\bm n} a^\dagger({\bm n})f({\bm n})|0\ra.
\eea 
View identical nucleons as having a hidden index $j=1,\dots A$
that makes all of the nucleons distinguishable. 
If we anti-seymmetrize all physical states over this extra index then all physical
observables are exactly recovered. (Actually this is the slater determinant of single particle wave functions.)
So, A-body initial state to be a slater determinant of single nucleon states,(thus,
actually the index is for state index of single particle wave function,
not a particle index.)
\bea 
|f_1,\dots ,f_A\ra &=&\left[\sum_{\bm n} a^\dagger({\bm n})f_1({\bm n})\right] \cdots 
                      \left[\sum_{\bm n} a^\dagger({\bm n})f_A({\bm n})\right]|0\ra       \no 
                   &=&\frac{1}{\sqrt{A!}}\sum_{P}
                     \left[\sum_{\bm n} a^\dagger_{P(1)}({\bm n})f_1({\bm n})\right]
                     \cdots \left[\sum_{\bm n} a^\dagger_{P(A)}({\bm n})f_A({\bm n})\right]|0\ra \no 
                   &=& \frac{1}{\sqrt{A!}}\sum_{P'} \mbox{sgn}(P')
                     \left[\sum_{\bm n} a^\dagger_{[1]}({\bm n})f_{P'(1)}({\bm n})\right]
                     \cdots \left[\sum_{\bm n} a^\dagger_{A}({\bm n})f_{P'(A)}({\bm n})\right]|0\ra.        
\eea 
where the summations are over all permutations P. With these hidden indices,
(the Hamiltonian also have to be
changed with index $J$,
$a^\dagger_i,\ a_i\to a^\dagger_{i,J} ,\ a_{i,J}$)
normal-ordered auxiliary-field transfer matrix $M^{(n_t)}(s)$ becomes
\bea 
M_A^{(n_t)}(s)&=&:\exp\left[\sum_J \left(-\alpha_t \hat{H}_{free}^J
           +\sum_{\vn} A[s(\vn,n_t)] \rho^J(\vn)\right) \right]:
          \no  
          &=&\prod_{J} :\exp\left[-\alpha_t \hat{H}^J_{free}
                     +\sum_{\vn} A[s(\vn,n_t)] \rho^J(\vn)\right]: =\prod_{J} M_J(s)\no 
          &=&\left[ 1- H^{(n_t)}(a^\dagger_{[1]},a_{[1]},s)\alpha_t\right]
            \cdots \left[ 1- H^{(n_t)}(a^\dagger_{[A]},a_{[A]},s)\alpha_t\right]           
\eea 
where $M_J(s)$ only acts on $|\phi\ra_J$. Thus, 
transfer matrix is factorized into product of one-particle operators. 
Thus,
\bea 
Z(n_t)&=&\la f_1\dots f_A| M^{(n_t-1)}(s)\dots M^{(0)}(s)|f_1\dots f_A\ra
\eea 
can be calculated as 
\bea 
\frac{1}{\sqrt{N!}}\sum_{j'_n}\frac{1}{\sqrt{N!}}\sum_{j_n}
 \epsilon_{j'_1 \dots j'_n}\epsilon_{j_1 \dots j_n}
 \la \phi_{j'_1} |M_{1}(s,L_t-1)\dots M_{1}(s,0)|\psi_{j_1}\ra_{1} \cdots 
 \la \phi_{j'_N} |M_{N}(s,L_t-1)\dots M_{N}(s,0)|\psi_{j_N}\ra_{N} 
\eea 
Because the matrix elements are all equivalent for 'hidden' index,
we can define matrix $X$ such that
\bea 
X_{j' j}(s)=\la \phi_{j'}|M(s,L_t-1)\dots M(s,0)|\phi_{j}\ra ,
\eea 
and
\bea 
\la \Psi| M(s,L_t-1)\dots M(s,0)|\Psi\ra
=
\frac{1}{N!}\sum_{j'_n}\sum_{j_n}
 \epsilon_{j'_1 \dots j'_n}\epsilon_{j_1 \dots j_n} X_{j'_1 j_1}(s)\dots X_{j'_N j_N}(s)
=\det X(s)  
\eea 

\subsection{example in 2-body case.}
As an illustrative example, consider anti-symmetric two fermion states $|k_1 k_2\ra$
and only one transfer matrix. 
Let us consider general action as $\sum_{ij} a^\dagger_i a_j A_{ij}$,
\bea 
M=:\exp(\sum_{ij} a^\dagger_i a_j A_{ij}):
\eea 
Then, we have amplitude
\bea 
\la k_1 k_2|M|k_1 k_2\ra 
&= &\la k_1 k_2|:\left(
 1+(\sum a^\dagger a A)+\frac{1}{2}(\sum a^\dagger a A)^2 
 \right): |k_1 k_2\ra
 \no &=& 
1+A_{k_1 k_1}+A_{k_2 k_2}+A_{k_1k_1}A_{k_2k_2}-A_{k_1k_2}A_{k_2k_1} 
\eea 
On the other hand, 
\bea 
& &X_{k' k}= \la k'|M|k\ra =\delta_{k' k}+A_{k' k},\no 
& &\det X=1+A_{k_1 k_1}+A_{k_2 k_2}+A_{k_1k_1}A_{k_2k_2}-A_{k_1k_2}A_{k_2k_1} 
\eea 

\subsection{Application of the theorem for observables}

In fact, during the derivation of the theorem, the explicit form of transfer matrix $M$ was not 
used except that it can be expanded in powers of density operator $\rho(\vn)$ and
only the one power of $\rho(\vn)$ is need to be evaluated for a single particle state. 
Thus, if we replace the transfer matrix with other observables, we may evaluate 
matrix element of a operator as
\bea 
\la \Psi| M\dots \hat{\cal O}\dots M|\Psi\ra =\det X(s,\hat{\cal O})
\eea 
where 
\bea 
X_{ij}(s,\hat{\cal O})=\la \phi_i| M\dots \hat{\cal O}\dots M |\phi_j\ra 
\eea 
with single particle states. 

Note that even though the $X_{ij}(s,\hat{\cal O})$ looks linear to $\hat{\cal O}$,
the full matrix element $\det X(s,\hat{\cal O})$ is not linear to $\hat{\cal O}$. 

Also, note that because what we actually wants to measure is a expectation value of operator for 
ground state, we divides with $Z_\Psi(L_t)$ or $Z_\Psi(L_t-1)$
\bea 
\la 0|\hat{O}|0\ra =\lim_{L_t\to \infty} \frac{\la \Psi|M\dots \hat{O}\dots M|\Psi\ra}
                                            {\la \Psi|M^{L_t}|\Psi\ra} 
                   =\lim_{L_t\to \infty} \frac{\la \Psi|M\dots \hat{O}\dots M|\Psi\ra}
                   {\la \Psi|M^{L_t-1}|\Psi\ra}                                                           
\eea 
 




\subsection{In lattice code}
We can express the trasnfer matrix elements for any $n_t\leq L_t$ as   
\bea 
& &\la \Psi_0|M^{L_t}|\Psi_0\ra = \la \Phi(n_t)|\Phi(n_t)\ra
 =
 \la \Phi(n_t+1)| M^{(n_t)}  |\Phi(n_t)\ra,
\eea 
where vector and dual vector are defined as 
\bea  
& &|\Phi(n_t)\ra = M^{(n_t-1)} \dots M^{(0)}|\Psi_0\ra = M^{(n_t-1)}|\Phi(n_t-1)\ra ,\no 
& &\la \Phi(n_t+1)|=\la \Psi_0|M^{(L_t-1)}\dots M^{(n_t+1)} =\la \Phi(n_t)|M^{(n_t+1)}.
\eea 

Thus, any observable ${\cal O}(n_t)$, we get 
\bea 
\la \Psi_0| M^{(L_t-1)}\cdots M^{(n_t)} {\cal O}^{(n_t)} \cdots M^{(0)}|\Psi_0\ra 
 = \la \Phi(n_t)| {\cal O}^{(n_t)}|\Phi(n_t)\ra 
\eea 


In lattice code 'zvecs(nx,ny,nz,nt,ns,ni,np)' corresponds to $\Phi(n_t)\ra $,
\bea 
|zvecs^{(n_t)}_{i}\ra= M^{(n_t-1)}(s)|zvecs^{(n_t-1)}_{i}\ra
\eea , 
starting from $|zvecs^{(0)}_{i}\ra=|zvecs_{i}\ra$ 
where $i=(ns,nt,ni,np)$. 
In a similar way, 'zdualvecs(nt)' corresponds to $\la \Phi(n_t)$,
\bea 
\la zdualvecs^{(n_t-1)}_i|=\la zdualvecs^{(n_t)}_i| M^{(n_t-1)}
\eea 


\subsection{Coulomb correction}
We expect the Coulomb potential can be treated perturbatively. 
Thus, with perturbative approximation, At a certain time
\bea 
M^{(n_t)}_{full}(s,\pi)=\exp(-H_0\alpha_t-H_{int}\alpha_t-V_c\alpha_t)
\simeq 1-H_0\alpha_t -H_{int}\alpha_t -V_c\alpha_t = M^{(n_t)}-V_c\alpha_t.
\eea 
Then, we may think
\bea 
Z_{\Psi}(L_t) &=& \la M^{(n_t)}_{LO}\ra \sim  e^{-E_{LO} L_t \alpha_t},\no 
\la M^{(n_t)}_{full}\ra &\sim&  e^{-E_{LO} L_t \alpha_t}(1-\la V_c\ra \alpha_t),\no 
\la V_c\ra &\sim&  -\frac{1}{\alpha_t}\left(\frac{\la M^{(n_t)}_{full}\ra -\la M^{(n_t)}_{LO}\ra}{Z_\Psi(L_t)}\right)
\eea 

\section{Exact method to solve lattice Hamiltonian?} 
For a few body case(2 or 3 fermions), one can solve the problem exactly on the lattice. 
{\color{red} I am not sure whether this is the way used in the MATLAB code}

We would like to obtain the energy spectrum of the lattice Hamiltonian. 

One way to solve the problem is to diagonalize the Hamiltonian.

One-body case it is simple. Suppose we list states
$|{\bm n}\ra=a^\dagger(\vn)|0\ra$ and obtain the 
matrix elements $\la \vn'|\hat{H}|\vn\ra$. 
By computing this matrix elements with lattice Hamiltonian, 
one can diagonalize the matrix $\la \vn'|\hat{H}|\vn\ra$
to obtain s.p. eigen states $|f\ra=\sum_{n} c_n |{\vn}\ra$ and energy levels.

In a similar way, if one prepare a complete set of two nucleon states 
$|\vn_1 \vn_2\ra$ and diagonalize $\la \vn'_1 \vn'_2|\hat{H}|\vn_1 \vn_2\ra $ with lattice Hamiltonian
one can get the full solutions. 


Another way is to compute the Euclidiean time evolution without using 
MC samplings. This may be possible with transfer matrix formalism without auxiliary field. 
With two-body interactions, the transfer matrix can be constructed such as
\bea 
M=:\exp\{-H_{free}\alpha_t-\sum V(\vn_1,\vn_2)\rho(\vn_1)\rho(\vn_2) \} :
\eea 
Computing the action of transfer matrix operator on two- or three-body initial states
$|\Psi_2\ra$, $\Psi_3$ in center of mass frame, one can compute the discrete Euclidean time evolution
exactly without using MC samplings.  
 
 As for example, two nucleon $^1S_0$ or $^3S_1$ state in CM frame may be written as
 \bea 
 |f,^{1}S_0\ra &\propto & \sum_{\vn_1\vn_2}\sum_{\alpha\beta}
  f(\vn_1-\vn_2) P_{\alpha\beta} a^\dagger_\alpha(\vn_1)a^\dagger_\beta(\vn_2)|0\ra, \no 
  |f,^{3}S_1,s\ra &\propto& \sum_{\vn_1\vn_2}\sum_{\alpha\beta}
 f(\vn_1-\vn_2) P^s_{\alpha\beta} a^\dagger_\alpha(\vn_1)a^\dagger_\beta(\vn_2)|0\ra  
 \eea 
 where, $\alpha,\beta$ are spin and isospin index 
 and $P_{\alpha\beta},P^s_{\alpha\beta}$ are matrix elements of $^1S_0$, $^3S_1$ projection 
 operators. (For example, $P=\sigma_2 \tau_2\tau_3$ for $^1S_0$, $P^s=\sigma_2\sigma_s \tau_2$
 for $^3S_1$ up to complex constant.)
 (In CM frame, only relative coordinate is relevant.
 And complete sets of two nucleon states in $^1S_0$ can be thought as 
 $|\vn_1\vn_2,^1S_0\ra=a^\dagger_\alpha(\vn_1) a^\dagger_\beta(\vn_2) P_{\alpha\beta}|0\ra$.) 
 
More generally, initial nucleon state may be expressed with some wave function $F$ such that 
\bea 
|F\ra = \sum_{\{\vn_i\}} \sum_{\{\alpha_i\} } F_{\{\alpha_i\}}(\{\vn_i\}) 
        \prod_i a^\dagger_{\alpha_i}(\vn_i) |0\ra 
\eea 
Thus, by computing action of lattice Hamiltonian on this states, one can compute the
$Z(\tau)$ exactly without using MC sampling. Though, it will be impractical for more than 3-body. 


\section{Pinhole algorithm}
Let us review the Lattice calculation. 
From the path integral, we convert
\bea
Z_\Psi(L_t)&=&\la \Psi| e^{-\int d\tau H(\tau)} |\Psi\ra 
            =\int {\cal D}s{\cal D}\pi {\cal D}\psi^\dagger \psi e^{-S_{ss}-S_{\pi\pi}-S_{\psi}}   \no 
           &=&  \int {\cal D}s{\cal D}\pi e^{-S_{ss}-S_{\pi\pi} }\la\Psi | M^{(L_t)}(s,\pi)|\Psi\ra \no 
           &=&  \int {\cal D}s{\cal D}\pi e^{-S_{ss}-S_{\pi\pi}}\det X(s,\pi)  \no 
           &\simeq& \frac{1}{N_{cf}}\sum_{i=1}^{N_{cf}} P(s_i,\pi_i) e^{i\theta_i} ,\quad 
                   P(s_i,\pi_i)= e^{-S_{ss}-S_{\pi\pi}+\ln|\det X(s_i,\pi_i)|}
           \no 
           &\simeq&  \frac{1}{N_{cf}}\sum'_{i\in P_i} e^{i\theta_i} =\la e^{i\theta}\ra 
\eea 
where $\theta$ is a phase of $\det X(s,\pi)$.

In the first line, one express the correlation $Z_\Psi(L_t)$ into a path integral. 
However, this is not easy to solve numerically. Thus, one converts the fermion path integral into a
matrix element of a transfer operator. This transfer matrix elements can be expressed as a
determinant of a correlation matrix $X(s,\pi)$ which can be solved numerically from 
operator relations. Then the bosonic path integral can be done using MC samplings 
or metropolis MC samplings according to the probability $P(s_i,\pi_i)$. 

In a similar way, a expectation value of a operator ${\cal O}$ can be computed as
\bea 
\la \Psi|\hat{\cal O}|\Psi\ra(L_t) 
&=& \la \Psi_0| e^{-H\tau}  \hat{\cal O}(t) e^{-H\tau}   |\Psi_0\ra
=\int {\cal D}s{\cal D}\pi e^{-S_{ss}-S_{\pi\pi}} 
     \la \Psi_0| M^{(L_t/2)}(s,\pi) \hat{\cal O} M^{(L_t/2)}(s,\pi)|\Psi_0\ra  \no 
&=& \int {\cal D}s{\cal D}\pi e^{-S_{ss}-S_{\pi\pi}}  \det X(s,\pi,{\cal O}) \no  
&\simeq & \frac{1}{N_{cf}}\sum_i  P(s_i,\pi_i) e^{-\ln|\det X(s,\pi)|+\ln\det X(s,\pi,{\cal O})}  \no 
&=&  \frac{1}{N_{cf}}\sum_i  P(s_i,\pi_i) e^{-\ln|\det X(s,\pi)|+\ln|\det X(s,\pi,{\cal O})|+i\theta(s,\pi,{\cal O}) }  \no 
&=&   \la   e^{-\ln|\det X(s,\pi)|+\ln|\det X(s,\pi,{\cal O})|+i\theta(s,\pi,{\cal O}) } \ra 
\eea 
where $\theta(s,\pi,{\cal O})$ is a phase of $\det X(s,\pi,{\cal O})$. In other words,
$\la \Psi|\hat{\cal O}|\Psi\ra(L_t) $ is the same as a average 
of $e^{-\ln|\det X(s,\pi)|+\ln|\det X(s,\pi,{\cal O})|+i\theta(s,\pi,{\cal O}) }$
over configurations. For ${\cal O}=I$, $\la \Psi|\hat{\cal O}|\Psi\ra(L_t)=\la e^{i\theta(s,\pi)}\ra= Z_\Psi(L_t)$. 
Thus, the expectation value of an operator which depends on configurations can be obtained from
\bea 
\frac{\la \Psi|{\cal O}|\Psi\ra }{\la \Psi|\Psi\ra }
=\lim_{L_t\to \infty} \frac{\la {\cal O} e^{+i\theta(L_t) } \ra }{\la e^{+i\theta(L_t) } \ra   }. 
\eea 

 
Thus, if $P(s_i,\pi_i)$ is already available and once  $\det X(s,\pi,{\cal O})$ is 
calculated, one can compute the expectation value $\la \Psi|\hat{\cal O}|\Psi\ra(L_t)$. 
From the Euclidean time evolution, we can extract the contribution from ground state or excited states. 

Let us consider A-body density operator,
\bea 
\rho_{i_1j_1\cdots i_A j_A}(\vn_1,\cdots,\vn_A)=:\rho_{i_1 j_1}(\vn_1)\cdots \rho_{i_A j_A}(\vn_A): , 
\eea 
where $i$ is a spin index, $j$ is a isospin index. 
We have identity,
\bea 
\sum_{i_1 j_1\cdots i_A j_A}\sum_{\vn_1,\cdots,\vn_A} 
  \rho_{i_1j_1\cdots i_A j_A}(\vn_1,\cdots,\vn_A) =A !. 
\eea 
Then, for following amplitude 
\bea 
Z_{f,i}(i_1,j_1,\cdots,i_A,j_A;\vn_1,\cdots \vn_A;L_t)
=\la \Psi_f| M^{L'_t}_{*} M^{L_t/2} \rho_{i_1j_1\cdots i_A j_A}(\vn_1,\cdots,\vn_A)
             M^{L_t/2}M^{L'_t}_{*}|\Psi_i\ra ,
\eea 
In other words, roughly speaking 
\bea 
Z_{f,i}(i_1,j_1,\cdots,i_A,j_A;\vn_1,\cdots \vn_A;L_t)
\simeq |c_0|^2 e^{-E_0 L_t\alpha_t}\la \Psi_0|\rho_{i_1j_1\cdots i_A j_A}(\vn_1,\cdots,\vn_A)|\Psi_0\ra 
\eea 
We have identity,
\bea 
Z_{fi}(L_t)=\frac{1}{A!} \sum_{i_1 j_1\cdots i_A j_A}\sum_{\vn_1,\cdots,\vn_A} 
         Z_{f,i}(i_1,j_1,\cdots,i_A,j_A;\vn_1,\cdots \vn_A;L_t)
\eea 

In other words, with summation over pinholes in MC samplings, 
we can do the lattice calculation by using  amplitudes 
$Z_{f,i}(i_1,j_1,\cdots,i_A,j_A;\vn_1,\cdots \vn_A;L_t)$.
In other words, we may extend the meaning of a configuration 
to include both pinhole configuration and (auxiliary) boson field configurations. 
 
While doing MC samplings, if we count the number of configurations for 
a distribution of nucleons, we may get the density distribution of a ground state.   

To do this, for given configuration of pinholes, extract the center of mass position. 
Then, compute the number of nucleons in radial distance from the c.m. Then, taking 
average over pinhole configurations will give the average number of nucleons in radial distance.
In other words, density distribution. 

In summary,  denoting pinhole configurations as $[i_x,j_x,\vn_x]$, boson configurations $[s_i,\pi_i]$
and both configuration as $[cf_i]$,
\bea 
\la \Psi|{\cal O}|\Psi\ra(L_t) 
&=& \frac{1}{A!} \sum_{[i_x,j_x,\vn_x]} 
\int {\cal D}s {\cal D}\pi e^{-S_{ss}-S_{\pi\pi}} \det X(s,\pi,[i_x,j_x,\vn_x],{\cal O}, L_t) \no 
&=& \frac{1}{A!} \sum_{[i_x,j_x,\vn_x]}\sum_{[s_i,\pi_i]}  
     P(s_i,\pi_i,[i_x,j_x,\vn_x])  \no & &\times  
    e^{-\ln|\det X(s_i,\pi_i,[i_x,j_x,\vn_x],L_t)|+\ln|\det X(s_i,\pi_i,[i_x,j_x,\vn_x],{\cal O}, L_t)|+i\theta(s_i,\pi_i,[i_x,j_x,\vn_x],{\cal O}, L_t)}\no 
&\propto& \la e^{-\ln|\det X([cf_i],L_t)|+\ln|\det X([cf_i],{\cal O}, L_t)|+i\theta([cf_i],{\cal O}, L_t) }\ra      
\eea
where average is over the all configurations $[cf_i]$ of pinholes and bosons . 
If ${\cal O}$ is the same as density operator, we only have  $\la e^{+i\theta([cf_i], L_t) }\ra$. 

In other words, we obtains the expectation value of an operator which depends on configurations of pinholes and bosons as  
\bea 
\frac{\la \Psi|{\cal O}|\Psi\ra }{\la \Psi|\Psi\ra }
=\lim_{L_t\to \infty} \frac{\la {\cal O}[cf_i] e^{+i\theta([cf_i], L_t) } \ra }{\la e^{+i\theta([cf_i], L_t) } \ra   }. 
\eea 



\newpage
\chapter{Lattice actions and the form of Transfer matrix operators}

\section{Background material: Discrete Fourier Transformation}
(For a DFT of general lattice, FCC and BCC, look at 
the Zheng and Gu, J. Math. Imaging. Vis.(2014), 49:530-550.)

\subsection{Relation between discretization and continuum limit} 

First, let us make it clear for the terms. 
For harmonic system, Period $T$, frequency $\nu=\frac{1}{T}$, 
angular frequency $\omega=2\pi\nu=\frac{2\pi}{T}$, and Energy $E=\hbar\omega$.
Wave length $\lambda$, angular wave number $k=\frac{2\pi}{\lambda}$,
momentum $p=\hbar k$.  Then, we have relations,
\bea 
& &\exp(-i\frac{Et}{\hbar})\exp(i\frac{p x}{\hbar})
  =\exp(-i\omega t)\exp(ik x) 
\eea 
\begin{tabular}{|c|c|c|c|c|c|c|c|}
	\hline 
	continuum &  $x$ & $p=\hbar k$ & $\int dx$ & $\int \frac{dp}{2\pi \hbar}$ 
	& $\delta(x)$ & $\delta(k)$ & $e^{ikx}$
	\\ 
	\hline 
	discrete  & $x_n=a n$ & $k_m=\frac{2\pi}{L}m$ & $a\sum_{n}$ & $\frac{1}{L}\sum_{m}$ 
	& $\frac{1}{a}\delta_{n0}$ & $\frac{L}{2\pi}\delta_{m0}$  & $ e^{i 2\pi \frac{m n}{N}}$
	\\ 
	\hline 
\end{tabular}
 
And delta function (orthogonality) relation 
\bea 
& &\int dx e^{i k x} =2\pi \delta(k) \no 
&\rightarrow&  a\sum_{n} e^{i (\frac{2\pi}{L} m) \cdot (a n)}
             =a \sum_{n} e^{i 2\pi \frac{mn}{N}} = (2\pi)\frac{L}{2\pi}\delta_{m0}
             = N a \delta_{m0}  ,\no 
&\rightarrow& \sum_{n} e^{i 2\pi \frac{mn}{N}} = N \delta_{m 0}               
\eea 

Discretization of Fourier transformation,
\bea 
& &f(x)=\int \frac{dp}{2\pi \hbar} e^{i k x} \hat{f}(k) \no 
&\to& f(x_n)=\frac{1}{L} \sum_{m} e^{i k_m x_n} \hat{f}(k_m) 
      =\frac{1}{N} \sum_{m} e^{i 2\pi \frac{mn}{N}} \frac{\hat{f}(k_m)}{a}
      =\frac{1}{N}\sum_{m} e^{i 2\pi \frac{mn}{N}} \hat{f}_m
\eea 
implies the momentum component $\hat{f}_m=\frac{1}{a}\hat{f}(k_m)$ 
have different dimension with $\hat{f}(k_m)$ though 
$f(x_n)$ and $f_n$ have the same dimension.    

And its reverse is 
\bea 
& &\hat{f}(k)=\int dx e^{-ikx} f(x) \no 
&\to& \hat{f}(k_m)= a \hat{f}_m= a\sum_{n} e^{-i 2\pi \frac{mn}{N}} f_n 
\eea 

\subsection{DFT}
Let us consider equally spaced N points. $x_j= j a$, $j=0,\dots N-1$.
Thus, it covers $x\in [0, L=N a]$ if it is considered to be periodic. 
Then, a (periodic) function can be expressed in DFT as
\bea 
\hat{f}_m&=& \hat{f}(k_m)=\sum_{n=0}^{N-1} e^{- i{x_n\cdot k_m}} {f}_n,\no 
f_n&=&f(x_n)=\frac{1}{N} \sum_{m=0}^{N-1} e^{i{x_n\cdot k_m}} \hat{f}_m,
\eea 
with angular wave number $k_m=\frac{2\pi m}{L}=\frac{2\pi m}{N a}$, 
\bea 
\exp(i\vx_n\cdot\vp_m)\to \exp\left(i n a\times \frac{2\pi m}{L}\right)
                     =\exp\left(i 2\pi \frac{n m}{N}\right) 
\eea 
(In fact, in analogy of $i\omega t=i \frac{2\pi}{T} t=i 2\pi \nu t$,
 we may call $\nu\to \frac{m}{N a}$ frequency.)  
In fact the definition or dimension of $\hat{f}$ depends on the convention. 




For given equal spaced points, $t_k=k\Delta$, $k=0,\dots,N-1$ and function values $f(t_k)$
\bea 
F_n=\sum_{k=0}^{N-1} f_k e^{-\frac{2\pi i}{N}n k}\quad \leftrightarrow \quad 
f_k=\frac{1}{N}\sum_{n=0}^{N-1} F_n e^{\frac{2\pi i}{N} n k}.
\eea 
In other words, basic equation is 
\bea 
\frac{1}{N}\sum_{k=0}^{N-1}e^{-i\frac{2\pi}{N}k(n-n')}=\delta_{nn'}
\eea 

In symmetric convention, complex valued function $a$ defined on a subset $[N]:=\{0,1,2,\dots,N-1\}$ 
can be DFT to function $\hat{a}$ on $k \in [N]$,
\bea 
\hat{a}(k)=\frac{1}{\sqrt{N}}\sum_{j\in[N]} a(j) e^{-2\pi i \frac{jk}{N}}.
\eea 

For general dimension, by using generators of a basis of Lattice L, one can construct 
\bea 
L:=\Big\{  \sum_{i=1}^d k_i {\bm u}_i :k_i \in Z\ \mbox{for}\ i=1,2,\dots d \Big\}.
\eea 
If there is a sub-lattice $L_0$, (in other words, if there is a periodicity 

we have
\bea 
\hat{a}(\bar{s})&=&
 \frac{1}{\sqrt{|G|}}\sum_{\bar{r}\in G} a(\bar{r})\cdot e^{-2\pi i \la {\bm r}\cdot{\bm s}\ra},\no 
{a}(\bar{r})&=&
\frac{1}{\sqrt{|G|}}\sum_{\bar{s}\in \hat{G}} \hat{a}(\bar{s})\cdot e^{2\pi i \la {\bm r}\cdot{\bm s}\ra} 	
\eea 
Or  
\bea 
\hat{f}(n)=\sum_{m=1}^{L} f_{in}(m) e^{-2\pi i {\bm r}^T_{m}\cdot {\bm s}_{n}}
\eea   
where     ${\bm r}_{m}$ is a point in $S_{\Omega}$ and ${\bm s}_{n}$
is a point in $S^A_{\Omega}=(V_{bs}^{-1})^T S^A_{\Omega,coef}$.
It is equivalent to
\bea 
\hat{f}(n)=\sum_{m=1}^{L} f_{in}(m) e^{-2\pi i {\bm [c]}^T_{m}\cdot(M^{-1})^T\cdot {\bm [s]}_{n}}
\eea   
where ${\bm [c]}_m$ is a point in $S_{\Omega,coef}$
and ${\bm [s]}_n$ is a point in $S^A_{\Omega,coef}$
	\bea 
\la {\bm r}_m,{\bm s}_n\ra ={\bm r}^T\cdot{\bm s}={[\bm c]}_m^T\cdot (M^{-1})^T\cdot {\bm [s]}_{n}
\eea 

When using numerical FFT, one have to keep in mind that 
\begin{itemize}
	\item In MATLAB and FORTRAN, the index of an array usually starts from 1. 
	      However, the FFT routine treats first element of $f$, $f(1)$ is at $x=0$.
	      Thus, in explicit indexing $f(x_i)$, $x_i=(i-1)$, 
	      \bea 
	       \hat{f}(m)=\sum_{l=1}^L f_{in}(l)\exp\left(-i \frac{2\pi}{L}\cdot(m-1)(l-1)\right)
	      \eea 
	      In a similar way, 2-dimensional FFT is
	      \bea 
	       \hat{f}(m,n)=\sum_{i=1}^{L}\sum_{j=1}^L f_{in}(i,j) \exp\left[-2\pi i\left(\frac{(i-1)(m-1)}{L}+\frac{(j-1)(n-1)}{L}\right)  \right] 
	      \eea  
	 \item Depending on the code, the normalization of $\hat{f}$ can be different
	      and also inverse transformation. 
	      For example, MATLAB fftn have $1/N$ factor for inverse transformation,
	      on the other hand, FFTW routine does not have $1/N$ factor.
	      Thus, successive FFT and iFFT does not give original $f_{in}$ but $L*f_{in}$. 
    \item FFTW use standard ordering of frequency $k/n$. It can be interpreted as
          the positive frequency in the first half of the output and the negative 
          frequencies are stored in backward order in the second half of the output.
          ( $-k/n$ is equivalent to $(n-k)/n$).
          \bea
          Y_{k}&=&\sum_{j=0}^{n-1} X_{j} e^{-2\pi i \frac{jk}{n}} \quad \mbox{forward FFT},\no 
          Y_{k}&=&\sum_{j=0}^{n-1} X_{j} e^{2\pi i \frac{jk}{n}} \quad \mbox{backward FFT}. 
          \eea  	
    \item in other words, the 1-dimensional lattice and its momentum(frequency) index are 
          \bea 
             x&=&a \left[0,1,2,\dots \frac{N}{2},\frac{N}{2}+1,\dots N-1\right],\no 
             p&=&\frac{2\pi}{L}\left[0,1,2,\dots, -\frac{N}{2}, -\frac{N}{2}+1,\dots, -1\right]
          \eea 
          Or, $p_k=\frac{2\pi}{L} k$ for $k<\frac{N}{2}$,
             $p_k=\frac{2\pi}{L}(k-N)$ for $k\geq \frac{N}{2}$.
             ( this may be written in Fortran as
             $\frac{2\pi}{L}(k-N\times \mbox{int}(\frac{k}{N/2}) )$ )
             
    \item In case DFT of real function, $X_j^*=X_j$ one can use the symmetry,
         \bea 
         Y_{k}=Y_{-k}^*=Y_{n-k}^*.
         \eea  
         FFTW of r2c and c2r use this symmetry.(the real array has length n, but complex array has 
         length $n/2+1$. )                    
\end{itemize}

\subsection{Derivative calculation using DFT} 
One may obtain the numerical derivative of function by approximating finite difference,
\bea 
f'(x_i)\simeq \frac{f(x_{i+1})-f(x_{i-1})}{2 a},\quad 
f''(x_i)\simeq \frac{f(x_{i+1})+f(x_{i-1})-2f(x_i)}{a^2}.
\eea  
But, for periodic function in range, we can also use DFT 
and inverse DFT to compute derivative.
\bea 
f'(x_n)=\frac{1}{N}\sum_{m} e^{i2\pi\frac{mn}{N}} (i \frac{2\pi m}{N a})  \hat{f}_m,\quad 
f''(x_n)=\frac{1}{N}\sum_{m} e^{i2\pi\frac{mn}{N}}(i \frac{2\pi m}{N a})^2\hat{f}_m.
\eea 
This will give exact results if the function is periodic $f(x+L)=f(x)$.

On the other hand, previous numerical derivative expression corresponds to 
\bea 
\frac{f(x_{n+1})+f(x_{n-1})-2f(x_n)}{a^2}
= \frac{1}{N}\sum_{m} e^{i2\pi\frac{mn}{N}} 
  \frac{1}{a^2}\left(e^{i\frac{2\pi m}{N} }+e^{-i\frac{2\pi m}{N} }-2\right) 
  \hat{f}_m.
\eea 


In fact, there is an ambiguity using the DFT for differentiation.
Any periodic shift in momentum (aliasing) will also give the same spatial
function values but will have significant difference in derivative. 
One convention is to use "minimal-oscillation" trigonometric interpolation
for iFFT :
\bea 
f(x)=\hat{f}_0+\sum_{0<k<N/2}\left( \hat{f}_{k}e^{i\frac{2\pi}{L} k x}
   +\hat{f}_{N-k}e^{-i\frac{2\pi}{L} k x} \right) 
   +\hat{f}_{N/2}\cos(\frac{\pi}{L} N x),
\eea 
where the Nyquist term ($k=\frac{N}{2}$) is absent for odd N. 

However, if the function is not periodic, the derivative calculation 
by using DFT will gives errors. 
One way to improve the problem is to sample data in $x\in [-1,1]$
at the points $x_n=\cos(n\pi/N)$ for n=0,\dots N and then to use 
Chevyshev interpolation. $f(x)=\sum_{n=0}^{N} a_n T_{n}(x)$ 
with Chevyshev polynomial $T_{n}(x)$. This is essentially the Fourier series
with change of variable $x=\cos\theta$.
\bea 
T_n(\cos\theta)=\cos n\theta,\quad T_{2n+1}(\sin\theta)=(-1)^n\sin((2n+1)\theta).
\eea 

\section{Lattice notation}
Let us denote $a_t$ and $a$ as a lattice interval in time and space respectively and
$L$ is the length of the cubic spatial lattice in each direction
such as
\bea 
L_t=N_t a_t,\quad L=N a,
\eea 
And $\alpha_t\equiv \frac{a_t}{a}$ is a ratio of lattice intervals. 
To discretize coordinate and momentum, we replace
\bea 
\vx \to a \vn ,\quad  \vp \to \hbar \frac{2\pi}{L}\vm ,\quad (\vn,\vm \mbox{ are sets of 3 integers })
\eea 
such that
\bea 
\int dt d\vx \to a_t a^3 \sum_{\vn,n_t}, \quad 
i\frac{\vx\cdot\vp}{\hbar} \to i \frac{2\pi}{N}\times\vn\cdot \vm .
\eea 
with $n_i,m_i=0,1,\dots L-1$. We can consider $\frac{{\bm m}}{N}$ as a frequency.
Because of periodicity, all integers are up to modular $N$. 
(For frequency, one can use $m_i=-N/2,\dots N/2-1$).

\begin{itemize}
\item ${\vec n}$ represents integer-valued lattice vectors on a three-dimensional spatial lattice
\item ${\vec p},{\vec q},{\vec k}$ represent integer-valued momentum lattice vectors.
    
\item $\hat{l}=\hat{1},\hat{2},\hat{3}$ are unit lattice vectors in the spatial directions
\item $n_t$ labels the number of time steps.
\item Everything will be written in a {\bf dimensionless parameters} 
      and operators which can be converted to physical dimension by appropriate power of $a$. 
\item $\hat{a}$ and $\hat{a}^\dagger$ are annihilation and creation operators.
\item spin and isospin indices are used as
      \bea 
      & a_{0,0}=a_{\uparrow,p}, & a_{0,1}=a_{\uparrow,n},\no 
      & a_{1,0}=a_{\downarrow,p}, & a_{1,1}=a_{\downarrow,n}
      \eea     
\item $\tau_{I=1,2,3}$ and $\sigma_{S=1,2,3}$ are Pauli matrices in isospin and spin.        
\end{itemize}

For a creation and annihilation operator for states $p$, $a^\dagger_p$ and $a_p$,
one can define position space operator 

\bea 
a(x)=\frac{1}{L}\sum_{p} e^{ip\cdot x} a_p, \quad
a^\dagger(x)=\frac{1}{L}\sum_{p} e^{-ip\cdot x} a^\dagger_p.
\eea 
\bea 
\int_0^L dx a^\dagger(x) a(x)&=&\sum_p a^\dagger_p a_p  
\eea 

Note that these operator does not have (energy) state index. 
\subsection{Discrete Fourier transformation}
Because lattice size acts as a momentum cutoff, in actual calculation we choose
size of $a$ such that 
\bea 
m_\pi< \frac{\pi}{a}< \Lambda_\chi.
\eea 
And because we will use periodic boundary condition in space,
range of $k$ values are
\bea 
-\Lambda < \frac{2\pi}{L}k_{1,2,3}\leq (\Lambda=\frac{\pi}{a})\ 
\to\ -\frac{N}{2}< k_{1,2,3}\leq \frac{N}{2}
\eea 
Let us consider {\bf 4-dimensional} F.T.,
\footnote{ Normalization factors $L_t L^3$ depends on the convention.}
\footnote{
	Though I use $k_*$, $n_*$ to distinguish dimensional quantity and dimension less quantity,
	we can always replace them as dimensionless qunatities keeping in mind above conevntion.
	So, $\vn_*=a\vn$,
	\bea 
	\sum_{k} e^{i\vk_*\cdot (\vn_*-\vn'_*)}=N^3 \delta_{\vn \vn'}
	\eea 
}
\bea 
\tilde{f}(\vk)&=&\frac{1}{\sqrt{L_t L^3}}\sum_{\vn_*} e^{i\vk_*\cdot\vn_*} f(\vn),\no 
f(\vn)&=&\frac{1}{\sqrt{L_t L^3}}\sum_{\vk} e^{-i\vk_*\cdot\vn_*} \tilde{f}(\vk),
\eea 
with
\bea 
\vk_*&=&\left(\frac{2\pi}{N_t}k_0,\frac{2\pi}{N}k_1,\frac{2\pi}{N}k_2,\frac{2\pi}{N}k_3\right),
\quad \mbox{$k_i$ are integers},\no   
\sum_{\vn_*}e^{i\vk_*\cdot\vn_*}&=& N_t N^3 \delta_{\vk 0}
\eea 

Then F.T. of some typical operators are\footnote{
	If we use other convention, there will be $\frac{1}{L_t L^3}$ factors to
	operators. 
} 
\bea 
& &\sum_{\vn} c^*_i(\vn)c'_i(\vn+\hat{0})
=\sum_{\vk}\sum_{\vk'} \tilde{c}^*_i(\vk') \tilde{c}'_i(\vk) 
\frac{1}{L_t L^3}
\sum_{\vn} e^{-i\vk'\cdot\vn} e^{-i\vk\cdot\vn}e^{-i\vk\cdot\hat{0}}
=\sum_{\vk} \tilde{c}^*_i(-\vk) \tilde{c}'_i(\vk)e^{-i\vk\cdot\hat{0}},\no 
& &\sum_{\vn} c^*_i(\vn)c'_i(\vn)=\sum_{\vk} \tilde{c}^*_i(-\vk) \tilde{c}'_i(\vk),
\quad 
\sum_{\vn} c^*_i(\vn)c'_i(\vn+\hat{l}_s)
=\sum_{\vk} \tilde{c}^*_i(-\vk) \tilde{c}'_i(\vk)e^{-i\vk\cdot\hat{l}_s},\no 
& &\sum_{\vn} [c^*_i(\vn)c'_i(\vn+\hat{l}_s)+c^*_i(\vn)c'_i(\vn-\hat{l}_s)]
=\sum_{\vk} \tilde{c}^*_i(-\vk) \tilde{c}'_i(\vk)
[e^{-i\vk\cdot\hat{l}_s}+e^{+i\vk\cdot\hat{l}_s}] 
=\sum_{\vk} \tilde{c}^*_i(-\vk) \tilde{c}'_i(\vk)
[2\cos(\vk\cdot\hat{l}_s)]      \no 
\eea 


\subsection{Spatial derivatives}
We approximate the derivative of a function in terms of hopping operators in lattice.
 
For each spatial direction $l=1,2,3$ and any lattice function $f(\vn)$, let
\bea 
\Delta_l f(\vn)\equiv \frac{1}{4}\sum_{\nu_{1,2,3}=0,1}
               (-1)^{\nu_l+1}f(\vn+{\vec \nu}),\quad 
               {\vec \nu}=\nu_1\hat{1}+\nu_2\hat{2}+\nu_3\hat{3}.
\eea 

Explicit expression for $l=1$ case is
\bea 
\Delta_{1} f(\vn)&=&\frac{1}{4}\Big[f(\vn+\hat{1})-f(\vn)
                             +f(\vn+\hat{1}+\hat{2})-f(\vn+\hat{2})
                             \no & &
                             +f(\vn+\hat{1}+\hat{3})-f(\vn+\hat{3})
                             +f(\vn+\hat{1}+\hat{2}+\hat{3})-f(\vn+\hat{2}+\hat{3})\Big]
\eea 
This can be thought as average of 1-direction derivative of values 
at $\vn$, $\vn+\hat{2}$,$\vn+\hat{3}$,$\vn+\hat{2}+\hat{3}$.

Double spatial derivative along direction $l$,
\bea 
\nabla_{l}^2 f(\vn)=f(\vn+\hat{l})+f(\vn-\hat{l})-2f(\vn).
\eea 


\section{local densities and currents} 
We define 
\bea 
\rho^{a^\dagger,a}(\vr)&=&\sum_{i,j=0,1}a_{i,j}^\dagger(\vr) a_{ij}(\vr),\no 
\rho^{a^\dagger,a}_S(\vr)&=&\sum_{i,j,i'=0,1} a^\dagger_{ij}(\vr)[\vs_S]_{ii'} a_{i'j}(\vr),
           \quad S=1,2,3\no 
\rho^{a^\dagger,a}_I(\vr)&=&\sum_{i,j,j'=0,1} a^\dagger_{ij}(\vr)[\tau_I]_{jj'} a_{ij'}(\vr),
           \quad I=1,2,3\no 
\rho^{a^\dagger,a}_{SI}(\vr)&=&\sum_{iji'j'=0,1} a^\dagger_{ij}(\vr)[\vs_S]_{ii'}[\tau_I]_{jj'}
                                  a_{i'j'}(\vr).
\eea 
where, each are local density, local spin density, local isospin densitym
local spin-isospin density.

For each static density, we also have an associated current density. 
Using definition ${\vec\nu}$ and ${\vec{\nu}}(-l)$, which is 
a reflecting l-th component of ${\vec\nu}$ about the center of the cube,
\bea 
{\vec \nu}&=&\nu_1\hat{1}+\nu_2\hat{2}+\nu_3\hat{3},\quad \nu_{1,2,3}=0,1,\no 
{\vec{\nu}}(-l) &=& {\vec{\nu}}+(1-2\nu_l)\hat{l},\quad l=1,2,3
\eea 

Omitting factors of $i$ and $m$\footnote{Exactly? $\frac{1}{2im}$? },
SU(4) invariant current density,( with a discretization of derivative)
\bea 
\Pi_l^{a^\dagger,a}(\vn)
&=&\frac{1}{4}\sum_{\nu_{1,2,3}=0,1}\sum_{i,j=0,1}
   (-1)^{\nu_l+1} a^\dagger_{i,j}(\vn+{\vec \nu}(-l)) a_{i,j}(\vn+{\vec{\nu}}).
\eea 

Explicit form for $l=1$ case,\footnote{ $a^\dagger(\vx)\nabla a(\vx)-\nabla a^\dagger(\vx) a(\vx)
\to a^\dagger(\vn)[a(\vn+1)-a(\vn)]-[a^\dagger(\vn+1)-a(\vn)]a(\vn)
= a^\dagger(\vn)a(\vn+1)-a^\dagger(\vn+1)a(\vn)
$}
\bea 
\Pi_1^{a^\dagger,a}(\vn)
&=&\frac{1}{4}\sum_{i,j=0,1}\Big[
             a_{ij}^\dagger(\vn)a_{ij}(\vn+\hat{1})-a_{ij}^\dagger(\vn+\hat{1})a_{ij}(\vn)
             \no & &
             +a_{ij}^\dagger(\vn+\hat{2})a_{ij}(\vn+\hat{1}+\hat{2})
             -a_{ij}^\dagger(\vn+\hat{1}+\hat{2})a_{ij}(\vn+\hat{2})
             \no & &
             +a_{ij}^\dagger(\vn+\hat{3})a_{ij}(\vn+\hat{1}+\hat{3})
             -a_{ij}^\dagger(\vn+\hat{1}+\hat{3})a_{ij}(\vn+\hat{3})
             \no & &  
             +a_{ij}^\dagger(\vn+\hat{2}+\hat{3})a_{ij}(\vn+\hat{2}+\hat{3})
             -a_{ij}^\dagger(\vn+\hat{1}+\hat{2}+\hat{3})a_{ij}(\vn+\hat{2}+\hat{3})                    
\Big]
\eea 

In a similar way, we can define spin current density, 
isospin current density and spin-isospin current density as
\bea 
\Pi_{l,S}^{a^\dagger,a}(\vn)
&=&\frac{1}{4}\sum_{\nu_{1,2,3}=0,1}\sum_{i,j,i'=0,1}
   (-1)^{\nu_l+1} a^\dagger_{i,j}(\vn+{\vec \nu}(-l))[\sigma_S]_{i i'} a_{i',j}(\vn+{\vec{\nu}}),\no 
\Pi_{l,I}^{a^\dagger,a}(\vn)
&=&\frac{1}{4}\sum_{\nu_{1,2,3}=0,1}\sum_{i,j,j'=0,1}
   (-1)^{\nu_l+1} a^\dagger_{i,j}(\vn+{\vec \nu}(-l))[\tau_I]_{j j'} a_{i,j'}(\vn+{\vec{\nu}}),\no 
\Pi_{l,SI}^{a^\dagger,a}(\vn)
&=&\frac{1}{4}\sum_{\nu_{1,2,3}=0,1}\sum_{i,j,i',j'=0,1}
   (-1)^{\nu_l+1} a^\dagger_{i,j}(\vn+{\vec \nu}(-l))[\sigma_S]_{i i'}[\tau_I]_{jj'}
    a_{i',j'}(\vn+{\vec{\nu}}),
\eea 

\section{Explicit form of lattice action} 
Until now, we have not used any specific expression for lattice action.
Let us write the lattice action explicitly in case of chiral EFT. 

\subsection{Example: complex scalar field on a circle of length L}
Momentum space
\bea 
|p\ra = a_p^\dagger|0\ra 
\eea 
Non-interacting Hamiltonian,
\bea 
\sum_p \frac{p^2}{2m}a_p^\dagger a_p
\eea 
Total number of particle,
\bea 
\sum_p a_p^\dagger a_p
\eea 
Position operator
\bea 
a(x)=\frac{1}{\sqrt{L}}\sum_p e^{ipx}a_p,\quad 
a^\dagger(x)=\frac{1}{\sqrt{L}}\sum_p e^{-ipx}a_p^\dagger 
\eea 
Local density operator $a^\dagger(x) a(x)$
\bea 
\int_0^L dx a^\dagger(x) a(x)=\sum_p a^\dagger_p a_p
\eea 
Space discretization
\bea 
& &\int_0^L dx\to a \sum_{n=0}^{N-1},\quad a=\frac{L}{N},\no 
& &a(x)\to a(n),\quad a(N)=a(0),\quad [a(n),a^\dagger(n')]=\delta_{n,n'},\no 
& &\psi(x)\to \psi(n).
\eea 
The momentum eigen-state may be written as
\bea 
|p\ra = a^\dagger_p|0\ra=\frac{1}{\sqrt{N}}\sum_{n=0}^{N-1} e^{ipn}a^\dagger(n)|0\ra,\quad p=\frac{2\pi}{N}k 
\eea 
We may approximate free Hamiltonian as
\bea 
H^{free}=-\frac{1}{2m}\frac{a}{a^2}\sum_{n=0}^{N-1}[ a^\dagger(n+1) a(n)+a^\dagger(n) a(n+1) -2a^\dagger(n) a(n) ] 
\eea 
This gives dispersion relation, 
\bea 
E(p)=\frac{1}{m}(1-\cos p)=\frac{p^2}{2m}+{\cal O}(p^4)
\eea 

The action of free transfer matrix can be calculated exactly for one-body system,
\bea 
M_{free}=:\exp[-H_{free}\alpha_t]: =:1 -H_{free}\alpha_t\cdots :
\eea 
And one nucleon state $|p\ra$, we have
\bea 
M_{free}|p=\frac{2\pi}{L}k \ra =\left(1+\frac{\alpha_t}{m}(\cos \frac{2\pi}{L}k-1)\right) |p\ra 
          =e^{-E(k)\alpha_t}|p\ra 
\eea 
with 
\bea 
E(k)=-\frac{1}{\alpha_t}\ln\left(1+\frac{\alpha_t}{m}(\cos \frac{2\pi}{L}k-1)\right) 
\eea 
For two nucleon state $|k,\uparrow;k',\downarrow\ra$, exponential of transfer matrix have to be expanded up to 
2nd order, 
\bea 
M_{free}|k,\uparrow;k',\downarrow\ra
&=& \left[1+\frac{\alpha_t}{2m}\sum_{n}(a^\dagger(n+1) a(n)+a^\dagger(n)a(n+1)-2 a^\dagger(n) a(n))\right]_{\uparrow}
    \no & &\times  
\left[1+\frac{\alpha_t}{2m}\sum_{n}(a^\dagger(n+1) a(n)+a^\dagger(n)a(n+1)-2 a^\dagger(n) a(n))\right]_{\downarrow}
 |k,\uparrow;k',\downarrow\ra \no 
& =& e^{-E_\uparrow(k)\alpha_t}e^{-E_\downarrow(k')\alpha_t}|k,\uparrow;k',\downarrow\ra
\eea 
\newpage 
\section{3-d free nucleon Hamiltonian}
\subsubsection{Approximate action} 
Consider fermion action
\bea 
-S_E[\psi,\psi^\dagger]&=& -\int dt_E \int d^3 x\left[ \psi^\dagger \del_t\psi 
         +{\cal H}_{free}+{\cal H}_{int}  \right]    ,\no
{\cal H}_{free} &=& -\frac{1}{2m}\psi^\dagger \nabla^2 \psi        
\eea 

\footnote{ Later it will be useful to define $h$
\bea 
h=\frac{\alpha_t}{2m}.
\eea 
}
For one small time slap, a simple discretization\footnote{
Use $f''=\frac{f_{+h}-2f_0+f_{-h}}{h^2}$. But, more improved action can be used.
} gives
\bea 
& &\int_{t}^{t+a_t} dt_E \int d^3 x\left[ \psi^\dagger \del_t\psi +{\cal H}_{free} \right] \no 
&\to& a_t a^3\sum_{\bm n} \frac{1}{a_t}(a^\dagger({\bm n},n_t)a({\bm n},n_t+1)-a^\dagger({\bm n},n_t)a({\bm n},n_t))
      \no 
& & +(-\frac{\alpha_t}{2m}\frac{a^3}{a}) \sum_{\bm n}\sum_{\bm l}
    (a^\dagger({\bm n},n_t)a^\dagger({\bm n}+{\bm l},n_t)
    +a^\dagger({\bm n},n_t)a^\dagger({\bm n}-{\bm l},n_t)
    -2a^\dagger({\bm n},n_t)a^\dagger({\bm n},n_t)) \nonumber 
\eea 
where ${\bm l}$ if for 3-direction and gives $\hat{H}_{free}$ for transfer matrix operator 
$M^{(n_t)}=:\exp(-\hat{H}_{free}\alpha_t):$.

If we use 'improved action', we get (set $a=1$)
\bea 
\hat{H}_{free}&=&\frac{1}{2m} 3\cdot 2 \omega_0 \sum_{\vn}\sum_{i,j} a^\dagger_{ij}(\vn) a_{ij}(\vn)\no 
        & &-\frac{1}{2m} \omega_1 \sum_{\hat{l}=1,2,3}\sum_{\vn}\sum_{i,j}
                  [a^\dagger_{ij}(\vn) a_{ij}(\vn+\hat{l})
                  +a^\dagger_{ij}(\vn) a_{ij}(\vn-\hat{l})]\no 
        & &+\frac{1}{2m} \omega_2 \sum_{\hat{l}=1,2,3}\sum_{\vn}\sum_{i,j}
                          [a^\dagger_{ij}(\vn) a_{ij}(\vn+2\hat{l})
                          +a^\dagger_{ij}(\vn) a_{ij}(\vn-2\hat{l})]+\cdots           
\eea 
where $i,j$ are spin and isospin index. 

{\color{red} Following expressions are not clear....(note that this is a 4-dimensional momentum not 3-dimensional)}

In momentum space, .
\bea 
S^E_{NN}&\to& \sum_{\vk,ij} 
\tilde{c}^\dagger_{ij}(-\vk)\tilde{c}_{ij}(\vk)
[e^{-ik_0}
-e^{-6h}-2 h \sum_{l_s}\cos(\vk\cdot\hat{l}_s)] 
\eea 
This corresponds to free neutron propagator as
\bea 
D_N(\vk)=\frac{1}{ e^{-ik_0}
	-e^{-6h}-2 h \sum_{l_s}\cos(\vk\cdot\hat{l}_s)}.
\eea 
And, the free nucleon correlation,
\bea 
\la c'_i(\vn)c^*_i(0) \ra 
=\frac{\int Dc' D c^* c'_i(\vn) c^*(0) \exp[-S_{NN}]}
{\int Dc' D c^* \exp[-S_{NN}]}
=\frac{1}{L_t L^3} \sum_{\vk}e^{-i\vk_*\cdot\vn} D_N(\vk)
\eea

\subsubsection{Momentum space expression(?)} 
Another way to compute the transfer matrix is to use DFT. 
\bea 
& &\int_{t}^{t+a_t} dt_E \int d^3 x {\cal H}_{free}\no 
&\to& a_t a^3\sum_{\bm n}a^\dagger({\bm n},n_t)\widehat{\left(-\frac{\nabla^2}{2m}\right) } a({\bm n},n_t)
=a_t a^3\sum_{\bm n}a^\dagger({\bm n},n_t)\widehat{\left(-\frac{\nabla^2}{2m}\right) } 
         \frac{1}{N^3}\sum_{{\bm m}} \hat{a}({\bm m},n_t) e^{i{\bm n}\cdot{\bm k_m}} \no 
&=& a_t a^3\sum_{\bm n}\frac{1}{N^3}\sum_{{\bm m}} 
 a^\dagger({\bm n},n_t)\left(\frac{\bm k_m^2}{2m}\right) \hat{a}({\bm m},n_t) e^{i{\bm n}\cdot{\bm k_m}} 
 \no          
 &=& a_t a^3\sum_{\bm n}a^\dagger({\bm n},n_t)
   \frac{1}{N^3}\sum_{{\bm m}} \hat{b}({\bm m},n_t) e^{i{\bm n}\cdot{\bm k_m}}
 = a_t a^3\sum_{\bm n}a^\dagger({\bm n},n_t) b({\bm n},n_t)
\eea 
where ${\bm k_m}=2\pi(\frac{m_1}{N},\frac{m_2}{N},\frac{m_3}{N})$ and 
\bea 
& &\hat{a}({\bm k_m},n_t)=\underbrace{\sum_{{\bm n}}a({\bm n},n_t)e^{-i{\bm n}\cdot{\bm k}_m}}_{FFT},\quad 
a({\bm n},n_t)=\frac{1}{N^3}\sum_{{\bm m}} \hat{a}({\bm m},n_t) e^{i{\bm n}\cdot{\bm k}_m} \no 
& &b({\bm n},n_t)=\frac{1}{N^3}\underbrace{\sum_{\bm m}\left(\frac{\bm k_m^2}{2m}\right) \hat{a}({\bm m},n_t)
                                e^{i{\bm n}\cdot{\bm k_m}} }_{iFFT}
\eea 



\section{instantaneous free pion lattice action}
Instantaneous pion action is
\bea 
-S_E[\pi_I]&=& -\int d\tau \int d^3 x \left[\frac{1}{2}(\nabla \pi_I)^2 
 +\frac{1}{2}m_\pi^2\pi_I^2+V(\pi_I)\right]  \no 
 &=& -\int d\tau \int d^3 x \left[\frac{1}{2} \pi_I (-\nabla^2) \pi_I+\frac{1}{2}m_\pi^2\pi_I^2+V(\pi_I)\right]                          
\eea 
Then free pion action for small time slap is
\bea 
H_{\pi\pi}a_t&=&  \int_{t}^{t+a_t}d\tau \int d\vx \frac{1}{2} {\vec\pi} (-\nabla^2+m_\pi^2){\vec\pi}\no 
 &\to & a_t a^3 \sum_{\vn,j}\frac{1}{2}\pi_j({\bm n},n_t)[\widehat{(-\nabla^2)}+m_\pi^2]\pi_j({\bm n},n_t)
\eea                

\subsubsection{Approximate action} 
If we use a simple approximation for Laplacian, we get
\bea 
&\to& a_t a^3\sum_{\bm n} \frac{1}{2}m_\pi^2\pi_j({\bm n},n_t)\pi_j({\bm n},n_t)
     +a_t a^3\sum_{\bm n}\sum_{\bm l}\frac{1}{2}\frac{1}{a^2}
                   \pi_j({\bm n},n_t)[2\pi_j({\bm n},n_t)-\pi_j({\bm n}+{\bm l},n_t)-\pi_j({\bm n}-{\bm l},n_t)  ]
                   \no 
&=&  \alpha_t a^4\sum_{\bm n} \frac{1}{2}\left( m_\pi^2+\frac{6}{a^2}\right) 
         \pi_j({\bm n},n_t)\pi_j({\bm n},n_t) \no  & &
    +\alpha_t (-\frac{a^4}{a^2}) 
    \sum_{\bm n}\sum_{\bm l} \frac{1}{2}\pi_j({\bm n},n_t)
    [\pi_j({\bm n}+{\bm l},n_t)+\pi_j({\bm n}-{\bm l},n_t)  ]    \no     
&=& \alpha_t a^4\sum_{\bm n} \frac{1}{2}\left( m_\pi^2+\frac{6}{a^2}\right) 
\pi_j({\bm n},n_t)\pi_j({\bm n},n_t)
+\alpha_t (-\frac{a^4}{a^2}) 
\sum_{\bm n}\sum_{\bm l}\pi_j({\bm n},n_t)\pi_j({\bm n}+{\bm l},n_t)
\eea 

where the last line uses the fact that $\pi$ is a real scalar. 

We may rescale pion field to simplify the action,
\bea 
\pi'_I(\vn)=\sqrt{q_\pi} \pi_I(\vn), \quad q_\pi=\alpha_t(m_\pi^2+6).
\eea 
This position space action as (set $a=1$ )
\bea 
\boxed{ 
S_{\pi\pi}^E(\pi')=\frac{1}{2}\sum_{\vn,j}\pi'_j(\vn,n_t)\pi'_j(\vn,n_t) 
               -\frac{\alpha_t}{q_\pi}\sum_{\vn,\vl,j}\pi'_j(\vn,n_t)\pi'_j(\vn+\vl,n_t) 
               }
\eea 


{\color{red} Following expressions need clarification.. (here only 3-d DFT is used)}

On the other hand, if we convert approximate action into momentum space action 
for original pion, it can be written as
\bea
S_{\pi\pi}^E(\pi)\to \frac{1}{L^3} \sum_{\vk,j} \frac{1}{2}\tilde{\pi}_j(-\vk,n_t)\tilde{\pi}_j(\vk,n_t)
\alpha_t\left[  
(m_\pi^2+6)-\sum_{\vl} 2 \cos(\frac{2\pi}{L}\vk\cdot\vl)     \right] 
\eea 
or for rescaled pion
\bea 
S_{\pi\pi}^E(\pi')&=& \frac{1}{L^3} \sum_{\vk,I} \frac{1}{2}  
\pi'_I(-\vk,n_t)\pi'_I(\vk,n_t)\left[ 1
-\frac{2\alpha_t}{q_\pi}\sum_{l_s} \cos(\frac{2\pi k_{l_s}}{L})\right], 
\eea 
\footnote{ 
	where, from 
	$\sum_{\vk}\tilde{\pi}_j(-\vk)\tilde{\pi}_j(\vk) f(\vk)
	=\sum_{\vk}\tilde{\pi}_j(-\vk)\tilde{\pi}_j(\vk) f(-\vk),
	$
	we can replace $e^{i\vk\cdot\vl}\to =\frac{1}{2}(e^{i\vk\cdot\vl}+e^{-i\vk\cdot\vl})
	=\cos(\vk\cdot\vl)$ for pion. }
This corresponds to the pion propagator
\bea 
D_\pi(\vk)=\frac{1}{2\left[ (\frac{m_\pi^2}{2}+3)\alpha_t
	-\alpha_t\sum_{\vl_s} \cos(\vk\cdot\vl_s)\right]}.
\eea 

\bea 
\la \pi(\vn)\pi(0)\ra=
\frac{\int D \pi  \pi(\vn)\pi(0) \exp[-S_{\pi\pi}] }{\int D \pi \exp[-S_{\pi\pi}]} 
=\frac{1}{L_t L^3} \sum_{\vk} e^{-i\vk_{*}\cdot\vn} D_\pi(\vk) 
\eea

Thus, for rescaled pion, 
\bea 
\la \pi'(\vn)\pi'(0)\ra=\frac{1}{L_t L^3}\sum_{\vk}e^{-i\frac{2\pi}{L_t} k_t\cdot n_t}
e^{-i\frac{2\pi}{L}\vk_s\cdot\vn_s} D_\pi(\vk_s),
\eea 
\bea 
D_\pi(\vk_s)=\frac{1}{1-\frac{2\alpha_t}{q_\pi}\sum_{l_s=1,2,3}\cos(\frac{2\pi k_{l_s}}{L})}
\eea 
Note here
\bea
(\frac{\alpha_t}{q_\pi}D(\vk_s))^{-1}
&=&\frac{q}{\alpha_t}-2\sum_{l_s}\cos(\frac{2\pi k_{l_s}}{L})
\simeq m_\pi^2+6 -2\sum_{l_s=1,2,3} \left[1-\frac{1}{2}(\frac{2\pi k_{l_s}}{L})^2
+\frac{1}{24}(\frac{2\pi k_{l_s}}{L})^4+\cdots \right] \no 
&\simeq& m_\pi^2+k_{*1}^2+k_{*2}^2+k_{*3}^2
-\frac{1}{12}(k_{*1}^4+k_{*2}^4+k_{*3}^4)+\dots 
\eea 

It is useful to define the two-derivative pion correlator,
\bea 
G_{S_1S_2}(\vn)&=&\la \Delta_{S_1} \pi'_I(\vn,n_t)\Delta_{S_2} \pi'_I({\vec 0},n_t)\ra 
\quad \mbox{(no sum on I) }\no 
&=& \frac{1}{16}\sum_{\nu_{1,2,3}=0,1}\sum_{\nu'_{1,2,3}=0,1}
(-1)^{\nu_{S_1}}(-1)^{\nu'_{S_2}}
\la \pi'_I(\vn+{\vec{\nu}}-{\vec{\nu}}', n_t) \pi'_I({\vec 0},n_t)\ra.
\eea 
with
\bea
\la \pi'_I(\vn,n_t)\pi'_I(0,n_t)\ra
&=&\frac{1}{L^3}\sum_{\vk} e^{-i\frac{2\pi}{L}\vk\cdot\vn} D_\pi(\vk), \no  
D_\pi(\vk)&=&\frac{1}{1-\frac{2\alpha_t}{q_\pi}\sum_{l} \cos(k_l)},
\quad k_l=\frac{2\pi}{L}\vk\cdot{\bm l}.
\eea 
We may further simplify the free pion action by rescaling pion field in momentum space,
\bea 
\pi'_I(-\vk,n_t)&=&\pi_I(-\vk,n_t)\sqrt{ m_\pi^2+6-\sum_{l}2\cos(\frac{2\pi}{L}\vk\cdot\vl)},\no 
\pi'_I(\vk,n_t)&=&\pi_I(\vk,n_t)\sqrt{ m_\pi^2+6-\sum_{l}2\cos(\frac{2\pi}{L}\vk\cdot\vl)} 
\eea 
so that the free pion action becomes in configuration space and momentum space as
\bea 
S_{\pi\pi}(\pi')=\frac{1}{2}\sum_{\vn,I} \pi'_I(\vn)\pi'_I(\vn)
=\frac{1}{L^3}\frac{1}{2}\sum_{\vk,I} \pi'_I(-\vk)\pi'_I(\vk)
\eea 

\subsubsection{Momentum space expression} 
If we set pions as real valued, we have $\hat{\pi}({\bm m})=\hat{\pi}^*(-{\bm m})$, 
\bea 
H_{\pi\pi}a_t
&\to& a_t a^3 \sum_{j}\sum_{\bm n} (\frac{1}{N^3})^2\sum_{\bm m,m'}
  \frac{1}{2}\hat{\pi}_j(-{\bm m'})\hat{\pi}({\bm m}) (m_\pi^2+{\bm q}_m^2)
   e^{i({\bm q}_{m}-{\bm q}_{m'})\cdot{\bm n}}.
\eea 
If we define
\bea 
\hat{\phi}({\bm m})=\sqrt{m_\pi^2+{\bm q}_m^2}\hat{\pi}({\bm m}),\quad 
\hat{\phi}(-{\bm m})=\sqrt{m_\pi^2+{\bm q}_m^2}\hat{\pi}(-{\bm m})
\eea 
Then, from iFFT of $\hat{\phi}({\bm m})$, we get simple free action,
\bea 
& &{\phi}({\bm n})=\frac{1}{N^3}\sum_{\bm m} \hat{\phi}({\bm m})e^{i{\bm q}_m\cdot{\bm n}}
               =\frac{1}{N^3}\sum_{\bm m} \hat{\phi}(-{\bm m})e^{-i{\bm q}_m\cdot{\bm n}} ,\no 
& &H_{\pi\pi}a_t\to a_t a^3 \sum_{j}\sum_{\bm n} \frac{1}{2}\phi_j({\bm n})\phi_j({\bm n}).
\eea 




\newpage 

\section{Lattice action for Cold atom}
Consider a cold atom system which interacts each other with short range interaction.
Cold atom can have spins. Then, we can write the effective Hamiltonian at leading order
as
\bea 
H_{LO}&=& H_{free}+V_{LO},\no 
H_{free}&=&\frac{1}{2m}\sum_{i=\uparrow,\downarrow}\int d^3\vr \nabla a_i^\dagger(\vr)\cdot
 \nabla a_i(\vr),\no 
V_{LO}&=& \frac{C}{2}\int d^3\vr : [\rho^{a^\dagger,a}(\vr)]^2:,\no 
\eea 
Then, we will have discretized lattice action,
\bea 
{\cal Z}={\rm Tr}[M^{L_t}],
\eea 
where,
\bea
M=:\exp\left\{ -H_{free}\alpha_t-\frac{1}{2}C\alpha_t\sum_{\vn}[\rho^{a^\dagger,a}(\vn)]^2
  \right\}:
\eea
Note the auxiliary coupling is not introduced yet. 

If we introduce auxiliary field $s(\vn)$, we get 
\bea 
{\cal Z}=\int {\cal D}s e^{-\frac{1}{2}\sum_\vn s(\vn)^2} {\rm Tr}[M(s)^{L_t}]
\eea 
where
\bea 
M(s)=:\exp\left\{ -H_{free}\alpha_t+\sqrt{-C\alpha_t}\sum_{\vn}\rho^{a^\dagger,a}(\vn) s(\vn)
\right\}:
\eea 
\section{Lattice action for Pionless EFT} 
In case of pionless EFT, we have to consider isospin too. 
If we denote spins $i=0,1$ as spin up and down and isospin $j=0,1$ as
isospin up and down, we can think of following local density operators.

Then, pionless EFT interaction becomes
\bea 
H_{LO}&=&H_{free}+V_{LO},\no 
H_{free}&=&\frac{1}{2m}\sum_{ij=0,1}\int d^3\vr \nabla a_{ij}^\dagger(\vr)\cdot\nabla a_{ij}(\vr),
\no 
V_{LO}&=&V+V_{I^2}+V^{3N},\no 
V&=&\frac{C}{2}\int d^3\vr :[\rho(\vr)]^2:,\no 
V_{I^2}&=&\frac{C_{I^2}}{2}\sum_{I=1,2,3}\int d^3\vr :[\rho_I(\vr)]^2:,\no 
V^{(3N)}&=& \frac{D}{6}\int d^3\vr :\rho(\vr)^3:
\eea 
Note that here $V^{3N}$ is required at leading order.

Then, the discretized lattice action becomes
\bea 
{\cal Z}={\rm Tr}[M^{L_t}],
\eea 
where,
\bea
M=:\exp\left\{ -H_{free}\alpha_t-\frac{1}{2}C\alpha_t\sum_{\vn}[\rho^{a^\dagger,a}(\vn)]^2
  -\frac{1}{2}C_{I^2}\alpha_t \sum_{\vn,I}[\rho^{a^\dagger,a}_I(\vn)]^2
  -\frac{1}{6}D\alpha_t\sum_{\vn}[\rho^{a^\dagger,a}(\vn)]^3
  \right\}:
\eea

{\color{red} Can we introduce auxiliary field ? D-term may be problematic.
	Refer next section for other terms with auxiliary field.} 

\section{Lattice action of LO Chiral EFT}
\subsection{Continuum limit}
The LO chiral Lagrangian is,
\bea 
{\cal L}&=&\frac{1}{2}\del_\mu{\vec{\pi}}\cdot\del^\mu{\vec{\pi}}
        -\frac{1}{2}m_\pi^2{\vec{\pi}}^2
        +N^\dagger i\del_0 N+N^\dagger \frac{\nabla^2}{2m} N \no  & &
        -\frac{g_A}{2f_\pi} N^\dagger{\bm\tau}\vs N\cdot\nabla{\bm \pi}
        -\frac{1}{2}C(N^\dagger N N^\dagger N)
        -\frac{1}{2}C_I (N^\dagger{\bm \tau}N)\cdot (N^\dagger{\bm \tau}N),
\eea  
which give rise the LO NN potential,
\bea 
V_{LO}=C+C_I\tau_1\cdot\tau_2 
   -(\frac{g_A}{2f_\pi})^2\tau_1\cdot\tau_2\frac{\vs_1\cdot\vq\vs_2\cdot\vq}{\vq^2+m_\pi^2}.
\eea 
Treating pions instantaneously\footnote{ 
In other words, pion does not have time derivative terms 
and acts as an auxiliary field which reproduce the leading order 
OPE potential.
}
we have continuum LO action with instantaneous pion as
\bea
-S_{E}&=&\int d\tau d^3 x {\cal L}_E ,\no 
{\cal L}_E&=&-\frac{1}{2}\nabla{\vec{\pi}}\nabla{\vec{\pi}}
             -\frac{1}{2}m_\pi^2 {\vec{\pi}}^2
             -N^\dagger \del_t^E N
             +N^\dagger \frac{\nabla^2}{2m} N  \no & & 
                     -\frac{g_A}{2f_\pi} N^\dagger{\bm\tau}\vs N\cdot\nabla{\bm \pi}
                     -\frac{1}{2}C(N^\dagger N N^\dagger N)
                     -\frac{1}{2}C_I (N^\dagger{\bm \tau}N)\cdot (N^\dagger{\bm \tau}N).
\eea 
The detailed procedure for the discretization will be continued on following sections.

We may use two different method to write the transfer matrix. 
By integrating out pions, we would get pure nucleon-nucleon interactions.
This form can be used to obtain $Z_\Psi=\la \Psi|M^{L_t}|\Psi\ra$. 
However, this form will be too complicate to be used for many nucleons.
Thus, we may keep pions as auxiliary fields and also introduce 
other auxiliary fields to remove contact interactions from the transfer matrix,
so that $Z_\Psi=\int D\pi Ds e^{-S_{\pi}-S_{s}} \la \Psi|M^{L_t}(\pi,s)|\Psi\ra $. 


\subsection{pion-nucleon term in approximate derivative form}
Interacting Hamiltonian for pion-nucleon term is
\bea 
H_{\pi NN}=\frac{g_A}{2f_\pi} \int d\vx  N^\dagger{\bm\tau}_I \vs_{S} N\cdot\nabla_{S} {\pi}_I
\eea 

We get
\bea 
H_{\pi NN}\alpha_t &\to& \frac{g_A}{2f_\pi}
            \alpha_t a^4 \sum_{\vn}\sum_{S,i,j}\sum_{I,i',j'}
            c^\dagger_{i,i'}(\vn,n_t) (\vs_{S})_{ij}(\tau_{I})_{i'j'}
            c_{j,j'}(\vn,n_t)\times \hat{\nabla}_S \pi_I(\vn,n_t) .
\eea 
One may approximate the derivative as
\bea 
\hat{\nabla}_S \pi_I(\vn,n_t)
\to \frac{1}{2a}[\pi_I(\vn+\hat{l}_s)-\pi_I(\vn-\hat{l}_s)].
\eea 
However, this form has disadvantage by computing derivative coarsely with two points values 
in  two step size.  
It is not necessary for pions to be defined at the same lattice position as nucleons.
Instead if we define pions at shifted positions(If there is no terms like $\pi N N$,
pion does not need to be evaluated at the same point of nucleons.)
\bea 
\vn_{pion}=\vn_{nul}-\frac{1}{2}\hat{1}-\frac{1}{2}\hat{2}-\frac{1}{2}\hat{3},
\eea 
then, One way to approximate $\Delta_S\pi(\vn)$ is,
to average surrounding pions near $\vn_{nuc}$\footnote{
For example, in 2-d, around $\vn$, 
\bea 
\Delta_{1} \pi(\vn)&=&\frac{1}{2}[\Delta_{1} \pi(\vn_{pion})+\Delta_{1} \pi(\vn_{pion}+\hat{2})]
\no 
                   &=&\frac{1}{2}[\frac{1}{2}(\pi(\vn_{pion}+\hat{1})-\pi(\vn_{pion}))
                               +\frac{1}{2}(\pi(\vn_{pion}+\hat{2}+\hat{1})
                                -\pi(\vn_{pion}+\hat{2}))]
\eea 
} , thus have approximation 
\bea 
\Delta_S \pi'_I(\vn_{nuc})\to \frac{1}{4}\sum_{\nu_1,\nu_2,\nu_3=0,1}
    (-1)^{\nu_S+1}\pi'_{I}(\vn_{pion}+{\bm \nu} ),
    \quad {\bm \nu}=\nu_1\hat{1}+\nu_2\hat{2}+\nu_3\hat{3}.
\eea 
Thus, with scaled pion, 
\bea 
S_{\pi N}^E &\to& \frac{g_A \alpha_t}{2f_\pi\sqrt{q_\pi}}
             \sum_{\vn}\sum_{S,I} \Delta_S \pi'_I(\vn) \rho_{S,I}(\vn),\no 
\rho_{S,I}(\vn)&=&\sum_{i,j,i',j'} 
            c^\dagger_{i,i'}(\vn) (\vs_{S})_{ij}(\tau_{I})_{i'j'}
            c_{j,j'}(\vn) 
\eea 

\subsection{pion nucleon term in momentum space}
Let us consider derivative couping of pion and nucleon in momentum space
for more exact calculation.
As shown in free pion, by rescale pion field, we can get a very simple free pion action,
\bea 
& &\hat{\phi}({\bm m})=\sqrt{m_\pi^2+{\bm q}_m^2}\hat{\pi}({\bm m}),\quad 
\hat{\phi}(-{\bm m})=\sqrt{m_\pi^2+{\bm q}_m^2}\hat{\pi}(-{\bm m}),\no 
& &{\phi}({\bm n})=\frac{1}{N^3}\sum_{\bm m} \hat{\phi}({\bm m})e^{i{\bm q}_m\cdot{\bm n}}
=\frac{1}{N^3}\sum_{\bm m} \hat{\phi}(-{\bm m})e^{-i{\bm q}_m\cdot{\bm n}} ,\no 
& &H_{\pi\pi}a_t\to a_t a^3 \sum_{j}\sum_{\bm n} \frac{1}{2}\phi_j({\bm n})\phi_j({\bm n}).
\eea 
Then for $\pi N$ coupling term, ({\color{red}mistake for $a^4$?})
\bea 
H_{\pi NN}\alpha_t &\to& \frac{g_A}{2f_\pi}
\alpha_t a^4 \sum_{\vn}\sum_{S,i,j}\sum_{I,i',j'}
c^\dagger_{i,i'}(\vn,n_t) (\vs_{S})_{ij}(\tau_{I})_{i'j'}
c_{j,j'}(\vn,n_t)\times \hat{\nabla}_S \pi_I(\vn,n_t) .
\eea 
Because in the code, it would be easier to use path integral over scaled pion $\phi(\vn)$ (or $\pi'_I(\vn)$),
to compute $\pi-N$ coupling, one have to convert $\pi'_I$ into true pion field $\pi_I$.
Considering DFT, (suppressing isospin and time index)

Thus, for given $\pi'_I(\vn)$ we can obtain $\pi_I(\vn)$ by the successive
Fourier transformation and division. 
(I.e. convert $\pi'_I(\vn)\to \hat{\pi}'_I(\vq)\to \hat{\pi}_I(\vq)\to \pi_I(\vn)$.) 
\bea 
\hat{\pi}(\vq)=\sum_{\vn'} e^{-i\vq\cdot\vn'} \pi(\vn') 
=\sum_{\vn'} e^{-i\vq\cdot\vn'} \left( 
\frac{1}{L^3}\sum_{\vp} e^{i\vp\cdot\vn'} \frac{\pi'_I(\vp)}{\sqrt{\vp^2+m_\pi^2}}         
\right) 
\eea 

As already shown in the first chapter, one can introduce qhop function, 
\bea 
H_{\pi NN}\alpha_t &\to& \frac{g_A}{2f_\pi}
\alpha_t a^4 \sum_{\vn}\sum_{S,i,j}\sum_{I,i',j'}
c^\dagger_{i,i'}(\vn,n_t) (\vs_{S})_{ij}(\tau_{I})_{i'j'}
c_{j,j'}(\vn,n_t)\times \sum_{n'_S}\pi ({\bm n}+n'_S\hat{s})\Delta(n'_S).
\nonumber
\eea 

\section{nucleon-nucleon contact term in lattice}  
At leading order, the contact terms can be written as 
\bea
S^E_{4N}=\frac{C\alpha_t}{2}\sum_{\vn}[\rho(\vn)]^2
          +\frac{C_I\alpha_t}{2}\sum_{\vn}\sum_{I=1,2,3} [\rho_I(\vn)]^2,
\eea 
\bea 
\rho(\vn)&=&\sum_{i,j=0,1} c^*_{i,j}(\vn) c_{i,j}(\vn),\no 
\rho_I(\vn)&=& \sum_{i,j,j'=0,1} c^*_{i,j}(\vn)[\tau_I]_{jj'}c_{i,j'}(\vn) 
\eea 

As shown in the first chapter, we may introduce auxiliary field $s$ and $s_I$,
\bea 
\exp\left(-\frac{C\alpha_t}{2}[\rho(\vn)]^2\right) 
=\frac{1}{\sqrt{2\pi}} \int_{-\infty}^{+\infty}
 ds \exp\left[-\frac{1}{2}s^2+\sqrt{-C\alpha_t}\rho(\vn)\cdot s\right],
\eea 
\bea 
\exp\left(-\frac{C_I \alpha_t}{2}\sum_{I}[\rho_I(\vn)]^2\right) 
=\int\prod_I \frac{ds_I}{\sqrt{2\pi}}  
 \exp\left[-\frac{1}{2}\sum_{I} s^2_I
           +i \sqrt{C_I\alpha_t}\sum_{I} \rho_I(\vn)\cdot s_I\right],
\eea 
where, $C<0$ and $C_I>0$.
Define auxiliary action,
\bea 
S_{ss}(s,s_I)=\frac{1}{2}\sum_{\vn} s(\vn)^2+\frac{1}{2}\sum_{\vn}\sum_{I} s^2_I(\vn)
\eea 
\bea 
\exp[-S^E_{4N}]
&\propto& \int D s D s_I \exp[-S_{ss}(s,s_I)]
                \exp\left[{\color{red}+}\sum_{\vn} \sqrt{-C\alpha_t}\rho(\vn)\cdot s
                     {\color{red}+}\sum_{\vn} i \sqrt{C_I\alpha_t}\sum_{I} \rho_I(\vn)\cdot s_I
                     \right]\no 
     &=& \int D s D s_I \exp[-S_{ss}(s,s_I)-S_{sNN}(s,s_I,c,c^*)]               
\eea 
because of measure, there are normalization factors in exact equation but we 
would not need to specify. 

\subsection{Equivalence of different expressions of LO contact interaction}
Let us summarize different possible expressions of LO contact interaction.
\begin{itemize} 
\item Original Weinberg contact interaction at leading order is
\bea 
:-\frac{1}{2}C'_1\bar{N}N\bar{N}N-\frac{1}{2}C'_2\bar{N}\vs N\cdot\bar{N}\vs N:
\eea 
where, $C'_1<0$ and $C'_2<0$.
\item It is preferred to have positive coefficients because of reducing sign problem,
we may write it as
\bea 
:-\frac{1}{2}C_1\bar{N}N\bar{N}N-\frac{1}{2}C_2\bar{N}\tau N\cdot\bar{N}\tau N:
\eea 
The equivalence with original Weinberg's form can be shown by using identity
\bea 
:\bar{N}N\bar{N}N:=-\frac{1}{2}:\bar{N}{\vec \sigma} N\cdot\bar{N}{\vec \sigma} N:
-\frac{1}{2}:\bar{N}{\vec \tau} N\cdot\bar{N}{\vec \tau} N: 
\eea 
and we got $C_2=-C'_2>0, C_1=C'_1-2 C'_2<0$.
\item Another possible form is
\bea 
C_{S=0,I=1}\left(\frac{1}{4}-\frac{1}{4}\vs_1\cdot\vs_2\right)
      \left(\frac{3}{4}+\frac{1}{4}\tau_1\cdot\tau_2\right) 
+C_{S=1,I=0}\left(\frac{3}{4}+\frac{1}{4}\vs_1\cdot\vs_2\right)
      \left(\frac{1}{4}-\frac{1}{4}\tau_1\cdot\tau_2\right)        
\eea 
The equivalence with other form can be shown by requiring anti-symmetric 
two nucleon states for S-wave $P_s=+1$
\bea 
P_s P_\sigma P_\tau=-1,\quad P_\sigma=\frac{1+\vs_1\cdot\vs_2}{2}.
\eea 
Thus, by using both identity, 
$\vs_1\vs_2\tau_1\tau_2=-3$, $\vs_1\cdot\vs_2=-2-\tau_1\cdot\tau_2$, it 
can be written as
\bea 
\frac{1}{4}(3C_{01}+C_{10})+\frac{1}{4}(C_{01}-C_{10})\tau_1\cdot\tau_2
\eea 
\item By expanding above form we get
\bea 
C_0+C_S \vs_1\cdot\vs_2+C_{I} \tau_1\cdot\tau_2 +C_{SI}\vs_1\cdot\vs_2\tau_1\cdot\tau_2
\eea 
such that
\bea 
C_0&=&-3C_{SI}=-\frac{3}{2}(C_S+C_I), \no 
C_0&=& \frac{3}{16}(C_{01}+C_{10})=-3C_{SI},\no 
C_{S}&=&\frac{1}{16}(-3C_{01}+C_{10}),\quad C_{I}=\frac{1}{16}(C_{01}-3C_{10}).
\eea 
\end{itemize}
Though all of them are equivalent, for numerical calculation some form may be preferred.

If we smear the contact interaction, we may consider it as a kind of mixture of leading order
and higher order interaction. 


\section{Chiral EFT action in lattice}
Let us summarize the lattice action for chiral EFT. 

{\color{red} However, the expressions given here use approximate free action and
	scaled pions. Thus, somewhat outdated. }

\subsection{Transfer matrix without pions and auxiliary fields} 

In case of approximate derivative form, 
\bea 
{\cal Z}_{LO}\propto {\rm Tr}\left[
           M_{LO}^{(N_t-1)}\cdots M_{LO}^{(0)}\right],
\eea 
\bea 
M^{(n_t)}&=& : \exp \Big[ -H_{free}\alpha_t-\frac{1}{2}C\alpha_t \sum_{\vn_s}[\rho(\vn_s)]^2
                  -\frac{1}{2}C\alpha_t \sum_{\vn_s}\sum_{I} [\rho_I (\vn_s)]^2 
                  \no & & 
                  +\frac{g_A^2 \alpha_t^2}{8 f_\pi^2 q_\pi}
                  \sum_{S_1,S_2,I}\sum_{\vn_{s1}\vn_{s2}}
                  G_{S_1S_2}(\vn_{s1}-\vn_{s2})\rho_{S_1,I}(\vn_{s1})\rho_{S_2,I}(\vn_{s2})
                  \Big]:
\eea 
where\footnote{ 
	This can be related with continuum expression,(check?)
	\bea 
	V^{OPE}&=& \sum_{S_1,S_2,I}\int d^3 \vr_1 d^3 \vr_2 G_{S_1 S_2}(\vr_1-\vr_2)
	:\rho_{S_1,I}(\vr_1)\rho_{S_2,I}(\vr_2):,\no 
	G_{S_1 S_2}(\vr_1-\vr_2)&=&
	-(\frac{g_A}{2 f_\pi})^2 \int \frac{d^3 q}{(2\pi)^3}
	\frac{q_{S_1}q_{S_2} e^{i\vq\cdot(\vr_1-\vr_2)}}{q^2+m_\pi^2}
	\eea 
}

\bea 
G_{S_1 S_2}(\vn_s)&=&\frac{\int D\pi'_I \Delta_{S_1}\pi'_I(\vn_s) 
           \Delta_{S_2}\pi'_I(0) \exp[-S_{\pi\pi}]} {\int D\pi'_I \exp[-S_{\pi\pi}]}\no 
      &=&\frac{1}{16}\sum_{\nu_1,\nu_2,\nu_3=0,1}
                     \sum_{\nu'_1,\nu'_2,\nu'_3=0,1}
                     (-1)^{\nu_{S_1}}(-1)^{\nu'_{S_2}}
                     \la \pi'_I(\vn_s+\nu-\nu')\pi'_I(0)\ra 
\eea 

or using coefficient function in momentum space,
\bea 
M_{LO}&=& :\exp\Big\{ -H_{free}\alpha_t -\frac{\alpha_t}{L^3}\sum_{\vq} f(\vq)
    [C_{S=0,I=1} V_{S=0,I=1}(\vq)+ C_{S=1,I=0} V_{S=1,I=0}(\vq)]
    \no & &
    +\frac{g_A^2\alpha_t^2}{8f_\pi^2 q_\pi} \sum_{S_1,S_2,I}
    \sum_{\vn_1,\vn_2}G_{S_1 S_2}(\vn_1-\vn_2)\rho_{S_1,I}(\vn_1)\rho_{S_2,I}(\vn_2)
\Big\}: 
\eea 
Coefficient function
\bea 
f(\vq)=f_0^{-1} \exp[-b\sum_{l}(1-\cos q_l)],
\quad f_0=\frac{1}{L^3}\sum_{\vq} \exp[-b\sum_{l}(1-\cos q_l)].
\eea 

In fact, better form may be using the expression given in the first chapter,
\bea 
V_{OPE}&=&-\frac{g_A^2}{8f_\pi^2}\sum_{n'n,S',S,I} : \rho_{S',I}(n')f_{SS'}(n'-n)\rho_{S,I}(n):
\eea  

\subsection{Transfer matrix with pion but no auxiliary field} 
\bea 
{\cal Z}_{LO}\propto 
            \int {\cal D}\pi'_I \exp[-S_{\pi\pi}(\pi'_I)]{\rm Tr}\left[
                M_{LO}^{(N_t-1)}(\pi'_I)\cdots M_{LO}^{(0)}(\pi'_I)
           \right], 
\eea 
\bea 
M_{LO}^{(n_t)}(\pi'_I)=:\exp[-H_{LO}^{(n_t)}(\pi'_I)\alpha_t]:
\eea 
where, contact interaction for $C$ and $C_I$ are considered.
\bea 
H_{LO}(\pi'_I)&=& H_{free}+\frac{C}{2}\sum_{\vn_s}[\rho(\vn_s)]^2
                          +\frac{C_I}{2}\sum_{\vn_s}\sum_{I}[\rho_I(\vn_s)]^2
                    \no  & &+\frac{g_A}{2f_\pi}\sum_{S,I}\sum_{\vn_s}
                           \Delta_S \pi'_I(\vn_s,n_t)\rho_{S,I}(\vn_s)
\eea 

\subsection {Transfer matrix with pions and auxiliary fields}

If we use auxiliary formalism, we get
\bea 
{\cal Z}_{LO}&\propto &\int {\cal D}\pi'_I 
                   {\cal D}s{\cal D}s_I
            \exp[-S_{\pi\pi}(\pi'_I)-S_{ss}(s,s_I)]{\rm Tr}\left[
           M_{LO}^{(N_t-1)}(\pi'_I,s,s_I)\cdots M_{LO}^{(0)}(\pi'_I,s,s_I)
           \right],
\eea 
We can write $M_{LO}$ as normlaized integral
\bea 
M_{LO}=\frac{\int D\pi'_I D s e^{-S_\pi- S_s} M^{(n_t)}(\pi'_I,s)}
            {\int D\pi'_I D s e^{-S_\pi- S_s}},
\eea 
with
\bea             
M^{(n_t)}(\pi'_I,s,s_I)&=&:\exp\Big\{
  -H_{free}\alpha_t-\frac{g_A\alpha_t}{2f_\pi\sqrt{q_\pi}}
  \sum_{S,I}\Delta_S \pi'_I(\vn_s,n_t)\rho_{S,I}(\vn_s) \no 
 & &  +\sqrt{-C\alpha_t}\sum_{\vn_s} s(\vn_s,n_t) \rho(\vn_s)  
 +i\sqrt{C_I\alpha_t}\sum_I\sum_{\vn_s} s_I(\vs_s,n_t)\rho_I(\vn_s)
\Big\} :
\eea 
and 
\bea 
S^{(n_t)}_{\pi\pi}(\pi'_I)&=&\frac{1}{2}\sum_{\vn,I}\pi'_I(\vn,n_t)\pi'_I(\vn,n_t)
           -\frac{\alpha_t}{q_\pi}\sum_{\vn,I,l} \pi'_I(\vn,n_t)\pi'_I(\vn+\hat{l},n_t)\no 
S_{ss}^{(n_t)}&=& \frac{1}{2}\sum_{\vn,\vn'} s(\vn,n_t) f^{-1}(\vn-\vn') s(\vn',n_t)
                +\mbox{(other auxiliary fields)} 
\eea 
with 
\bea 
f^{-1}(\vn-\vn')=\frac{1}{L^3}\sum_{\vq}\frac{1}{f(\vq)} e^{-i\vq\cdot(\vn-\vn')}.
\eea 

If one use improved derivative operator, it can be generlaized with hopping coefficients.
\bea 
\exp(-S^{OPEP}_{int}(c^*,c))
=\int \prod_{I} D\pi_I \exp(-S_{\pi_I\pi_I}(\pi_I)-S_{\pi_I}(c^*,c,\pi_I)).
\eea 
with free pion action 
\bea 
S_{\pi_I\pi_I}(\pi_I)&=&\frac{1}{2}\alpha_t m_\pi^2 \sum_{\vn,n_t,I} \pi_I^2(\vn,n_t) \no 
   & &\quad +\frac{1}{2}\alpha_t \sum_{k=0,1,2\dots} (-1)^k w_k \sum_{\vn,n_t,I,\hat{l}} 
      \pi_I(\vn,n_t)\left[ \pi_I(\vn+k\hat{l},n_t)+\pi_I(\vn-k\hat{l},n_t)\right], 
\eea 
where $w_k$ is the same hopping coefficients for free nucleons (because it is an approximation for
Laplacian.)
In above expression $\pi'_I$ is scaled to absorb the $m_\pi^2$ and $w_0$ terms.

The pion coupling to the nucleon is 
\bea 
S_{\pi}(c^*,c,\pi_I)=\frac{g_A\alpha_t}{2f_\pi}\sum_{\vn,n_t,l,I} \Delta_l \pi_I(\vn,n_t) c^*(\vn,n_t) \sigma_l \tau_I c(\vn,n_t),
\eea 
where $l=1,2,3$ and one can approximate the derivative
\bea 
\Delta_l \pi_I(\vn,n_t)=\frac{1}{2}\sum_{k=1,2,\dots} (-1)^{k-1} o_k 
  \left[  \pi(\vn+k\hat{l},n_t)- \pi(\vn-k\hat{l},n_t)\right], 
\eea 
with coefficients $o_k$ corresponding to a hopping parameter expansion of momentum,
\bea 
P(p_l)=\sum_{k=1,2,\dots}(-1)^{k-1} o_k \sin(k p_l)
\eea 
The hopping coefficients are chosen to match the continuum result, $P(p_l)=p_l$.

For example, for $k=1$, $o_1=1,o_2=0,o_3=0$,
for $k\leq 2$, $o_1=\frac{4}{3}, o_2=\frac{1}{6}, o_3=0$,
and for $k\leq 3$, $o_1=\frac{3}{2},o_2=\frac{3}{10},o_3=\frac{1}{30}$.

\section{New non-local contact interaction} 
Ref: PRL119,222505(2017)

To reduce the sign oscillation problem, one introduce new non-local SU(4) interaction. 
\bea 
H_B=H_{free}+V_0+V_{OPE}
\eea 

\subsection{Free Hamiltonian}
\begin{itemize}
	\item $\sum_{\la \vn'\vn\ra}$ : summation over nearest neighbor lattice sites of $\vn$.
	\item $\sum_{\la \vn'\vn\ra_i}$ : sum over nearest neighbor lattice sites of $\vn$ along the i-th spatial axis.
	\item $\sum_{\la\la \vn'\vn\ra\ra_i}$ : sum over next-to-nearest neighbor lattice sites of $\vn$ along the i-th spatial axis.
	\item $\sum_{\la\la\la \vn'\vn\ra\ra\ra_i}$ : sum over next-to-next-to-nearest neighbor lattice sites of $\vn$ along the i-th spatial axis.
\end{itemize}
\bea 
H_{free}&=&\frac{49}{12m}\sum_{\vn} a^\dagger(\vn) a(\vn)
    -\frac{3}{4m}\sum_{\vn,i}\sum_{\la \vn'\vn\ra_i} a^\dagger(\vn')a(\vn)
     \no & & 
    +\frac{3}{40m}\sum_{\vn,i}\sum_{\la\la \vn'\vn\ra\ra_i} a^\dagger(\vn')a(\vn)
    -\frac{1}{180m}\sum_{\vn,i}\sum_{\la\la\la \vn'\vn\ra\ra\ra_i} a^\dagger(\vn')a(\vn)
\eea 

This is an improved action. Note that the local term got additional factors from 
$$\sum_{l=1,2,3}a^\dagger(\vn)(a(\vn+0\hat{l})+a(\vn-0\hat{l})) =6 a^\dagger(\vn)a(\vn).$$

\subsection{nonlocal operators} 

Nonlocal annihilation and creation operators, $s_{NL}$ is a free parameter to be fitted,
\bea 
a_{NL}(\vn)&=&a(\vn)+s_{NL}\sum_{\la \vn' \vn\ra } a(\vn')\no 
a^\dagger_{NL}(\vn)&=&a^\dagger(\vn)+s_{NL}\sum_{\la \vn' \vn\ra } a^\dagger(\vn'),
\eea 
Point-like densities
\bea 
\rho(\vn)&=&a^\dagger(\vn) a(\vn), \no 
\rho_S(\vn)&=&a^\dagger(\vn)[\sigma_S]a(\vn), \no 
\rho_I(\vn)&=&a^\dagger(\vn)[\tau_I]a(\vn), \no 
\rho_{S,I}(\vn)&=&a^\dagger(\vn)[\sigma_S \tau_I]a(\vn)
\eea 
Smeared nonlocal densities,
\bea 
\rho_{NL}(\vn)&=&a^\dagger_{NL}(\vn) a_{NL}(\vn), \no 
\rho_{S,NL}(\vn)&=&a^\dagger_{NL}(\vn)[\sigma_S]a_{NL}(\vn), \no 
\rho_{I,NL}(\vn)&=&a^\dagger_{NL}(\vn)[\tau_I]a_{NL}(\vn), \no 
\rho_{S,I,NL}(\vn)&=&a^\dagger_{NL}(\vn)[\sigma_S \tau_I]a_{NL}(\vn)
\eea 

The new nonlocal SU(4) interaction is defined as
\bea 
V_0=\frac{c_0}{2}\sum_{\vn',\vn,\vn''} :\rho_{NL}(\vn')f_{s_L}(\vn'-\vn)f_{s_{L}}(\vn-\vn'')\rho_{NL}(\vn''):
\eea 
where, with $c_0, s_L$ are free parameters to be fitted, 
\bea 
f_{s_L}(\vn)&=& 1 \mbox{ for } |\vn|=0, \no 
            &=& s_L \mbox{ for } |\vn|=1, \no 
            &=& 0 \mbox{ otherwise }.
\eea 

The action can be changed with auxiliary field $s$ as shown in the first chapter,
\bea 
V_{ss}^{(n_t)}=\frac{1}{2}s^2(\vn,n_t),
\eea 
\bea 
V_{s}^{(n_t)}={\color{red}-}\sqrt{-c_0}\sum_{\vn\vn'}\rho_{NL}(\vn)f_{s_L}(\vn-\vn')s(\vn',n_t).
\eea 
Note that transfer matrix is $M\propto \exp(-V_s^{(n_t)}\sqrt{\alpha_t})$

\subsection{one pion exchange term}
\bea 
V_{OPE}=-\frac{1}{2}\frac{g_A^2}{4f_\pi^2}\sum_{\vn',\vn,S',S,I} 
         :\rho_{S',I}(\vn')f_{S'S}(\vn'-\vn)\rho_{S,I}(\vn) :, 
\eea  
where
\bea 
f_{S'S}(\vn'-\vn)=\frac{1}{L^3}\sum_{\vq} \frac{\exp(-i\vq\cdot(\vn'-\vn)-b_\pi\vq^2) q_{S'} q_S}{\vq^2+m_\pi^2},
\eea 
and each $q_S$ is an integer multiplied by $2\pi/L$. $b_\pi=0.7$ is chosen.

This OPE exchange can be rewritten as pion action
\bea 
V_{\pi\pi}^{(n_t)}=\frac{1}{2}\sum_{\vn,\vn',I} \pi_I(\vn',n_t)f^{\pi\pi}(\vn'-\vn)\pi_I(\vn,n_t),
\eea 

\bea 
V_{\pi}^{(n_t)}&=&{\color{red}-}\frac{g_A}{2f_\pi}\sum_{\vn,\vn',S,I}\rho_{S,I}(\vn')f^\pi_S(\vn'-\vn)\pi_I(\vn,n_t)
\no 
&=&{\color{red}+}\frac{g_A}{2f_\pi}\sum_{\vn,\vn',S,I} \pi_I(\vn',n_t)f^\pi_S(\vn'-\vn)\rho_{S,I}(\vn)
\eea 
with
\bea 
f_S^\pi(\vn'-\vn)=\frac{1}{L^3}\sum_{\vq} \exp(-i\vq\cdot(\vn'-\vn))({\color{red}i}q_S),
\eea 
\bea 
f^{\pi\pi}(\vn'-\vn)=\frac{1}{L^3}\sum_\vq \exp(-i\vq\cdot(\vn'-\vn)+b_\pi\vq^2)(\vq^2+m_\pi^2).
\eea 

\subsection{transfer matrix}
\bea 
M=\int D s^{(n_t)} D\pi^{(n_t)} M^{(n_t)},
\eea 
\bea 
M^{(n_t)}=:\exp\left(-H_{free}\alpha_t-V_s^{(n_t)}\sqrt{\alpha_t}-V_{ss}^{(n_t)}
   -V_\pi^{(n_t)}\alpha_t-V_{\pi\pi}^{(n_t)}\alpha_t  \right):
\eea 
Where one can use auxiliary pion field $\phi_I(\vn)$,
\bea 
-V_{\pi\pi}^{(n_t)}\alpha_t&=&-\frac{\alpha_t}{2}\sum_{\vn \vn'}\pi_I(\vn)f^{\pi\pi}(\vn'-\vn)\pi_I(\vn) \no 
&=&-\frac{\alpha_t}{2}\frac{1}{L^3}\sum_{\vq} \hat{\pi}_I(\vq)\hat{\pi}_I(-\vq) 
\exp(b_\pi\vq^2)(\vq^2+m_\pi^2) \no 
&=& -\frac{\alpha_t}{2}\frac{1}{L^3}\sum_{\vq} \hat{\phi}_I(\vq)     \hat{\phi}_I(-\vq)
=  -\frac{\alpha_t}{2}\sum_{\vn} {\phi}_I(\vn)     {\phi}_I(\vn) 
\eea  
where
\bea 
\hat{\phi}_I(\vq)=\sqrt{\exp(b_\pi \vq^2)(\vq^2+m_\pi^2)}\hat{\pi}_I(\vq)
\eea 
 
 
One may refer the first chapter for more detailed derivation.  
\newpage


\chapter{Numerical Realization}

If we denote all relevant bosonic fields as $U$, by using transfer matrix formulation,
we would define
\bea 
Z_{\Psi}(L_t)=\int {\cal D}[U] e^{-S[U]}\la \Psi| [M(U)]^{L_t}|\Psi\ra 
\eea 
Any observable can be written as
\bea 
\la O \ra_{\Psi,L_t}= \frac{\int {\cal D}[U] e^{-S[U]}\la \Psi| M\dots O M\dots M |\Psi\ra  }
         { \int {\cal D}[U] e^{-S[U]}\la \Psi| [M(U)]^{L_t}|\Psi\ra  }  
\eea 
By sampling $U$ according to given probability $P[U]$,
we will have 
\bea 
\la O \ra_{\Psi,L_t}\simeq  
          \frac{\frac{1}{N}\sum_{[U]} P[U] \frac{1}{P[U]}
                   e^{-S[U]}\la \Psi| M\dots O M\dots M |\Psi\ra  }
         { \frac{1}{N}\sum_{[U]} P[U] \frac{1}{P[U]} e^{-S[U]}\la \Psi| [M(U)]^{L_t}|\Psi\ra  }
         = \frac{\frac{1}{N}\sum_{[U]\in P[U]}  \frac{1}{P[U]}
                            e^{-S[U]}\la \Psi| M\dots O M\dots M |\Psi\ra  }
                  { \frac{1}{N}\sum_{[U]\in P[U]}  \frac{1}{P[U]} e^{-S[U]}\la \Psi| [M(U)]^{L_t}|\Psi\ra  } 
\eea 
Usually, we define  
\bea 
P[U]=e^{-S[U]}|\det X[U,L_t]|,\quad \det X[U,L_t] =\la \Psi| [M(U)]^{L_t}|\Psi\ra
                                    =|\det X[U,L_t]| e^{i\theta(U,L_t)}
\eea
Then,
\bea 
\la O \ra_{\Psi,L_t}&\simeq&\frac{\sum_{[U]\in P[U]}  e^{i\theta(U,L_t)}
                            \frac{\la \Psi| M\dots O M\dots M |\Psi\ra}{|\det X[U,L_t]|} }
                  { \sum_{[U]\in P[U]} e^{i\theta(U,L_t)}  } 
\eea 
However, in principle we can use any kind of probability distribution $P[U]$.
 
\section{Rough Sketch}
Basically the code structure will be such that
\begin{itemize}
\item[(1)] Initialization, Declaration
 \begin{itemize}
   \item preparation for common arrays
   \item assign random values for auxiliary variables 
 \end{itemize}
\item[(2)] Main Monte Carlo
 \begin{itemize}
   \item make new configuration
   \item Compute Probability of old and new configuration
   \item check acceptance and update the configuration 
 \end{itemize} 
\item[(3)] Computing observables
 \begin{itemize}
   \item Compute observables and averages
 \end{itemize} 
\item[(4)] Finalization 
\end{itemize}

Of course, each steps are composed of many smaller steps. 

The main object is to compute an expectation value of operator

\section{Nuclear wave function}  
In the code, we define initial one body wave functions of $np$-th particle in spin-isospin
state $\alpha$ at lattice point $\vn$ as $v_{np}(\vn,\alpha,n_t=0)$. Then, the many body wave 
function will be a slater determinant of them. However, we only need to keep track of
one-body wave function evolution in auxiliary field formulation. Any boson fields
or auxiliary field configuration can be written as $s(\vn,\beta, n_t)$ for all lattice points
and time. The action of transfer matrix is calculated as
series of single particle state vectors and dual vectors,
\bea 
& &|\phi_i^{(n_t+1)}\ra=M^{(n_t)}(s)|\phi_i^{(n_t)}\ra, \no  
& &\la \tilde{\phi}_i^{(n_t-1)}|= \la \tilde{\phi}_i^{(n_t)}| M^{(n_t)}(s)
\eea 
Then, final matrix element $X$ can be computed by 
\bea 
X_{j' j}(s)&=&\la \phi_{j'}| M^{L_t}|\phi_{j}\ra =
\la v_{j'}(n_t=0)|v_{j}(n_t=L_t) \ra \no 
           &=&\la vd_{j'}(n_t=L_t)|v_{j}(n_t=L_t) \ra
\eea 
where we defined dual vectors,\footnote{Is $M(s)$ a hermitian, real?}
\bea 
|vd_j(n_t-1)\ra=M(n_t)|vd(n_t)\ra ,\quad |vd_j(L_t)\ra=|v_j(0)\ra
\eea 
\footnote{
If we are interested in computing different time length t,
we need to compute $X_{j'j}(s,t=L_t)$ and 
$X_{j'j}(s,t=L_t-1)=\la v_{np=j'}(n_t=0)|v_{np=j}(n_t=L_t-1)\ra $.

} 

In the same way, we can write
\bea 
X_{j' j}(s)&=&\la \phi_{j'}| M^{L_t}|\phi_{j}\ra \no 
           &=&\la \tilde{\phi}^{(L_t)}_{j'}|M^{L_t}|\phi_j^{(0)}\ra \no 
           &=& \la \tilde{\phi}^{(n_t)}_{j'}|\phi_j^{(n_t)}\ra \quad \mbox{for any } n_t 
\eea 

If the initial state is not normalized such as $v_{np}(\vn,ns,ni,n_t=0)=1$, 
whenever the scalar product of vectors appears, they
must be re-normalized by multiplying $1/N^3$.

\section{Transfer matrix calculation }
We want to compute the transfer matrix elements for given 
auxiliary field $s(\vn,n_t)$,
\bea 
\la \Psi| \{ M_A(s,L_t-1)\dots M_A(s,0)   \} |\Psi\ra =\det X(s) , 
\eea
with  
\bea 
M_A(s,n_t)=:\exp\left[-\alpha_t \hat{H}_{free}+\sum_{\vn} A[s(\vn,n_t)] \rho(\vn) \right]:
\eea
and 
\bea 
X_{j' j}(s)=\la \phi_{j'}|M(s,L_t-1)\dots M(s,0)|\phi_{j}\ra.
\eea 

Thus, in effect, even though we are considering multi-particle states,
we only need to computes the action of transfer matrix on a single 
particle state. 

Suppose we have defined a one particle wave vector $|\phi\ra $ as 
\footnote{
For more explicit derivation, one can use single particle state is
\bea 
\la \vn|\phi\ra =\sum_{\vp} c_\vp \la \vn|a^\dagger_{\vp}|0\ra = \sum_{\vp} c_\vp \phi_p(\vn),
\quad a(\vn)=\sum_{\vp} \phi_\vp(\vn) a_\vp 
\eea 
Or one can express state as
\bea 
a(\vn)|f\ra=f(\vn)|0\ra 
\eea 
 }
\bea 
\la \vn|\phi\ra = \la 0|a(\vn)|\phi\ra=\phi(\vn). 
\eea 
We get (since it is a single particle state complete set is $1=|0\ra\la 0|+\sum_{p}|p\ra \la p|$),
\bea 
& &\la \vn|a^\dagger(\vn') a(\vn'))|f\ra=\la \vn|a^\dagger(\vn')|0\ra \la 0|a(\vn'))|f\ra
=\delta_{\vn,\vn'}f(\vn), \no 
& &\la \vm|a^\dagger(\vn')a(\vn)s(\vn)|f\ra=\delta_{\vn',\vm} s(\vn) f(\vn) ,\no 
& &\la \vm|s(\vn') a^\dagger(\vn')a(\vn)|f\ra=\delta_{\vn',\vm} s(\vn') f(\vn) .
\eea 

In general, one-body operators can be written as $\sum_{ij} A_{ij} a^\dagger(\vn_i) a(\vn_j)$
and as long as it acts on one-body state, we can write
\bea 
:\exp(\sum_{ij} A_{ij} a^\dagger(\vn_i) a(\vn_j)):=:1+\sum_{ij} A_{ij} a^\dagger(\vn_i) a(\vn_j):
\eea 
Then
\bea 
\la \vn|:\exp(\sum_{ij} A_{ij} a^\dagger(\vn_i) a(\vn_j)):|\phi\ra 
&=&\la \vn|:1+\sum_{ij} A_{ij} a^\dagger(\vn_i) a(\vn_j):|\phi\ra
=\la \vn|\phi\ra+\sum_{ij}A_{ij} \delta_{\vn,\vn_i} \la \vn_j  |\phi\ra \no 
&=&\la \vn|\phi\ra+\sum_{\vn_j} A_{\vn,\vn_j}\la \vn_j  |\phi\ra
\eea  

In other words,
\begin{framed}
\bea 
\la n| : \left(1+\sum_{ij} A_{ij} a^\dagger(\vn_i) a(\vn_j)\right) :|\phi\ra 
=\la n|\phi\ra +\sum_{j} A_{\vn_i \vn_j}\la \vn_j|\phi\ra   
\eea 
\end{framed} 

For a simple cases,
\bea 
|\phi'_1\ra &=& \sum_{\vn'} a^\dagger(\vn')a(\vn')|\phi\ra,\no 
|\phi'_2\ra &=& \sum_{\vn'} a^\dagger(\vn')a(\vn'-\vl)|\phi\ra=\sum_{\vn'} a^\dagger(\vn'+\vl)a(\vn')|\phi\ra,\no    
|\phi'_3\ra &=& \sum_{\vn'} a^\dagger(\vn')a(\vn'+\vl)|\phi\ra=\sum_{\vn'} a^\dagger(\vn'-\vl)a(\vn')|\phi\ra
\eea 
we have respectively 
\bea 
\la \vn|\phi'_1\ra =\la \vn|\phi\ra ,\quad \la \vn|\phi'_2\ra =\la \vn-\vl|\phi\ra,\quad 
\la \vn|\phi'_3\ra =\la \vn+\vl|\phi\ra  
\eea 
Thus, simply acts as hopping lattice values at lattice points. 
Also note that because of periodic boundary condition of in space.
Actual hopping should be calculated as
\bea 
\la \vn|\sum_{\vn'} a^\dagger(\vn') a(\vn'-\vl)|\phi\ra 
= \la \mbox{mod}(N+\vn -\vl,N)|\phi\ra  
\eea 

Let us consider one transfer matrix acting on one particle state vector,
with implicit spin indices,
\bea 
\la \vn|\phi(n_t+1)\ra=\la \vn| M(s,n_t) |\phi(n_t)\ra
=\la \vn|\left(1-\alpha_t H_{free}+\sum_{\vn'}A[s(\vn',n_t)]\rho(\vn')\right)
|\phi(n_t)\ra.
\eea  
We can easily compute $\la \vn|\phi(n_t+1)\ra$ by considering 'improved action'
with hopping operators.

\bea 
\la {\bm n}|:1-H_{free}\alpha_t:|\phi\ra 
&=&\la {\bm n}|\phi\ra -\frac{\alpha_t}{2m} \sum_{j=0}^{j_{max}}\sum_{\vn',l=1,2,3}
(-1)^j \omega_j \la {\bm n}[a^\dagger(\vn')a(\vn'+j\hat{l})+a^\dagger(\vn')a(\vn'-j\hat{l})]|\phi\ra \no 
&=&\la {\bm n}|\phi\ra -\frac{\alpha_t}{2m} \sum_{j=0}^{j_{max}}\sum_{l=1,2,3}
(-1)^j \omega_j \left( \la {\bm n}+j\hat{l}|\phi\ra +\la {\bm n}-j\hat{l}|\phi\ra   \right) 
\eea 
Interaction with auxiliary is simple except the derivative coupling
\bea 
\phi^{(n_t+1)}_{\alpha}({\bm n}) =\la {\bm n};\alpha|: \sum_{\vn'}A[s(\vn',n_t)]\rho(\vn'): |\phi\ra 
 =\sum_{\beta} A[s(\vn,n_t)] \rho_{\alpha \beta}\phi_{\beta}^{(n_t)}({\bm n}) 
\eea 
where $\alpha,\beta$ represent spin and isospin and $\rho_{\alpha\beta}$ is an matrix in spin and 
isospin. For the last time-step of $\la \phi|M|\phi^{(n_t-1)}\ra$,
one can use
\bea 
\la \phi'|a^\dagger(\vn) a(\vn)|\phi\ra
=\la \phi'|\vn\ra \la \vn|\phi\ra  
\eea 
More detail on spin and isospin is explained in next section. 
\section{More examples of operator action on states}
Let us consider actions of operators on states with spin and isospins.

For one-body states, let us consider a general density operator,
\bea 
\hat{\rho}_{S}(\vn_s)=\sum_{ii'} a^\dagger_{i}(\vn_s)[\vs_S]_{ii'}
                 a_{i'}(\vn_s),
\eea 
where $i$ represents general quantum numbers of one-body states.

Then, for one-body states $|\phi\ra$ in $|\vn, \alpha \ra $ basis, 
\bea 
\la \vn,\alpha|\phi'\ra &=& 
\la \vn,\alpha| \sum_{\vn_s} \hat{\rho}_{S}(\vn_s)|\phi\ra 
= \sum_{\vn_s} \sum_{\vn',\beta}\la \vn,\alpha| \hat{\rho}_{S}(\vn_s)|\vn',\beta\ra 
  \la \vn',\beta   |\phi\ra ,\no  
&=&\sum_{\vn_s,\beta} [\vs_S]_{\alpha\beta} \delta_{\vn,\vn_s}\la \vn_s,\beta|\phi\ra
 =\sum_{\beta} [\vs_S]_{\alpha\beta}\la \vn,\beta|\phi\ra. 
\eea 
For two-body states, $\phi_{2B}$ in $|\vn_1,\alpha_1,\vn_2,\alpha_2\ra $ basis,
\bea 
& &\la \vn_1,\alpha_1,\vn_2,\alpha_2| \sum_{\vn_{s1},\vn_{s2}} 
       : \hat{\rho}_{S_1}(\vn_{s1})\hat{\rho}_{S_2}(\vn_{s2}):|\phi\ra \no 
&=& \sum_{\vn'_1\vn'_2}\sum_{\beta_1,\beta_2} 
    \la \vn_1,\alpha_1,\vn_2,\alpha_2| \sum_{\vn_{s1},\vn_{s2}} 
            :\hat{\rho}_{S_1}(\vn_{s1})\hat{\rho}_{S_2}(\vn_{s2}):
      |\vn'_1,\beta_1,\vn'_2,\beta_2\ra  
      \la \vn'_1,\beta_1,\vn'_2,\beta_2|\phi\ra \no 
&=& \sum_{\beta_1,\beta_2} \Big[ 
    [\vs_{S_1}]_{\alpha_2\beta_2}[\vs_{S_2}]_{\alpha_1\beta_1}
    \la \vn_1,\beta_1,\vn_2,\beta_2|\phi\ra 
   -[\vs_{S_1}]_{\alpha_1\beta_2}[\vs_{S_2}]_{\alpha_2\beta_1}
   \la \vn_2,\beta_1,\vn_1,\beta_2|\phi\ra 
    \no & & 
   -[\vs_{S_1}]_{\alpha_2\beta_1}[\vs_{S_2}]_{\alpha_1\beta_2}
      \la \vn_2,\beta_1,\vn_1,\beta_2|\phi\ra
   +[\vs_{S_1}]_{\alpha_1\beta_1}[\vs_{S_2}]_{\alpha_2\beta_2}
      \la \vn_1,\beta_1,\vn_2,\beta_2|\phi\ra   \Big]
\eea  

\subsection{Non-local operators} 
Let us consider the case of non-local interactions. As introduced, in previous chapter, 
new non-local contact term gives 
\bea 
V_{s}^{(n_t)}={\color{red}-}\sqrt{-c_0}\sum_{\vn}\rho_{NL}(\vn)s_{smear}(\vn,n_t)
\eea 
where I introduced a smeared auxiliary field 
\bea 
s_{smear}(\vn)=s(\vn)+s_L \sum_{\la \vn'\vn\ra } s(\vn'),
\eea 
and Smeared nonlocal densities,
\bea 
\rho_{NL}(\vn)&=&a^\dagger_{NL}(\vn)a_{NL}(\vn), \no 
\eea 
and non-local operators are
\bea 
a_{NL}(\vn)&=&a(\vn)+s_{NL}\sum_{\la \vn' \vn\ra } a(\vn')\no 
a^\dagger_{NL}(\vn)&=&a^\dagger(\vn)+s_{NL}\sum_{\la \vn' \vn\ra } a^\dagger(\vn'),
\eea 
In fact, we can introduce larger smearing with $\la\la \vn'\vn\ra\ra$ and so on, but 
it is a simple extension.  
To compute the action of 
\bea 
|\phi(n_t)\ra =\hat{V}_s(n_t-1)|\phi(n_t-1)\ra, 
\eea 
Compute smeared auxiliary field $s_{smear}(\vn,n_t-1)$
is straightforward. In a similar way, 
we may define $\phi_{smear}(\vn)$ such that 
\bea 
\phi_{smear}(\vn)=\la 0| a_{NL}(\vn)|\phi\ra 
                 =\phi(\vn)+s_{NL}\sum_{l=1,2,3} \left[\phi(\vn+\hat{l})+\phi(\vn-\hat{l})\right].
\eea 
Then, action of transfer matrix can be considered as (constant factor $\sqrt{-c_0\alpha_t}$ is omitted.)
\bea 
M|\phi\ra &=&\left(1-H_{free}-V_{s}\right)|\phi\ra 
          =|\phi_1\ra-V_{s}|\phi\ra =|\phi_{new}\ra, \no  
|\phi_1\ra &=& (1-H_{free})|\phi\ra ,\no 
-V_{s}|\phi\ra&=&+\sum_{\vn}\rho_{NL}(\vn)s_{smear}(\vn,n_t-1)|\phi\ra \no 
              &=& \sum_{\vn} a^\dagger_{NL}(\vn)|0\ra \times s_{smear}(\vn,n_t-1) \phi_{smear}(\vn).
\eea 

Thus, action on a s.p wave function is 
\bea 
\la \vn|\phi_{new}\ra &=& \la \vn|\phi_1\ra  \no 
  & &+ \sum_{\vn'}z_{smear}(\vn')
   \left\la \vn\left| a^\dagger(\vn')+s_{NL}\sum_{l=1,2,3}(a^\dagger(\vn'+\hat{l})+a^\dagger(\vn'-\hat{l})  )   \right|0\right\ra 
\eea 
where $ z_{smear}(\vn)=s_{smear}(\vn,n_t-1) \phi_{smear}(\vn)$.

Another explanation
\bea 
\la \vn|: \sum_{\vn',\vn''} \rho_{NL}(\vn') f_{s_L}(\vn'-\vn'') s(\vn'') :|f\ra 
\eea 
Define
\bea 
s_{smeared}(\vn)&=&\sum_{\vn'}f_{s_L}(\vn-\vn') s(\vn')
=s(\vn) +\sum_{\la \vn',\vn\ra } s_L s(\vn'),  \no 
f_{smeared}(\vn)&=& f(\vn) +\sum_{\la \vn',\vn\ra } s_{NL} f(\vn')                 
\eea 
Then,
\bea 
& & \la \vn|: \sum_{\vn',\vn''} \rho_{NL}(\vn') f_{s_L}(\vn'-\vn'') s(\vn'') :|f\ra  \no 
&=& \sum_{\vn'} \left(\delta_{\vn\vn'}+s_{NL}\sum_{\la \vn'',\vn'\ra }\delta_{\vn,\vn''} \right)
s_{smeared}(\vn') f_{smeared}(\vn')
\eea 


\subsubsection{pion coupling term}
\bea 
V_{\pi}^{(n_t)}=\frac{g_A}{2f_\pi}\sum_{\vn,\vn',S,I}\rho_{S,I}(\vn')f^\pi_S(\vn'-\vn)\pi_I(\vn,n_t),
\eea 
We may introduce 
\bea 
\phi_{SI}(\vn)=\sum_{\vn'} f^\pi_S(\vn'-\vn)\pi_I(\vn,n_t) 
\eea 
Then
\bea 
M|f\ra &\to& -V_{\pi}^{(n_t)}|f\ra = -\frac{g_A}{2f_\pi}\sum_{\vn,S,I}\rho_{S,I}(\vn')\phi_{SI}(\vn)|f\ra 
  \no
  & = &  -\frac{g_A}{2f_\pi}\sum_{\vn,S,I} a^\dagger_{s1,i1}(\vn)[\tau_I]_{i1,i2}[\sigma_S]_{s1,s2}
   a_{s2,i2}(\vn) \phi_{SI}(\vn) |f\ra 
\eea 



\section{Transfer matrix projection method} 
To make the M.T. calculation to get lowest energy states efficiently,
transfer matrix projection method can be used. 
This method use 
simple approximate Hamiltonian to project out higher energy states
from initial Slater determinant free particle standing waves, $|\Psi^{free}_{Z,N}\ra$ 
and obtain trial wave function closer to ground state
for faster convergence. 

First get trial wave function by using
\bea 
|\Psi(t')\ra =\exp[-H_{SU(4)} t']|\Psi^{free}_{Z,N}\ra, 
\eea 
with this trial wave function, define amplitude,
\bea 
Z(t)=\la \Psi(t')|\exp[-H_{LO} t]|\Psi(t')\ra, 
\eea 
and transient energy
\bea 
E(t)=-\frac{\del}{\del t}[\ln Z(t)].
\eea 
Then, at large t limits, we get ground state energy
\bea 
\lim_{t\to \infty} E(t)= E_0.
\eea 
Note that trial wave function $|\Psi(t')\ra $ must be non-orthogonal to ground state $|\Psi_0\ra$.

For some normal ordered operator,
\bea 
Z_O(t)=\la \Psi(t')|\exp[-H_{LO} t/2] O \exp[-H_{LO} t/2]|\Psi(t')\ra, 
\eea 
\bea 
\lim_{t\to \infty}\frac{Z_O(t)}{Z(t)}=\la \Psi_0|O|\Psi_0\ra. 
\eea 


\section{Hybrid Monte Carlo}
The hybrid Monte Carlo method is to make the importance sampling to be 
more efficient by using the Hamiltonian dynamics for Markov Chain updates.
In other words, by constructing Hamiltonian $H(x,p)$ such that 
the Hamiltonian dynamics actually leads to the distribution of x 
satisfies some target distribution $p(x)$. 

For any energy function $E(\theta)$ with variable $\theta$,
we can define cannonical distribution, $p(\theta)$ as
\bea
p(\theta)=\frac{1}{Z} e^{-E(\theta)}.
\eea
By setting energy function as Hamiltonian, canonical distribution
becomes
\bea 
p(\vx,\vp)\propto e^{-H(\vx,\vp)}=e^{-[U(\vx)-K(\vp)]}
  = e^{-U(\vx)}e^{-K(\vp)} \propto p(\vx) p(\vp)
\eea 
Thus, the distribution disjoints and it means the cannonical 
distribution $p(\vx)$ of $\vx$
is independent of conjugate momentum $\vp$. 
Therefore, we can use Hamiltonian dynamics to sample 
from the joint canonical distribution over $\vp$ and $\vx$ 
and simply ignore the momentum contributions. Now, 
we can choose $K(\vp)=\frac{1}{2}\vp^{T}\vp$ and 
$U(\vx)=-\log p(\vx)$. If we can calculate $\frac{d U(\vx)}{d\vx}$, 
the molecular dynamic can be used. 

Once the fermion integral is done, we have auxiliary field integration,
\bea 
Z_{\Psi}=\prod_{\vn,n_t}[\int d_A s(\vn,n_t)]\det X(s)
        =\prod_{\vn,n_t}[\int d_A s(\vn,n_t)]e^{\log[\det X(s)]}    , 
\eea 
For a choice of auxiliary field,
\bea 
\int d_A s(\vn,n_t)
&=&\frac{1}{\sqrt{2\pi}}\int_{-\infty}^{\infty} ds(\vn,n_t) e^{-\frac{1}{2}s^2(\vn,n_t)},\no 
A[s(\vn,n_t)]&=&\sqrt{-C\alpha_t} s(\vn,n_t),
\eea 
we have weights for $s(\vn,n_t)$,
\bea 
Z_{\Psi}\propto\int {\cal D} s\ e^{-\sum_{\vn}\frac{1}{2}s^2(\vn,n_t)}e^{\log[\det X(s)]}  
\eea
Thus we might do importance sampling with weights, 
$e^{-\sum_{\vn}\frac{1}{2}s^2(\vn,n_t)}e^{\log[\det X(s)]}$. But, there will
be many rejected moves for sampling $s$. Instead, we may introduce adjoint 
field variable $p(\vn,n_t)$ which is normal distributed and change
\bea
Z_\Psi &\propto&  \int{\cal D}p \int {\cal D} s
    \exp\left[
    -\left(\sum_\vn \frac{1}{2}p^2(\vn,n_t)+\frac{1}{2}s^2(\vn,n_t)\right)
    +\log[\det X(s)]
    \right] \no 
    &=&\int{\cal D}p \int {\cal D} s
    \exp\left[-H(s,p)
    \right], \no 
H(s,p)&=&\sum_{\vn}\frac{1}{2} p^2(\vn,n_t)+V(s),\quad
V(s)=\left( \sum_{\vn}\frac{1}{2} s^2(\vn,n_t) \right)-\log[\det X(s)] \no
\eea
Then, if we move $s$ and $p$ according to Molecular dynamic way,
all such $s$ will have the same $H$ values and can be acceptable...(?).
Because MD only choose sampling with same $H(s,p)$, we have to
disrupt them after some samplings so that sample with different $H$.  
 
Molecular Dynamics equation is
\footnote{Note that there is no time advance for $s$ and $p$.
MD evolution is just for updating auxiliary field $s(\vn,n_t)$. 
And also, to advance MD, initial momentum $p$ can be
half-step advance first and half-step back at last. 

Another note: To do molecular dynamics, we use leap-frog method.
Make first move conjugate momentum in half and use the value for
the advance of postion and so on. Then move momentum  half-step back.   
}

Given an arbitrary initial configuration $s^0(\vn,n_t)$,
the conjugate momentum is chosen from a random Gaussian distribution,
\bea 
P[p_s^0(\vn,n_t)]\propto \exp(-\frac{1}{2}[p_s^0(\vn,n_t)]^2).
\eea 
Molecular dynamic calculation of equation of motion is done up to final step.
But, at first, the method begins with a "half-step" forward in the conjugate momentum,
\bea 
\tilde{p}^0_{s}(\vn,n_t)=p^0_s(\vn,n_t)-\frac{\epsilon_{step}}{2}\left[ \frac{\del V(s)}{\del s(\vn,n_t)}\right]_{s=s^0}. 
\eea 
and then next steps are 
\bea 
s^{i+1}(\vn,n_t)&=&s^i(\vn,n_t)+\epsilon_{step} \tilde{p}^i(\vn,n_t),\no 
\tilde{p}^{i+1}(\vn,n_t)&=&\tilde{p}^{i}(\vn,n_t)
               -\epsilon_{step}\left[\frac{\del V(s)}{\del s(\vn,n_t)} \right]_{s=s^{i+1}} .
\eea  
until the last half-step backward in $\tilde{p}_s$
\bea 
p_s^{N_{step}}(\vn,n_t)=\tilde{p}^{N_{step}}_{s}(\vn,n_t)+\frac{\epsilon_{step}}{2}\left[ \frac{\del V(s)}{\del s(\vn,n_t)}\right]_{\color{red}s=s^0(?)}
\eea 
Note that, $\tilde{p}^{i}$ and $s^{i}$ are only intermediate steps
and we need only to compare $s^{0}(\vn,n_t),p_s^0$ and $s^{N_{step}}(\vn,n_t), p^{N_{step}}_s$
and decide whether to accept it or not. The new configuration 
is accepted if 
\bea 
r < \exp[-H(s^{N_step},p_s^{N_step})+H(s^0,p^0_s) ].
\eea  
with random number $r\in [0,1)$.

In case of several auxiliary configuration,
we have
\bea 
\exp(-V(s))\to |Z(L_t)|\exp[-S_{ss}(s)-S_{s_I s_I}(s_I)-S_{\pi_I\pi_I}(\pi_I)],
\eea 
where $|Z(L_t)|\propto \ln |\det X(s,s_I,\pi_I)|$.


The derivative of $V(s)$ can be computed by ,
\bea 
\frac{\del V(s)}{\del s(\vn,n_t)}
&=& \frac{\del S_{ss}(s)}{\del s(\vn,n_t)}-\frac{\del {\rm Re}\ln\det X}{\del s(\vn,n_t)}\no 
&=& \frac{\del S_{ss}(s)}{\del s(\vn,n_t)}
  -{\rm Re}\left[\frac{1}{\det X}\sum_{k,l} \frac{\del \det X}{\del X_{kl}}\frac{\del X_{kl}}{\del s(\vn,n_t)}
  	  \right] \no 
&=& \frac{\del S_{ss}(s)}{\del s(\vn,n_t)}
  -  {\rm Re}\left[\sum_{k,l} X^{-1}_{lk}(s)\frac{\del X_{kl}(s)}{\del s(\vn,n_t)}
  \right]  
\eea 
and (the derivation of this is shown in next subsection)
\bea 
\frac{\del X_{j' j}(s)}{\del s(\vn,n_t)}
&=&\frac{\del A(s)}{\del s(\vn,n_t)} \la vd_{j'}(n_t+1)|\vn\ra \la \vn |v_{j}(n_t)\ra.
\eea 

\subsection{ Derivative of single particle transfer matrix $X$   } 
Hybrid Monte Carlo requires the calculation of 
functional derivative of matrix $X$.
We can compute this by
\bea 
\frac{\del X_{j' j}(s)}{\del s(\vn,n_t)}
&=&\frac{\del}{\del s(\vn,n_t)} \la \phi_{j'}|M(s,L_t-1)\dots M(s,0)|\phi_{j}\ra \no 
&=&\underbrace{ \la \phi_{j'}|M(s,L_t-1)\dots M(s,n_t+1)}_{=\la vd_{j'}(n_t+1)|} 
              \frac{\del M(s,n_t)}{\del s(\vn,n_t)}
              \underbrace{ M(s,n_t-1)\dots M(s,0)|\phi_{j}\ra}_{= |v_{j}(n_t)\ra}.           
\eea 

and using the functional derivative form,\footnote{
Not quite sure about this
\bea 
\frac{\del M(s,n_t)}{\del s(\vn,n_t)} 
&=& \frac{\del A(s)}{\del s(\vn,n_t)}\rho(\vn) M(s,n_t)\ ?,\no 
&  &\mbox{ or } \frac{\del A(s)}{\del s(\vn,n_t)} M(s,n_t)  \rho(\vn)\ ?,\no 
&  &\mbox{ or } \frac{\del A(s)}{\del s(\vn,n_t)}\rho(\vn)\ ?
\eea 
} 
\bea 
\frac{\del}{\del s(\vn,n_t)} 
:\exp\left[-\alpha_t \hat{H}_{free}+\sum_{\vn'} A[s(\vn',n_t)] \rho(\vn') \right]:
= \frac{\del A(s)}{\del s(\vn,n_t)}\rho(\vn)
\eea 
, thus,
\bea 
\frac{\del X_{j' j}(s)}{\del s(\vn,n_t)}
&=&\frac{\del A(s)}{\del s(\vn,n_t)} \la vd_{j'}(n_t+1)|\vn\ra \la \vn |v_{j}(n_t)\ra \no       
\eea 

\section{Example: HMC update for non-local chiral action}

Let us look at more detail on the HMC update with an example.
In HMC, we can write 
\bea 
{\cal Z}(L_t)&=&\int {\cal D} p{\cal D}p^I_\pi {\cal D}s {\cal D}\pi_I 
         e^{-\sum_{\vn} \left[ \frac{1}{2} p(\vn)^2  + \frac{1}{2} s(\vn)^2
                                    +\frac{1}{2} p_\pi^I(\vn)^2 +\frac{1}{2} \phi_I^2(\vn)\right] 
         } \no & &\times \la \Psi| M^{L_t}(s,\pi_I)|\Psi\ra  \no 
     &=& \int {\cal D} p{\cal D}p^I_\pi {\cal D}s {\cal D}\pi_I 
     \exp\left( -\sum_{\vn} \left[ \frac{1}{2} p(\vn)^2  + 
     	+\frac{1}{2} p_\pi^I(\vn)^2  +V(s,\pi_I)\right] \right) 
\eea 
with 
\bea 
V(s,\pi_I)=\sum_\vn\left[ \frac{1}{2} s(\vn)^2+\frac{1}{2} \phi_I^2(\vn)\right] -\ln \det X(s,\pi_I,L_t)
\eea 

In fact, the code use two different $M$ matrix for probability calculation. Thus, actual HMC potential is 
\bea 
V(s,\pi_I)=\sum_\vn\left[ \frac{1}{2} s(\vn)^2+\frac{1}{2} \phi_I^2(\vn)\right] 
    -a_0\ln \det X_0(s,\pi_I,L_t)-a_1\ln \det X_1(s,\pi_I,L_t)
\eea 


The fermion transfer matrix gives HMC potential,
\bea 
\la \Psi|M^{Lt}|\Psi\ra = \det X(s,\pi_I,L_t)=\exp(-(-\ln \det X(s,\pi_I,L_t))).
\eea 
\bea 
M^{(n_t)}=:\exp\left(-H_{free}\alpha_t-V_s^{(n_t)}\sqrt{\alpha_t}
-V_\pi^{(n_t)}\alpha_t  \right):
\eea 
\bea 
V_{s}^{(n_t)}={\color{red}-}\sqrt{-c_0}\sum_{\vn\vn'}\rho_{NL}(\vn)f_{s_L}(\vn-\vn')s(\vn',n_t).
\eea 
\bea 
V_{\pi}^{(n_t)}&=&{\color{red}-}\frac{g_A}{2f_\pi}\sum_{\vn,\vn',S,I}\rho_{S,I}(\vn')f^\pi_S(\vn'-\vn)\pi_I(\vn,n_t)
\no 
&=&{\color{red}+}\frac{g_A}{2f_\pi}\sum_{\vn,\vn',S,I} \pi_I(\vn',n_t)f^\pi_S(\vn'-\vn)\rho_{S,I}(\vn)
\eea 

where, we can think 
$\sum_{\vn}f_S^\pi(\vn'-\vn)\pi_I(\vn)=\nabla_S \pi_I (\vn')$.
(Note that ,In the code, 
the second form $\sum_{\vn'}\pi_I(\vn') f_S^\pi(\vn'-\vn)=-\nabla_S \pi_I (\vn')$ is used
Thus, be careful for the sign. Also, it is important to note that the derivative is done with auxiliary pion
not true pion. )

Then,
\bea 
\frac{\del V(s,\pi_I)}{\del s(\vn,n_t)}
&=& s(\vn,n_t)- a_0 {\rm Re}\left[ \sum_{kl} X^{-1}_{lk} 
            \la zd_k(n_t+1)|\frac{(-V_{s}^{(n_t)}\sqrt{\alpha_t})}{\del s(\vn,n_t)}| zv_l (n_t)\ra  
                 \right] 
            - a_1 \left[ (X_0\to X_1))     \right] ,\no 
\frac{\del V(s,\pi_I)}{\del \phi_I(\vn,n_t)}
&=& \phi_I(\vn,n_t)- a_0 \sum_{S}\sum_{\vn'} 
 \frac{\del \nabla_S\pi_I(\vn',n_t)}{\del \phi_I(\vn,n_t)}
{\rm Re}\left[ \sum_{kl} X^{-1}_{lk} 
 \la zd_k(n_t+1)|\frac{\del (-V_{\pi}^{(n_t)}\alpha_t) }{\del \nabla_S\pi_I(\vn',n_t)}
| zv_l (n_t)\ra  
\right]  \no & &
- a_1 \left[ (X_0\to X_1))     \right]            
\eea 

And
\bea 
\la zd_k(n_t+1)|\frac{\del (-V_{s}^{(n_t)}\sqrt{\alpha_t})}{\del s(\vn,n_t)}| zv_l (n_t)\ra 
&=&
 \sqrt{-c_0\alpha_t}\la zd_k(n_t+1)| \sum_{\vn'} \hat{\rho}_{NL}(\vn') f_{s_L}(\vn'-\vn)| zv_l (n_t)\ra \no 
&=& \sqrt{-c_0\alpha_t} \sum_{\vn'} \la zd_k(n_t+1)|\vn'\ra_{smear} f_{s_L}(\vn'-\vn)\la \vn'| zv_l (n_t)\ra_{smear}
 , \no                                  
\la zd_k(n_t+1)|\frac{\del (-V_{\pi}^{(n_t)}\alpha_t)}{\del \nabla_S\pi_I(\vn,n_t)}
| zv_l (n_t)\ra
&=&  -\frac{g_A}{2f_\pi}  
   \la zd_k(n_t+1)|\hat{\rho}_{SI}(\vn)| zv_l (n_t)\ra \no 
   &=& 
    = -\frac{g_A}{2f_\pi}  \Big(\la zd_k(n_t+1)|\vn\ra \sigma_S \tau_I \la \vn | zv_l (n_t)\ra \Big)
\eea 

We may define,
\bea 
dV(\vn,n_t,S,I)=
{\rm Re}\left[ \sum_{kl} X^{-1}_{lk} 
   \sum_{s_1,s_2,i_1,i_2}
 \la zd_k(n_t+1)|\vn,s_1 i_1\ra (\sigma_S)_{s_1 s_2}(\tau_I)_{i_1 i_2} \la \vn, s_2 i_2| zv_l (n_t)\ra  
\right] 
\eea 
Then,
\bea 
\frac{\del V(s,\pi_I)}{\del \phi_I(\vn,n_t)}&=&\phi_I(\vn,n_t)
 -\sum_{S}\sum_{\vn'}\frac{\del \nabla_S\pi_I(\vn',n_t)}{\del \phi_I(\vn,n_t)} (-\frac{g_A\alpha_t}{2f_\pi})dV(\vn,n_t,S,I) \no 
 &=& \phi_I(\vn,n_t)
 +\frac{g_A\alpha_t}{2f_\pi} \sum_{S} 
  \frac{1}{L^3}\sum_\vq e^{i\vq\cdot\vn}(-i\vq)_S \frac{ \widehat{dV}(\vq,n_t,S,I)}{\sqrt{\vq^2+m_\pi^2}}. 
\eea 
The second term can be computed either as
\bea 
& &\frac{g_A\alpha_t}{2f_\pi} \sum_{S} 
\frac{1}{L^3}\sum_\vq e^{i\vq\cdot\vn}(-i\vq)_S \frac{ \widehat{dV}(\vq,n_t,S,I)}{\sqrt{\vq^2+m_\pi^2}} \no 
 &=& \frac{g_A\alpha_t}{2f_\pi} \sum_{S}\sum_{\vn'} \Delta_S(\vn-\vn') pdV(\vn'),
 \quad \widehat{pdV}(\vq)=\frac{\widehat{dV}(\vq)}{\sqrt{\vq^2+m_\pi^2}},
\eea 
or as 
\bea 
\frac{\del V(s,\pi_I)}{\del \phi_I(\vn,n_t)}&=&\phi_I(\vn,n_t)+\frac{g_A\alpha_t}{2f_\pi}\sum_{S}W(\vn,n_t,S,I),
\quad \widehat{dW}(\vq,n_t,S,I) = (-i\vq)_S \frac{ \widehat{dV}(\vq,n_t,S,I)}{\sqrt{\vq^2+m_\pi^2}} 
\nonumber 
\eea 


\section{Example: Obervable in Unitary limit calculation }
Until now the calculation concerned is
\bea 
Z(L_t) =\la \Psi| M^{L_t}|\Psi\ra =\int d_A s\ \la \Psi| M^{L_t}(s)|\Psi\ra
\eea 
Thus, if the initial many particle wave is a linear combination of 
exact many-body eigen states,
\bea 
|\Psi\ra&=&\sum_{k} c_k |\Psi_k\ra, \quad H|\Psi_k\ra=E_k\Psi_k\ra,\no   
Z(L_t)&=& \sum_{k} |c_k|^2 e^{-E_k L_t}
\eea 
Thus, we can compute ground state energy by 
\bea 
E(t)&=&\frac{1}{\alpha_t}\log\frac{Z(t-\alpha_t)}{Z(t)},\no 
\lim_{t\to \infty} E(t) &= & E_{0}+\dots   
\eea 
Because the actual MC sampling should be done with positive definite weights,
we separate 
\bea 
\det X(s,t)=|\det X(s,t)|e^{i\theta(s,t)}=e^{\log|\det X(s,t)|}e^{i\theta(s,t)}
\eea 
and include $e^{\log|\det X(s,t)|}$ part into weights. 
Thus, actual MC average is done by calculating
\bea 
Z(t)=\la e^{i\theta(s,t)} \ra_{t}
\eea 
where $\la \dots\ra_t$ means MC average using  $e^{\log|\det X(s,t)|}$
as part of weights.
Then, in the same way,
\bea 
Z(t-\alpha_t)
=\left\la \frac{e^{\log|\det X(s,t-1)|}
              e^{i\theta(s,t-1)}}{e^{\log|\det X(s,t)|}}\right \ra_t
\eea 

On the other hand, we may define probability distribution as
\bea 
P[s]\propto e^{-S_s} [a_0|\det X(s,t)| +a_1 |\det X(s,t-\alpha_t)|]
\eea 
with some constant $a_0$ and $a_1$. Let us introduce simplified notation,
$V_0=|\det X(s,t)|$ and $V_1=|\det X(s,t-\alpha_t)|$. Then, 
we may write the probability action as
\bea
P[s]\propto e^{-act},\quad 
act= S_s -\log (a_0 V_0+a_1 V_1)
\eea 
Note that because of the change in probability choice , one have to compensate the change in the 
$Z(L_t)$ calculation. However, in other observables which were inserted in the calculation
\bea 
\la \Psi| M^{L_t-n_t}O(n_t)M^{n_t}|\Psi\ra = \int d s e^{-S_s}\la \Psi| M^{L_t}(s) O(t)|\Psi\ra 
 =  \int d s e^{-S_s+V(s)} e^{-V(s)}\det X(s,O) \to \frac{1}{N}\sum'_{i} e^{-V(s_i)}\det X(s_i,O)
\eea 
The new correlator including observable is calculated and used to 
obtain the expectation value of the observable. Note that if the observable is inserted at $n_t$,
\bea 
\la \Psi| M^{L_t}(s) O(t)|\Psi\ra=\la vd(n_t,L_t-1)| M(n_t) O(n_t)|v(n_t,L_t-1)\ra 
\eea 
Thus, one have to use zvecs and zdualvecs for $L_t-1$. 


\section{Example: density operator evaluation}

Let us consider a expectation value of $:\rho^3(\vn)/3!:$ summed over $\vn$(actually the operator should be a summation over spin and iso-spin). 
For a given $\vn$, one have to evaluate
\bea 
\la \Psi(s) | \frac{1}{3!}:\left(\sum_{s,i} \rho_{s,i}  (\vn,n_t) \right)^3 :| \Psi(s)  \ra 
\eea 
and then summed over $\vn$ and averaged over $s$. 
To evaluate this matrix element, let us consider,
\bea 
& &\la \Psi_0| M\cdots :\exp\left(\sum_{\vn'}\sum_{s,i}\epsilon(\vn') \rho_{s,i}(\vn',n_t)\right) : \cdots M|\Psi_0\ra   
 = \det X([s], [\epsilon] ),\no 
& &X_{ij}([s], [\epsilon] )=\la \phi_i| M\cdots :\exp\left(\sum_{\vn'}\sum_{s,i}\epsilon(\vn') \rho_{s,i}(\vn',n_t)\right) : \cdots M|\phi_j \ra 
\eea 
Then for given $\epsilon(\vn)$ function, one can compute $X_{ij}$ and thus $\det X$. 
Thus, the original expectation value may be considered as
\bea 
\la \Psi(s) | \frac{1}{3!}:\left(\sum_{s,i} \rho_{s,i}  (\vn,n_t) \right)^3 :| \Psi(s)  \ra 
 = \frac{\del^3}{\del \epsilon(\vn)^3} \det X([s], [\epsilon]=0) 
\eea 
In fact, since we only need derivative of at a specific $\vn$, 
we may set $\epsilon(\vn')=\epsilon \delta_{\vn',\vn}$ and evaluate, $ X_{ij}([s],\epsilon,\vn )$,
\footnote{Need verification whether it should be $n_t$ or $n_t+1$ for zdvecs.}
\bea 
 X_{ij}([s],\epsilon,\vn )= \epsilon \sum_{s,i} \la zdvecs(n_t)|\vn ,s,i\ra   \la \vn,s,i |zvecs(n_t)\ra 
\eea 
Then, one can obtain the derivative as numerical differences of $\det X([s], \epsilon,\vn)$
at various $\epsilon$ values, 
\bea 
\frac{\del^3}{\del \epsilon(\vn)^3} \det X([s], [\epsilon]=0) 
=\frac{\del^3}{\del \epsilon^3}|_{\epsilon=0} \det X([s], \epsilon,\vn) 
\eea 

In a similar way, one can compute the expectation value of 
\bea 
& &\la \Psi(s,L_t/2)|: \rho(\vn_1)\rho(\vn_2): |\Psi(s,L_t/2)\ra 
=\frac{\del^2}{\del \epsilon(\vn_1)\del \epsilon(\vn_2)}|_{\epsilon=0} \det X([s],\epsilon(\vn_1),\epsilon(\vn_2)) ,\no 
& &X_{ij} ([s],\epsilon(\vn_1),\epsilon(\vn_2)) =\la \phi_i (L_t/2)|(1+\epsilon(\vn_1)\rho(\vn_1)+\epsilon(\vn_2)\rho(\vn_2))|\phi_j(L_t/2)\ra. 
\eea 


%====================================================================================
\chapter{scattering in lattice}
To fix LECs, we need to compute scattering phase shifts.
Also we may want to obtain the scattering information 

\section{L\"{u}scher's method} 
{\color{red} This section requires more clear explanation.}

 
L\"{u}scher's formula relates the two-particle energy levels in a periodic cube of length 
L to the S-wave phase shift,
\bea 
p\cot\delta_0(p)=\frac{1}{\pi L} S(\eta),\quad \eta=(\frac{Lp}{2\pi})^2, 
\eea  
where $L$ is the lattice size, $p$ is related with energy level
 $E=\frac{p^2}{m}=\frac{\eta}{m}(\frac{2\pi}{L})^2$ .
 $S(\eta)$ is the three dimensional zeta function, 
 \bea 
 S(\eta)=\lim_{\Lambda\to\infty}\left[ 
    \left( \sum_{\vn}\frac{\theta(\Lambda^2-\vn^2)}{\vn^2-\eta} \right) 
   -4\pi\Lambda \right]  
 \eea  
If we draw graph $S(\eta)$ with some large value of $\Lambda$, $S(\eta)=0$ have 
several roots and closest to zero root is $\eta=-0.095901$. 
This corresponds to infinite scattering length, or unitary limits. 
($\eta<0$ means it is a bound state??)

In lattice simulation, one computes the energy level $E$ and equivalently $p$ or $\eta$ in finite volume. 
(However, actually it is not trivial to obtain $p$ from finite volume energy $E(L)$.)
Thus, one can compute $S(\eta)$ and $p$. From this information , one can obtain 
$\cot\delta_0(p)=\frac{1}{p} \frac{S(\eta)}{\pi L}$. In other words, phase shift for energy $E$
can be obtained. If $L$ is varied, one can obtain the phase shifts at different energies. 
In this way, one can computes the scattering phase shifts from finite volume calculation. 

If we only consider low energy scattering, we may expand $S(\eta)$ for $|\eta|<1$,
\bea 
S(\eta)=-\frac{1}{\eta}+\lim_{\Lambda\to \infty}\left[ 
    \left( \sum_{\vn\neq 0}\frac{\theta(\Lambda^2-\vn^2)}{\vn^2-\eta} \right) 
   -4\pi\Lambda \right]
   =-\frac{1}{\eta}+S_0+S_1\eta+S_2 \eta^2+S_3 \eta^3+\dots 
\eea 
where,
\bea
S_0&=&\left[ \left( \sum_{\vn\neq 0}\frac{\theta(\Lambda^2-\vn^2)}{\vn^2} \right) 
   -4\pi\Lambda \right],\no 
S_j&=&\sum_{\vn\neq 0}\frac{1}{(\vn^2)^{j+1}},\quad j\geq 1.
\eea 
where ${\bm n}=(n_1,n_2,n_3)$ are integers.\footnote{
Including negative integer, 0, positive integer.
By setting large $\Lambda$ values, we can calculate coefficients 
regardless of lattice size,
\bea 
& S_0=-8.913631, & S_1=16.532288, \no 
& S_2=8.401924,  & S_3=6.945808, \no 
& S_4=6.426119,  & S_5=6.202149, \no 
& S_6 = 6.098184, & S_7=6.048263.
\eea 
 } In other words, regardless the details of the interaction or system,
the low energy limit is universal.( Be careful that one can not simply send $p\to 0$
for finite $L$ values because spectrum $p$ is related with lattice size.  )

\subsection{Fermion-Dimer S-wave phase shift }
{\bf REF: Eur. Phys. J. A(2013) 49:151}
How to obtain the phase shift $\delta_0(p)$ from finite volume energy $E(L)$ for the fermion-dimer system.

Fermion-dimer energy in the periodic box of size $Lb$ for s-wave scattering:
\bea 
E^{fd}(p,L)=E^{fd}(p,\infty)+\tau^d(\eta)\Delta E_0^d(L).
\eea 
We want to extract $p$ satisfying above equation 
from a direct finite volume energy computed in the lattice, $E^{fd}(L)$
and dimer binding energy $B(L)$.

The first term,
\bea 
E^{fd}(p,\infty)=\frac{p^2}{2m_d}+\frac{p^2}{2m}-B(\infty),
\eea 
shows the relation between momentum and the energy from infinite volume limit. 
Here dimer mass $m_d$ and $B(\infty)$ have to be extracted from the lattice result $B(L)$. 
Approximately, they can be obtained from $B(L)$ with very large $L$ values. 

The second term is a finite volume corrections to the fermion-dimer energy
due to the dimer wave function wrapping around the periodic boundary,
where $\Delta E^d_{\vec 0}(L)=B(\infty)-B(L)$ is the finite volume energy shift for the
bound dimer state in the two-body center of mass frame. 
 (Thus, $\Delta E^d_{\vec 0}(L)$ can be obtained easily from $B(L)$.) 
 \bea 
 \tau^d(\eta) = \frac{1}{{\cal N}} \sum_{\vk} \frac{\tau(\vk,1/2)}{(\vk^2-\eta^2)^2},
 \eea 
 where
 \bea 
 {\cal N}=\sum_\vk \frac{1}{(\vk^2-\eta^2)^2},\quad \tau(\vk,1/2)=\frac{1}{3}\sum_{i=1}^3 \cos(k_i L/2)
 \eea 
 $\vk$ is all possible center of mass momentum $\vk$ of dimer(?
 I am not sure what is the actual summation of $\vk$. ).
 (Because of $\eta$, this is a function of $p$. 
 Thus the equation for $E^{fd}(p,L)$ is a non-linear function of $p$
 and can be solved by iteration starting from initial guess $\tau^d(\eta)=1$.
 ) 



\section{Spherical Wall method} 
{\color{red} This section requires more clear explanation.}

\section{Adiabatic projection method}
The Euclidean time evolution is called 'projection method'
because it projects the initial state into a ground state or
excitation spectrum of the system by Euclidean time evolution. 

Adiabatic projection method exploits that there are two time scales
in the projection. The first one is a faster time scale to form a 
alpha clusters. The second one is a slow time scale to form a superpositions
of clusters. 
Thus, at a {\color{red} properly chosen time }\footnote{
But how to get the proper time scale for given system?} , 
One can obtain the continuum state information. 

For an example, let us consider 3-nucleons system which forms a 
nucleon-dimer two-body scattering state. 

Choose a C.M. frame and set the origin at the one nucleon.
\footnote{C.M. frame means the total momentum of the system is zero. 
Suppose we choose a spin up proton is at the origin. 
However, since there is a wave function for a proton spin up,
the origin corresponds to the average central position of a 
spin up proton and there will be a distribution of probability to find 
spin up proton around the origin.
}  We may prepare several initial states corresponding to a nucleon-dimer
with separation ${\vec R}$, $|{\vec R}\ra$ and get a dressed cluster states
\bea 
|{\vec R}\ra &=& a^\dagger_{\uparrow}({\vec R})
                 a^\dagger_{\uparrow}({\vec 0})a^\dagger_{\downarrow}({\vec 0})|0\ra,\no 
|{\vec R}\ra_\tau &=& \exp(-H\tau)|{\vec R}\ra. 
\eea 
 
In fact, since we need to have total momentum zero state of clusters. 
One can make such state as factorized in long separation , 
\bea 
|{\vec R}\ra =\sum_{\vr} |\vr+{\vec R}\ra\otimes |\vr\ra 
\eea  
where $\vr$ represent a position of one cluster. The sum of all possible $\vr$
makes the state translation invariant and thus provide total momentum zero state.  
 
For the case of L number of lattice points in one direction,
we may have $L^3-1$ number of possible ${\vec R}$ values. 
In the limit of large $\tau$, 
dressed cluster states $|{\vec R}\ra_\tau$ will span the 
low energy spectrum of the original Hamiltonian. 

The dressed cluster states will not be orthogonal in general. Thus, we will have 
norm matrix 
\bea
[N_\tau]_{ {\vec R}, {\vec R}'} ={}_\tau\la {\vec R}|{\vec R}'\ra_\tau 
\eea 
We may define inverse of norm matrix and 
\bea 
\delta_{{\vec R},{\vec R}'} &=& \sum_{{\vec R}''}  [N_\tau^{-1}]_{ {\vec R}, {\vec R}''}[N_\tau]_{ {\vec R}'', {\vec R}'}
 =\sum_{\vec R''}   [N_\tau^{-1}]_{ {\vec R}, {\vec R}''}
   {}_\tau\la {\vec R}''|{\vec R}'\ra_\tau \no 
  &=& {}_\tau( {\vec R}| {\vec R}'\ra_\tau.
\eea 
where, dual vector is defined as
\bea 
{}_\tau( {\vec R}|v\ra  =\sum_{\vec R''}   [N_\tau^{-1}]_{ {\vec R}, {\vec R}''}
{}_\tau\la {\vec R}''| v\ra. 
\eea 
The dual vector will annihilate any vector which is orthogonal to all dressed cluster states,
\bea 
{}_\tau\la {\vec R}|v\ra=0 \mbox{ for all }{\vec R} \quad \Rightarrow \quad 
{}_\tau( {\vec R}|v\ra=0 \mbox{ for all }{\vec R} .
\eea 
Also, as shown before 
\bea 
{}_\tau ({\vec R} | {\vec R}\ra_\tau=\delta_{{\vec R},{\vec R}'}.
\eea 

By computing the matrix elements of Hamiltonian, we can write the 
adiabatic Hamiltonian as
\bea 
[H^a_\tau]_{{\vec R},{\vec R}'}&=&{}_\tau({\vec R}| H|{\vec R}'\ra_\tau,\no 
        &=& \sum_{\vec R''} [N_\tau^{-1}]_{{\vec R},{\vec R}''} 
          {}_\tau\la {\vec R}''| H |{\vec R}'\ra_\tau . 
\eea 
With similarity transform, One can obtain adiabatice Hamilronian 
which is Hermitian,
\bea 
[H^{a'}_\tau]_{{\vec R},{\vec R}'}
=\sum_{ {\vec R}'',{\vec R}'''} \left[ N_\tau^{-1/2}\right]_{{\vec R},{\vec R}'' }
        {}_\tau\la {\vec R}''|H|{\vec R}'''\ra_\tau 
       \left[ N_\tau^{-1/2}\right]_{{\vec R}''',{\vec R}'}                            
\eea 
{\color{red} what exactly is the definition of $\left[ N_\tau^{-1/2}\right]_{{\vec R}''',{\vec R}'}$ ?  Defined as  $N_\tau^{-1}=N_\tau^{-1/2} \cdot N_\tau^{-1/2}$ ? } 

In other words, (1) Prepare many $|{\vec R}\ra$ states (2) compute dressed states $|{\vec R}\ra_\tau$
(3) compute the norm matrix $[N_\tau]_{{\vec R},{\vec R}'}$ and also compute expectation value 
${}_\tau\la {\vec R}''|H|{\vec R}'''\ra_\tau$ (4) compute $\left[ N_\tau^{-1/2}\right]_{{\vec R}''',{\vec R}'}$
(5) One can obtain $[H_\tau^{a'}]_{{\vec R},{\vec R}'}$ and also $[H_\tau^{a}]_{{\vec R},{\vec R}'}$. 
(6) One can compute the phase shifts between clusters by solving $[H_\tau^{a}]_{{\vec R},{\vec R}'}$ eigen-value 
equation. 

In the code, the computation of matrix element can be done
\bea 
{}_\tau\la {\vec R}| H |{\vec R}'\ra_\tau &=& \left(\sum_{\vr} \la \vr+{\vec R}|\otimes\la \vr|\right)
           e^{-H\tau} H e^{-H\tau} \left(\sum_{\vr'} |\vr'+{\vec R}'\ra\otimes |\vr'\ra \right) \no 
        &\simeq&  \sum_{\vr} \left(\la \vr+{\vec R}|\otimes\la \vr|\right)  
              e^{-H\tau} H e^{-H\tau} \left(|{\vec R}'\ra\otimes |{\vec 0}\ra \right)
\eea 
where the position of one cluster is fixed at the origin in the initial state
and the sum over possible shifts $\vr$ is done as a part of MC simulation.
Since the sum over $\vr$ makes the total momentum of final state as zero,
only the part of the initial state which have zero total momentum can contribute the matrix element. 
Also, the sum over $\vr$ can use some probability distribution so that 
\bea 
\sum_{\vr} f(\vr)=\sum_\vr P(\vr) \frac{f(\vr)}{P(\vr)}\simeq \frac{1}{N} \sum'_i \frac{f_i}{P_i},
\eea 
where $\sum'_i$ is a sum over sampling via probability distribution $P$. 


\chapter{Sign Problem}
When we compute the Monte Carlo calculation, we needs some weight function 
which have to be positive definite because it should be interpreted as probability.
However, in some cases, the weight becomes complex or negative
which makes it difficult to get reliable numerical results.
In example case with complex weight $\det X(s,t)$,
we have to compute expectation value of $Z(t)=\la e^{i\theta(s,t)} \ra_{t}$
in limit of large $t$. 
However, this expectation value rapidly decreases to zero
because of the sign ocillation for large t, there is a cancellation 
between positive weight and negative weights. Thus, it is very difficult to compute
observable reliably because we needs
\bea
\la O\ra =\frac{\la O e^{i\theta}\ra_{pq}}{\la e^{i\theta}\ra_{pq}},\quad
\la O\ra_{pq}\equiv \frac{ \int d{s} |\det X(s)| O}{ \int d{s} |\det X(s)|}
\eea 

And, if the lattice volume is $\Omega$ and number of fermions $\Delta f$, 
\bea
\la e^{i\theta}\ra_{pq}=\frac{Z_{full}}{Z_{pq}}=e^{-\Omega\Delta f}\to 0.
\eea 

In case of lattice QCD, the complex action appears with chemical potential $\mu$, 
\bea
[\det X(\mu)]^*=\det X(-\mu^*) 
\eea

\section{Origin of sign problem?}
Let us consider a general matrix $M$ which have eigen states $\phi$ such that 
$M\phi=\lambda\phi$. Here, we want to prove that if there exists some 
anti-symmetric unitary matrix $U$, $U^T=-U$, such that $U^\dagger M U = M^*$,
then $\det M$ is semi-positive definite.

Let us define a vector $\tilde{\phi}=U\phi^*$. Then, 
\bea 
M\tilde{\phi}&=&M(U\phi^*)=U(U^\dagger M U)\phi^*= U M^*\phi^*=U(M\phi)^*=U(\lambda\phi)^*
             =\lambda^* (U\phi^*) \no 
             &=&\lambda^*\tilde{\phi}. 
\eea 
Thus, $\tilde{\phi}$ is also eigen-vector of $M$ with eigen-value $\lambda^*$. 
On the other hand,
\bea 
\phi^\dagger\tilde{\phi}&=&\phi^\dagger U \phi^* =(\phi^\dagger U \phi^*)^T
           =\phi^\dagger U^T \phi^* \no 
           &=& -\phi^\dagger U \phi^*=0.
\eea 
\footnote{ 
In a similar way, using $U^*=-U^\dagger$,
\bea 
\tilde{\phi}^\dagger\phi&=&\phi^T U^\dagger \phi = (\phi^T U^\dagger \phi)^T
                       =\phi^T U^* \phi \no 
                       &=& -\phi^T U^\dagger \phi=0.
\eea  
}   
Thus, $\phi$ and $\tilde{\phi}$ is always orthogonal as long as such $U$ exists.
Thus, implies the eigenvalues of $M$ is always paired and
determinant would be $\det M\propto \lambda^* \lambda \propto |\lambda|^2\geq 0$.
When $\lambda$ is real, it implies there exists degenerate eigen states
and also $\det M\geq 0$.

For example, simplest case is operator $M=I$ and $M=i I$. 
Any antisymmetric unitary matrix will satisfy the condition for $M=I$
if $M=I$ is even dimensional. On the other hand, there is no such matrix
for $M=i I$ in any dimension. 

Another example is , in (iso)spin space\footnote{Because,
\bea 
\sigma_2^\dagger \sigma_{1,3} \sigma_2=\sigma_2 \sigma_{1,3} \sigma_2 =-\sigma_{1,3}\quad 
\sigma_2^\dagger\sigma_2\sigma_2=\sigma_2.                                     
\eea 
} , $i\sigma_{S=1,2,3}$ would not show any sign problem because 
\bea 
\sigma_2^\dagger(i\sigma_{1,3})\sigma_2=-i\sigma_{1,3}=[i\sigma_{1,3}]^*,\quad 
\sigma_2^\dagger (i\sigma_2) \sigma_2 = i\sigma_2 =[i\sigma_2]^* .
\eea 
On the other hand, operator $\sigma_S$ (without $i$) would show sign problem. 

In spin-isospin space, 
$\sigma_2 I_\tau $, $\sigma_2\tau_{1,3}$ or $\tau_2 I_\sigma$, $\tau_2\sigma_{1,3}$
can acts as a antisymmetric operators. However, there would be some constraints on the 
operators. 

Actual amplitude oe transfer matrix have dependence on space, spin and iso-spin.
But, we can use different gauge(unitary matrix) at different position so that
most problem would occur in spin and isospin part. In terms of spin and iso-spin,
possible operator forms are
\bea 
I, iI, \sigma_S, i\sigma_S, \tau_I, i\tau_I, \sigma_S\tau_I, i\sigma_S\tau_I
\eea 
and their combinations. 


Consequence: Let us consider the two consecutive unitary transformation, 
$V^\dagger M V=M'$ and $U^\dagger M' U= (M')^*$ and $U^T=-U$. The overall
transformation is, $(VU)^\dagger M (VU)=(V^\dagger M V)^*=V^T M^* V^*$ 
and $(VU)$ is not anti-symmetric. However, $\det M=\det M'\geq 0$. 

Because ${}^4He$, which is spin and isospin singlet, does not have preferred
spin and iso-spin direction, we can rotate them separately. 
Thus, $H=C_S i\vs_S +C_I i \tau_I$ case. If we use transformation by $\tau_2$,
$\tau_2 H\tau_2 = C_S i\vs_S +C_I(i\tau_I)^*$ and then transform by $\sigma_2$,
$(\sigma_2^\dagger\tau_2^\dagger) H (\tau_2\sigma_2)=H^*$. 
Thus, in case of isospin singlet channel, $c_S$ and $c_I$ term does not gives 
sign problem from interference. 


\subsubsection{modified action}  
Instead of using strict leading order action,
it is possible to modify the action such that (1) smearing contact interactions
and (2) additional contact terms which are effectively accounting for higher order
interactions. One reason for introducing smearing contact interaction is 
to avoid clustering instability of LO interaction. 

However, in terms of physics there are only two low energy constants $C_{1S0}, C_{3S1}$
or $C,C_{I}$
to be fixed from phase shifts at leading order, and also
additional contact interactions, $C_{S},C_{SI}$ , may leads to non-physical
S-wave symmetric spin-isospin states. 
To avoids such incidents, we will set relations
$C_{S},C_{SI}$ such that they are not independent.  
For two-body states, contact terms acts as operators,
\bea 
\frac{1}{2}\bar{N}A_iN \bar{N}B_jN|\alpha\ra_{2B}\to 
\frac{1}{2}(\hat{A}_{1}^i \hat{B}_2^{j}+\hat{A}_{2}^i \hat{B}_1^{j})|\alpha\ra_{2B}
\eea  
Then, contact interactions corresponds to following operators
\bea
\hat{O}=C\ I+ C_{S^2}\vs_1\cdot\vs_2+C_{I^2}\tau_1\cdot\tau_2+C_{S^2I^2}\vs_1\cdot\vs_2\tau_1\cdot\tau_2.
\eea 
Thus, to make symmetric spin-isospin states does not contribute,
\bea 
& &\hat{O}|S=0,I=0\ra=(C-3C_{S^2}-3C_{I^2}+9C_{S^2I^2})|S=0,I=0\ra=0,\no 
& &\hat{O}|S=1,I=1\ra=(C+C_{S^2}+C_{I^2}+C_{S^2I^2})|S=1,I=1\ra=0,\no 
\eea 
we set
\bea 
C=-3C_{S^2,I^2}=-\frac{3}{2}(C_{S^2}+C_{I^2}),
\eea 
and the smeared contact interactions are non-zero only for even parity channels
where we have anti-symmetry in spin-siospin.
(Note that if there was no $C_{SI}$, the relation
implies the equivalence between $C_{S}$ and $C_{I}$.
We will have relation between $C_{1S0}, C_{3S1}$ with $C,C_{I}$
or $C,C_{S}$.
)
{\bf Question:} But, if there is actually only two independent LECs,
                what is exactly the additional contact interactions do? 
                Does it actually reflect some part of higher order correction?

Now, in momentum space,
\bea
\rho(\vq_s)&=&\sum_{\vn_s}\rho(\vn_s) e^{i\vq_s\cdot\vn_s},\no 
\rho_I(\vq_s)&=&\sum_{\vn_s} \rho_{I}(\vn_s)e^{i\vq_s\cdot\vn_s},
\eea 
\bea 
\vq_s=\frac{2\pi}{L}\vk_s, \quad \mbox{ $\vk_s$ are integers from 0 to L-1. }
\eea 
Then leading order form, by introducing smearing in momentum space, 
\bea
& &-\frac{1}{2}C\alpha_t\sum_{\vn_s} [\rho(\vn_s)]^2
-\frac{1}{2}C_I\alpha_t\sum_{I=1,2,3}\sum_{\vn_s}[\rho_I(\vn_s)]^2 \no 
&=&\frac{1}{L^3}\sum_{\vq_s}\left[
  -\frac{1}{2}C\alpha_t \rho(\vq_s)\rho(-\vq_s)
  -\frac{1}{2}C_I \alpha_t \sum_{I=1,2,3} \rho_I(\vq_s)\rho_I(-\vq_s)
\right]\no 
&\Rightarrow& \frac{1}{L^3}\sum_{\vq_s}f(\vq_s^2)\left[
  -\frac{1}{2}C\alpha_t \rho(\vq_s)\rho(-\vq_s)
  -\frac{1}{2}C_I \alpha_t \sum_{I=1,2,3} \rho_I(\vq_s)\rho_I(-\vq_s)
\right]
\eea 
where, modification was done in the form of
\bea 
f(\vq_s^2)=f_0^{-1}\exp\left[-b\sum_{l_s=1,2,3}(1-\cos q_{l_s})  \right],
\eea 
with normalization,
\bea 
f_0=\frac{1}{L^3}\sum_{\vq_s}\exp\left[-b\sum_{l_s=1,2,3}(1-\cos q_{l_s})  \right],
\eea 
and coefficient $b$ is fixed to reproduce the effective range. This form
of correction gives at small $\vq$ limits,
\bea 
f(\vq_s^2)\simeq f_0^{-1} \exp\left( -\frac{b}{2}\vq_s^2 \right)
\eea 
Then modified action with higher derivative corrections are
\bea 
H^{(n_t)}_{LO_2}(\pi'_I)&=& H_{free}+\frac{1}{2}C\sum_{\vq} f(\vq) :\rho(\vq)\rho(-\vq):\no 
  & &+\frac{1}{2}C_{S^2}\sum_{\vq} f(\vq) :\rho_S(\vq)\rho_S(-\vq):\no 
  & &+\frac{1}{2}C_{I^2}\sum_{\vq} f(\vq) :\rho_I(\vq)\rho_I(-\vq):\no 
  & &+\frac{1}{2}C_{S^2I^2}\sum_{\vq} f(\vq) :\rho_{S,I}(\vq)\rho_{S,I}(-\vq):\no 
  & &+\frac{g_A}{2f_\pi\sqrt{q_\pi}}\sum_{\vn,S,I}\Delta_S\pi'_I(\vn,n_t)\rho_{S,I}(\vn)
\eea 
We can change the expression into position space form by F.T., 
\bea 
\frac{1}{L^3}\sum_{\vq} f(\vq^2) \rho(\vq)\rho(-\vq)
=\sum_{\vn}\sum_{\vn'} f(\vn-\vn') \rho(\vn)\rho(\vn').
\eea 
From the Gaussian integration formula,
\bea 
\exp\left( -\frac{1}{2} \sum_{\vn\vn'} f(\vn-\vn') \rho(\vn)\rho(\vn')\right) 
\propto \int {\cal D} s
       \exp\left( \frac{1}{2}\sum_{\vn,\vn'} s(\vn)f^{-1}(\vn-\vn') s(\vn')
            +\sum_{\vn} \rho(\vn) s(\vn) \right). 
\eea 

Then, auxiliary lattice action can be written as
\bea 
{\cal Z}_{LO}\propto \int D\pi'_I Ds D s_I \exp[-S_{\pi\pi}-S_{ss}]
   {\rm Tr}[M^{(L_t-1)}(\pi'_I,s,s_I)\times\dots M^{(0)}(\pi'_I,s,s_I)].
\eea 
With auxiliary action,
\bea 
S_{ss}&=&\frac{1}{2}\sum_{\vn}s^2(\vn) +\frac{1}{2}\sum_{I}\sum_{s} [s_I(\vn)]^2 \no 
&\Rightarrow& \frac{1}{2}\sum_{\vn_s,\vn'_s,n_t} s(\vn_s,n_t)f^{-1}(\vn_s-\vn'_s) s(\vn'_s,n_t)
+\frac{1}{2}\sum_{I} \sum_{\vn_s,\vn'_s,n_t} s_I(\vn_s,n_t) f^{-1}(\vn_s-\vn'_s) s_I(\vn'_s,n_t).
\eea 
and the function $f^{-1}$ is defined as
\bea 
f^{-1}(\vn_s-\vn'_s)=\frac{1}{L^3}\sum_{\vq_s}\frac{1}{f(\vq_s^2)}e^{-i\vq_s\cdot(\vn_s-\vn'_s)}
\eea 



In auxiliary formulation
\bea 
M_{LO,aux}^{(n_t)}(s,s_{S},s_{I},s_{S,I},\pi'_I )
&=& :\exp\left\{ -H_{free}\alpha_t 
  -\frac{ g_{A}\alpha_t}{2f_\pi\sqrt{q_\pi}}\sum_{\vn,S,I}
   \Delta_{S}\pi'_{I}(\vn,n_t)\rho_{S,I}(\vn) \right. \no & & 
   +\sqrt{-C\alpha_t}\sum_{\vn}s(\vn,n_t)\rho(\vn)
   +i\sqrt{C_{S^2}\alpha_t}\sum_{\vn,S}s_S(\vn,n_t)\rho_S(\vn) 
   \no & &
   \left.
   +i\sqrt{C_{I^2}\alpha_t}\sum_{\vn,I}s_I(\vn,n_t)\rho_I(\vn) 
   +i\sqrt{C_{S^2,I^2}\alpha_t}\sum_{\vn,S,I}s_{SI}(\vn,n_t)\rho_{SI}(\vn) 
 \right\}: \no 
\eea  

Also we add $SU(4)$ symmetric transfer matrix
\bea 
M_4^{(n_t)}&=&:\exp[-H_4\alpha_t]:,\quad H_4=H_{free}
+\frac{1}{2}C_4\sum_{\vq}f(\vq):\rho(\vq)\rho(-\vq):,\quad C_4<0,\no 
M^{(n_t)}_{4,aux}(s)&=&:\exp\left\{ -H_{free}\alpha_t
   +\sqrt{-C_4\alpha_t}\sum_{\vn}s(\vn,n_t)\rho(\vn) \right\}:
\eea 
Because the $SU(4)$ Hamiltonian corresponds to $I$ structure in spin-isospin,
there would be no sign problem for even number of nucleons. 

\subsubsection{Sign problem in modified action} 
In case of chiral EFT, the Hamiltonian have operator structures of
$$ I, i\sigma_S, i\tau_I, i\sigma_S\tau_I, \sigma_S\tau_I $$
Among these, some can have antisymmetric unitary matrix such that satisfy previous 
conditions. \footnote{
\bea 
& &\vs_2^\dagger\vs_1\vs_2=-\vs_1,
\quad \vs_2^\dagger\vs_2\vs_2=\vs_2,
\quad \vs_2^\dagger\vs_3\vs_2=-\vs_3
\no 
& & \vs_2^\dagger(\sigma_S\tau_2)\vs_2 = (\vs_S\tau_2)^*,
\quad \vs_2^\dagger(i\vs_S)\vs_2 = (i\vs_S)^*,  \no  & &
\quad \vs_2^\dagger(i\tau_2)\vs_2= (i\tau_2)^*,
\quad \vs_2^\dagger(i\vs_S\tau_1)\vs_2=   (i\vs_S\tau_1)^*,
\quad \vs_2^\dagger(i\vs_S\tau_3)\vs_2=   (i\vs_S\tau_3)^*
\eea 
} 
\begin{equation}
\begin{array}{ccccccccc}
       & I & i I& \vs_S & i\vs_S & \tau_I & i\tau_I & \vs_S\tau_I & i \vs_S\tau_I \no 
 \vs_2 & O & X  &  X     &  O  &   \tau_1,\tau_3 &  i\tau_2   &  \vs_1\tau_2,\vs_3\tau_2  & i\vs_S\tau_1, i\vs_S\tau_3  \no
 \tau_2 & O & X &  \vs_1,\vs_3 & i\vs_2 & X & O &
   \vs_2\tau_1,\vs_2\tau_3 & i\vs_1\tau_I, i\vs_3\tau_I \no 
\vs_2\tau_3 & O & X & X & O & \tau_2,\tau_3 & i\tau_1 &\vs_S\tau_1
   & i\vs_S\tau_2, i\vs_S\tau_3 
\end{array} 
\end{equation}

The possibility of unitary transformation which satisfies
the condition depends on the operator structure and 
also the initial wave function. 
If the state is spin singlet or isospin singlet,
we can use any $\sigma_S$ or $\tau_I$ and may find
some combination which satisfies the condition.
However, if the states are not singlet, 
we cannot arbitrary transform states using 
$\sigma_S$ and $\tau_I$. Only available
matrix which does not change the configuration would be
$\sigma_3$ for non-zero spin, 
$\tau_3$ for non-zero isospin state.

In some special case of initial state,
which is even number of neutrons paired to spin-singlet and 
even number of protons paired to spin-singlet,
it is possible to find an antisymmetric representation of 
both $\sigma_2$ and $\sigma_2\tau_3$. 
In such cases, all operators in the above
expression would not pose sign problem. However, the exception is 
$i\tau_3$ and $\sigma_S\tau_3$ operators. This implies that
,in case of ${}^4He$, $i\tau_3$ and $\sigma_S\tau_3$ operators
will be the only source of sign problem. Of course, 
it depends on the initial states.

\newpage

\chapter{Sign problem analysis}

To check the properties of operators and their effects in sign problem,
let us compute the phase factor of $\det X(s)$ for various combination of 
interactions.
\bea 
M_{LP,aux}^{(n_t)}(s,s_{S},s_{I},s_{S,I},\pi'_I )
&=& :\exp\left\{ -H_{free}\alpha_t 
  -\frac{ g_{A}\alpha_t}{2f_\pi\sqrt{q_\pi}}\sum_{\vn,S,I}
   \Delta_{S}\pi'_{I}(\vn,n_t)\rho_{S,I}(\vn) \right. \no & & 
   +\sqrt{-C\alpha_t}\sum_{\vn}s(\vn,n_t)\rho(\vn)
   +i\sqrt{C_{S^2}\alpha_t}\sum_{\vn,S}s_S(\vn,n_t)\rho_S(\vn) 
   \no & &
   \left.
   +i\sqrt{C_{I^2}\alpha_t}\sum_{\vn,I}s_I(\vn,n_t)\rho_I(\vn) 
   +i\sqrt{C_{S^2,I^2}\alpha_t}\sum_{\vn,S,I}s_{SI}(\vn,n_t)\rho_{SI}(\vn) 
 \right\}: \no 
\eea  
Also we add $SU(4)$ symmetric transfer matrix
\bea 
M_4^{(n_t)}&=&:\exp[-H_4\alpha_t]:,\quad H_4=H_{free}
+\frac{1}{2}C_4\sum_{\vq}f(\vq):\rho(\vq)\rho(-\vq):,\quad C_4<0,\no 
M^{(n_t)}_{4,aux}(s)&=&:\exp\left\{ -H_{free}\alpha_t
   +\sqrt{-C_4\alpha_t}\sum_{\vn}s(\vn,n_t)\rho(\vn) \right\}:
\eea 
Though the interpolating Hamiltonian was defined as
\bea 
H=d_h H_{LO}+(1-d_h) H_4,
\eea 
all following results are for $d_h=1$. 

\begin{itemize}
\item all calculations are done for $^4He$.
\item $L=6$, $L_t=2*L_{t,out}+L_{t,in}$ , $L_{t,out}=10$, $L_{t,in}=4$,
      $1/a=100$ MeV, $1/a_t=150$ MeV.  
\item Only some of interactions
      are turned on/off by changing overall coupling strength.
      However, not all combinations are considered. 
\item we set $d_h=1.0$.             
\item Here, $SU(4)$ represents the $SU(4)$ symmetric transfer matrix acting 
       as a filter at the
       beginning and ending time steps. 
       This is different from the above $C_4$ in interpolating Hamiltonian.
\item $c_0$ represents the coupling $C(d_h)=d_hC+(1-d_h)C_4$. 
      In similar way, $c_S$, $c_I$,$c_{SI}$ and $g_A$ corresponds to
      $C_{S^2}(d_h)$, $C_{I^2}(d_h)$, $C_{S^2,I^2}(d_h)$ and $g_A(d_h)$.
      \bea 
      & &C_0=-0.192\times 10^{-4}, C_S=0.4\times 10^{-5}, C_I=0.87\times 10^{-5}, \no 
      & &C_{SI}=0.64\times 10^{-5}, C_4=-0.7\times 10^{-4}, \mbox{ in MeV}^{-2},\no 
      & & g_A=1.29
      \eea 
\end{itemize} 

\begin{table}
\caption{Results of various couplings combinations.
Index is for quick look up the result files and
$L_{t,in}$ is the length of time step used in the calculation. 
The numbers represents the numbers multiplied to the default
coupling values. Characters $x,y,z$ represents
the included isospin components in the calculation.
}
\label{tbl:signphase:he4}
\begin{center} 
\begin{tabular}{c|cccccc|ccc}
index  & $SU(4)$&$c_0$&$c_S$&$c_I$&$c_{SI}$&$g_A$& 
                        $ Re \la e^{i\theta}\ra$ & B.E. & raw amplitude\\ \hline 
full &  O &   O &  O  & O   & O      & O   & 0.922&  -27.3(3) &  0.641(3)\\                         
 1   &  O &   X & X   & X   & X      & X   & 1.0 & 28.2(2) & 1.46(3) \\
13   &  O &   X & O   & X   & X      & X   & 1.0  & 25.8(2) &  1.41\\
14   &  O &   X & X   & O   & X      & X   &  1.0 &  21.5(2)& 1.33 \\
15   &  O &   X & X   & X   & O      & X   & 0.992 & 13.1(1) & 1.18 \\
16   &  O &   X & X   & X   & X      & O   &  1.0   &   29.8(1)  & 1.48 \\
 2   &  O &   O & X   & X   & X      & X   & 1.0 & 13.5(3) & 1.20(5) \\
 3   &  O &   O & O   & X   & X      & X   & 1.0 & 7.74(2) & 1.11(3)\\
 4   &  O &   O & X   & O   & X      & X   & 1.0 & -0.4(3)  & 0.999(5) \\
 5   &  O &   O & X   & X   & O      & X   & 0.991 & -14.5(3) & 0.817(4) \\ 
 6   &  O &   O & X   & X   & X      & O   & 1.0 & 18.1(2) & 1.27 \\
29  &  O &   X & X   & X   & O      & O   & 0.951 & 18.6(1) & 1.22\\  
30  &  O &   X & O   & O   & X      & X   & 1.0   & 19.2(1) & 1.29\\
31  &  O &   X & O   & X   & O      & X   & 0.991 & 7.77(9) & 1.09\\
32  &  O &   X & O   & X   & X      & O   & 1.0   & 28.1(1) & 1.45\\
33  &  O &   X & X   & O   & O      & X   & 0.989 & 0.1(1)  & 0.99\\
34  &  O &   X & X   & O   & X      & O   & 1.0   & 24.8(1) & 1.39\\
17  & O   & O  & O  & O  & X     & X  &  1.0 & -4.5(3) & 0.942(5) \\
18 & O  &   O & O  & X  &  O  & X  & 0.991  & -21.6(3) & 0.742(3) \\
 19 & O & O & O & X & X & O & 1.0 & 13.6(2) & 1.19 \\
 20 & O  &  O  & X  & O  & O & X & 0.990 & -31.9(3) & 0.647(3) \\
 21 & O & O  & X & O & X & O & 0.998(1) & 8.5(6) & 1.12(1) \\
 22 & O & O  & X & X & O & O & 0.945 & -5.2(3)  & 0.881(3) \\
 23 & O & O & O & O & O & X & 0.9799 & -37.7(3) & 0.592(3) \\
 24 & O & O & O & O & X & O & 0.998(1) & 2.6(6) & 1.04(1) \\
 25 & O & O & O & X & O & O & 0.942 & -12.2(3) & 0.801(4) \\
 26 & O & O & X & O & O & O & 0.938 & -21.5(3) & 0.704(2) \\
 27 & O & X & O & O & O & O & 0.917 & 3.6(1) & 0.96 \\
 0   &  X &   X & X   & X   & X      & X   & 1.0 & 0. & 1.0 \\
  7  &  X &   O & X   & X   & X      & X   & 1.0 & -1.58(6) & 0.979(1) \\
 8  &  X &   X & O   & X   & X      & X   & 1.0 & -0.339(9) & 0.955(1) \\
11  &  X &   X & X   & O   & X      & X   & 1.0 & -0.88(1) & 0.988 \\
12  &  X &   X & X   & X   & O      & X   & 0.999  & -1.65(2) & 0.977 \\
 9  &  X &   X & X   & X   & X      & O   & 1.0 & -0.10(1) & 0.998 \\
28  &  X &   X & X   & X   & O      & O   & 0.997 & -1.63(2) & 0.975\\  
10  &  X &   O & O   & O   & O      & O   & 0.995 & -5.07(7) & 0.929 \\ 
\end{tabular}  
\end{center} 
\end{table} 

Comments on results in Table.\ref{tbl:signphase:he4} are
\begin{itemize} 
\item When SU(4) interaction is off, the initial and final time steps 
      would not projects to the ground state. Thus, the interaction is
      not strong enough and the system remains as 
      dilute gas and binding energy does not change much from zero. 
      Thus, the results without SU(4) interaction may not represent ground state well.
      
\item In a similar reason, the complex phase becomes smaller without SU(4) interaction.
      Thus, from now on, let us consider the cases with $SU(4)$ interactions. 
       
\item To check sign problem better, 
      it might be better to use larger $L_{t,in}$ values.

\item Looking at the case with (or without) only one interaction, 
      it seems to be the $C_{SI}$ interaction have most important contribution
      to the phase.
      
\item Looking at the case with two interactions, 
     the main origin of phase seems to be the interference between $c_{SI}$ and $g_{A}$.
     Without both $c_{SI}$ and $g_{A}$, phase are usually small. 
     
\item The phase is very small(or zero) even 
      when $c_0,c_S,c_{I},g_A$ interactions are on. 
          
\item The smallest value(or largest phase) occurs when only $c_0$ is off.  
     It may be because $c_0$ interaction provides 
     attractive $SU(4)$ interaction
     to the system to lessen the sign problem.
\end{itemize} 




\begin{table}
\caption{Variation in coupling strength for $^4He$}
\begin{center}
\begin{tabular}{c|cccccc|cc}
index & SU(4) & $C_0$ & $C_S$ & $C_I$ & $C_{SI}$ & $g_A$ 
  & $Re \la e^{i\theta}\ra$ & B.E. \\ \hline
      & 1     &       &       &       &          & 3 
      & 0.586 &  \\
      & 1     &       &       &       &  3       &  
      & 0.871 &  \\      
      & 1     &       &   3    &      &  3       &  
      & 0.997 &  \\
      & 1     &   3   &       &       &         &  
      & 1 &  \\
      & 1     &   1   &   3   &       &         &  
      & 1 &  \\
      & 1     &   1    &       &       &         & 3 
      & 0.369 &  \\
      & 1     &   1    &   3    &   3    &         &  
      & 0.997 & \\
      \hline
      & 1     &   1    &   1    &   1    &   1   & $\sqrt{3}$,y  
      & 0.929 & -28.3(2)\\
      & 1     &   1    &   1    &   1    &   1   & $\sqrt{3}$,z  
      & 0.929 &   \\  
      & 1     &   1    &   1    &   1    &       & 1  
      & 0.998 &   \\   
      \hline
      & 1     &   1    &   1    &   1    & 0.25      & 1  
      & 0.977 &   -3.8(3)\\
       & 1     &   1    &   1    &   1    & 0.5      & 1  
       & 0.959 &   -10.8(2)\\
       & 1     &   1    &   1    &   1    & 0.75      & 1  
       & 0.941 &   -18.9(2)\\
       & 1     &   1    &   1    &   1    & 1      & 1  
       & 0.922 &   -27.3(3)\\                 
       & 1     &   1    &   1    &   1    & y      & y  
       & 0.981 &   -12.3\\            
       & 1     &   1    &   1    &   1    & x      & y  
       & 0.993 &   -13.3\\  
       & 1     &   1    &   1    &   1    & y      & x  
       & 0.993 &   -13.5\\                                      
\end{tabular}  
\end{center} 
\end{table} 

\begin{table}
\caption{Variation in coupling strength for $^6He$}
\begin{center}
\begin{tabular}{c|cccccc|cc}
index & SU(4) & $C_0$ & $C_S$ & $C_I$ & $C_{SI}$ & $g_A$ 
  & $Re \la e^{i\theta}\ra$ & B.E. \\ \hline
      & 1     &    1   &       &       &          &  
      & 1 &  \\
      & 1     &    1   &  1     &       &          &  
      & 1 &  \\            
      & 1     &    1   &       &    1,x   &          &  
      & 1 &  \\ 
      & 1     &    1   &       &    1,y   &          &  
      & 1 &  \\  
      & 1     &    1   &       &    1,z   &          &  
      & 0.887 &  \\
      & 1     &    1   &   1    &    1,z   &          &  
      & 0.898 &  \\ 
     & 1     &    1   &   1    &    1,x,y   &          &  
      & 0.986 &  \\                              
\end{tabular}  
\end{center} 
\end{table} 

\newpage
Let us consider the cases with default interaction
which gives no sign problem for ${}^6He$.
The default combination are
\bea 
& &SU(4),\quad C_0, \quad C_S i\sigma_S ,\quad C_I i\tau_x,\no 
& & C_{SI} i\sigma_S\tau_y,\ C_{SI} i\sigma_S\tau_z,
\quad g_A \sigma_S\tau_x.
\eea
All these interactions can be simultaneously transformed 
to it's complex conjugate
by $\sigma_2\tau_z$. Thus, these interactions does not give rise
to any sign problem.   
Remaining interactions are
\bea
 C_I i \tau_y,\ C_I i \tau_z,\ C_{SI} i\sigma_S\tau_x,\
 g_A \sigma_S\tau_y,\ g_A \sigma_S\tau_z
\eea
Following table shows the effects of each terms to the
complex phase of fermion determinant and
binding energy for $^6He$.
  
\begin{table}
\caption{Variation in coupling strength for $^6He$}
\begin{center}
\begin{tabular}{c|ccccc|cc}
index & $C_{I} i\tau_y$ & $C_I i\tau_z$ 
      &$C_{SI} i\sigma_S\tau_x$ 
      & $g_A\sigma_S\tau_y$ & $g_A\sigma_S\tau_z$ 
      & Re$\la e^{i\theta}\ra$ & B.E. \\ \hline
default &   &  &  &  &  & 1.0 & -21.2(3)\\ 
26    & 1 &  &   &   &   & 0.972 & -22.2(4) \\
27    &   & 1&   &   &   & 0.905 & -22.0(3) \\
29    &   &  & 1 &   &   & 0.927 & -24.3(3) \\
30    &   &  &   & 1 &   & 0.969 & -16.2(3) \\ 
31    &   &  &   &   & 1 & 0.966 & -14.9(3) \\
28    & 1 & 1&   &   &   & 0.875 & -24.1(3) \\
33    & 1 &  & 1 &   &   & 0.897 & -28.3(3) \\
34    & 1 &  &   & 1 &   & 0.939 & -17.7(3) \\
35    & 1 &  &   &   & 1 & 0.933 & -17.0(2) \\
36    &   & 1& 1 &   &   & 0.831 & -27.2(4) \\
37    &   & 1&   & 1 &   & 0.871 & -16.7(4) \\
38    &   & 1&   &   & 1 & 0.868 & -16.3(3) \\
39    &   &  & 1 & 1 &   & 0.891 & -20.0(4) \\
40    &   &  & 1 &   & 1 & 0.882 & -18.9(3) \\
32    &   &  &   &  1& 1 & 0.929 & -9.8(2) \\
41    & 1 & 1& 1 &   &   & 0.802 & -31.6(4) \\
42    & 1 & 1&   & 1 &   & 0.840 & -19.3(3) \\
43    & 1 & 1&   &   & 1 & 0.834 & -18.8(3) \\
44    & 1 &  & 1 & 1 &   & 0.862 & -24.5(3) \\
45    & 1 &  & 1 &   & 1 & 0.849 & -22.6(4) \\
46    & 1 &  &   & 1 & 1 & 0.893 & -11.5(4) \\
47    &   & 1& 1 & 1 &   & 0.793 & -22.9(4) \\
48    &   & 1& 1 &   & 1 & 0.787 & -21.9(4) \\
      &   & 1&   & 1 & 1 &       &          \\
49    &   &  & 1 & 1 & 1 & 0.841 & -14.7(4) \\
50    & 1 & 1 & 1 & 1 &  & 0.768 & -27.9(5) \\
51    & 1 & 1 & 1 &   & 1 & 0.755& -26.5(5) \\
52    & 1 & 1 &  & 1  & 1 & 0.794 &-13.7(4) \\
53    & 1 &   & 1 & 1 & 1 & 0.808 & -19.5(4) \\
54   &    & 1 & 1 & 1 & 1 & 0.743 & -17.4(5)\\
full & 1 & 1 & 1  & 1 & 1 & 0.714 & -23.0(4) \\
\end{tabular}  
\end{center} 
\end{table} 


\chapter{Extrapolation}
There are two extrapolation we are considering. One is the Euclidean time extrapolation,
$T\to \infty$ and the other is the sign extrapolation with some parameter $d\to 1$.
Let us consider sign extrapolation as
\bea 
H(d)=H_0+d H_1 +(1-d) H'
\eea 
where $H'$ is additional auxiliary interaction which may help the convergence
and have some parameter $C_4$
and $H_0+H_1$ is the original full Hamiltonian. Thus, at a finite time interval,
the resulting energy will depends on $N_t$, $C_4$, $d$ and we will take a limit
to extract physical values
\bea
E_0=\lim_{N_t\to \infty, d\to 1} E(N_t,C_4,d)
\eea 
For the time extrapolation we will use following equation,
(T. A. Lahde et.al. J. Phys. G: Nucl. Part. Phys.42(2015)034012) 
\bea 
E_A^j(N_t)=E_{A,0}+\sum_{k=1}^{k_{max}}|c_{A,j,k}|
\exp\left(-\frac{\Delta_{A,k} N_t}{\Lambda_t}  \right) 
\eea 
where, $t=N_t/\Lambda_t$, $\Delta_{A,k}=E_{A,k}-E_{A,0}$
and the index j denotes all different choices of environments(
For example, the introduction of additional interaction
$e^{ H_4 t'}$ for faster ground state projection
introduces dependence on $t'$ and $H_4$).
We require $E_{A,0}$ and $\Delta_{A,k}$ does not depends on the $N_t$ or $j$.

On the other hand, sign extrapolation was done with
\bea 
X(d_h,C_4)=X_0+X_0^{SU(4)}(1-d_h)+\sum_{j=1}^n X_j^{SU(4)}\sin(j\pi d_h).
\eea 
where $X_{0,j}^{SU(4)}$ are dependent on $SU(4)$ interaction. 
The form allows oscillation. And it satisfies the property at $d_h\to 1$
limit and $d_h\to 0$ limits. 

Then, how we should use extrapolation? One possibility is to fit globally using
all data with different $N_t$, $d_h$ values but while keeping one fixed
auxiliary interactions,
\bea 
E(N_t,d_h)=E_0+
\eea 
 

\chapter{Explicit example?}
Let us consider two nucleon system in 1-D lattice size $L=3$
with external delta function potential at 0(this is not inter particle interaction)
\bea 
M=:\exp\left[-H_{free}\alpha_t-C\alpha_t \rho_\uparrow(0)\right]:
\eea 
for two nucleon state, $|k,\uparrow; k',\downarrow\ra $,
\bea 
M_{free}|k,\uparrow; k',\downarrow\ra =e^{-E(k)\alpha_t}e^{-E(k')\alpha_t}|k,\uparrow; k',\downarrow\ra
\eea 
There are three lattice points $0,1,2$ and three momentum eigen states $|p_0\ra,|p_1\ra,|p_2\ra$.  
Let us use single particle  wave function as column vectors
\bea 
 \colmthr{1}{0}{0}= a^\dagger(0)|0\ra,\quad 
 \colmthr{0}{1}{0}= a^\dagger(1)|0\ra,\quad  
 \colmthr{0}{0}{1}= a^\dagger(2)|0\ra. 
\eea 
This implies that we can express a momentum eigenstate as
\bea 
|p\ra = a^\dagger_p|0\ra \propto \sum_{x_n} a^\dagger(x_n) e^{-ip\cdot x_n}|0\ra 
      =\colmthr{e^{-ip\cdot x_0}}{e^{-ip\cdot x_1}}{e^{-ip\cdot x_2}}
\eea 
(For example, zero momentum state becomes $(1, 1 , 1)^T$)

For a single particle state, one only needs linear expansion of transfer matrix,
(note that this expression is only correct for single particle state)
\bea 
M&\simeq &1+\frac{\alpha_t}{2m}[a^\dagger_\uparrow(1) a_\uparrow(0)
                             +a^\dagger_\uparrow(0) a_\uparrow(1)
                             -2 a^\dagger_\uparrow(0) a_\uparrow(0) ]  \no 
     & & + \frac{\alpha_t}{2m}[a^\dagger_\uparrow(2) a_\uparrow(1)
     +a^\dagger_\uparrow(1) a_\uparrow(2)
     -2 a^\dagger_\uparrow(1) a_\uparrow(1) ] \no 
    & &+ \frac{\alpha_t}{2m}[a^\dagger_\uparrow(0) a_\uparrow(2)
    +a^\dagger_\uparrow(2) a_\uparrow(0)
    -2 a^\dagger_\uparrow(2) a_\uparrow(2) ]   \no                     
    & & -C\alpha_t a^\dagger_\uparrow(0) a_\uparrow(0) 
\eea 
Then, we can express M as 
\bea 
M=\threedmat{1-\frac{\alpha_t}{m}-C\alpha_t}{\frac{\alpha_t}{2m}}{\frac{\alpha_t}{2m}}
            {\frac{\alpha_t}{2m}}{1-\frac{\alpha_t}{m}}{\frac{\alpha_t}{2m}}
            {\frac{\alpha_t}{2m}}{\frac{\alpha_t}{2m}}{1-\frac{\alpha_t}{m}}
\eea 
For example, for the eigen-state of $p_1=\frac{2\pi}{3}$
\bea 
|p_1\ra&=& \colmthr{1}{e^{-i\frac{2\pi}{3}}}{e^{-i\frac{4\pi}{3}}}
       = \colmthr{1}{-\frac{1}{2}-i\frac{\sqrt{3}}{2}} {-\frac{1}{2}+i\frac{\sqrt{3}}{2}}\no 
M_{free}|p_1\ra &\sim & e^{-E(p_1)\alpha_t }|p_1\ra         
\eea  

Let us consider two-nucleon states with zero center of mass momentum
and consider zero-range two-nucleon interaction.
\bea 
M=:\exp\left[ -H_{free}\alpha_t -C\alpha_t \sum_{n=0}^{L-1}\rho_\uparrow(n)\rho_\downarrow(n)\right]: 
\eea 

For a zero-center of mass momentum state, with $L=3$, we may have three different relative momentum states,
(Note this represent different states from above example)
\bea 
\colmthr{1}{0}{0}&=&\frac{1}{\sqrt{3}}\left[ a^\dagger_\uparrow(0) a^\dagger_\downarrow(0)
                +a^\dagger_\uparrow(1) a^\dagger_\downarrow(1)
                +a^\dagger_\uparrow(2) a^\dagger_\downarrow(2)\right] |0\ra ,\no 
\colmthr{0}{1}{0}&=&\frac{1}{\sqrt{3}}\left[ a^\dagger_\uparrow(1) a^\dagger_\downarrow(0)
+a^\dagger_\uparrow(2) a^\dagger_\downarrow(1)
+a^\dagger_\uparrow(3) a^\dagger_\downarrow(2)\right] |0\ra ,\no                 
\colmthr{0}{0}{1}&=&\frac{1}{\sqrt{3}}\left[ a^\dagger_\uparrow(2) a^\dagger_\downarrow(0)
+a^\dagger_\uparrow(0) a^\dagger_\downarrow(1)
+a^\dagger_\uparrow(1) a^\dagger_\downarrow(2)\right] |0\ra. 
\eea 
expansion of M
\bea 
M&=& [1+\frac{\alpha_t}{2m}\sum_{n=0}^2 [a^\dagger_\uparrow(n+1) a_\uparrow(n)
                                       +a^\dagger_\uparrow(n) a_\uparrow(n+1)
                                       -2 a^\dagger_\uparrow(n) a_\uparrow(n) ]  ]
    \times [\uparrow\leftrightarrow \downarrow] \no & &
    - C\alpha_t \sum_{n=0}^2 a^\dagger_\uparrow(n)a_\uparrow(n)a^\dagger_\downarrow(n)a_\downarrow(n)
    +\mbox{( terms that vanish in this space )}
\eea 
Considering action of each operators on two nucleon states, we get
\bea 
M&\to&\threedmat{1-\frac{\alpha_t}{m}-C\alpha_t}{\frac{\alpha_t}{2m}}{\frac{\alpha_t}{2m}}
{\frac{\alpha_t}{2m}}{1-\frac{\alpha_t}{m}}{\frac{\alpha_t}{2m}}
{\frac{\alpha_t}{2m}}{\frac{\alpha_t}{2m}}{1-\frac{\alpha_t}{m}}
\threedmat{1-\frac{\alpha_t}{m}-C\alpha_t}{\frac{\alpha_t}{2m}}{\frac{\alpha_t}{2m}}
{\frac{\alpha_t}{2m}}{1-\frac{\alpha_t}{m}}{\frac{\alpha_t}{2m}}
{\frac{\alpha_t}{2m}}{\frac{\alpha_t}{2m}}{1-\frac{\alpha_t}{m}}
+\threedmat{-C\alpha_t}{0}{0}{0}{0}{0}{0}{0}{0} \no 
 &=&\threedmat{A-C\alpha_t}{B}{B}{B}{A}{B}{B}{B}{A}
\eea 
where
\bea 
A=1-2\frac{\alpha_t}{m}+\frac{3\alpha_t^2}{2m^2},\quad 
B=\frac{\alpha_t}{m}-\frac{3\alpha_t}{4m^2}.
\eea 
Then, one can obtain exact solution by solving the eigenvalue equation of $M$.
(eigen-solution will be a linear combination of two-nucleon free states). 

Note that above three states are not eigen-state of $M_{free}$. For example,
one of the eigen-state is $(1,1,1)^T$ which corresponds to 
$|p_0,\uparrow\ra\times|p_0,\downarrow\ra  $ which have zero relative momentum,
\bea 
|p_0,\uparrow\ra\times|p_0,\downarrow\ra
=\left(a^\dagger(0)+a^\dagger(1)+a^\dagger(2)\right)_\uparrow  
 \left(a^\dagger(0)+a^\dagger(1)+a^\dagger(2)\right)_\downarrow |0\ra 
\eea 
In fact, relative energy without interaction can be obtained as
\bea 
E_{rel}=A+2B\cos(\frac{2\pi}{N}k)
\eea 

In fact, one can determine the phase-shift with finite range interaction
from energy level.

In case of 1-D continuum with periodic boundary condition, in center of mass frame of two nucleon. 
Suppose interaction is parity symmetric.
If there is no interaction, 
even parity wave function will be in the form of $\cos(px)$.
If interaction is introduced, the wave function at long distance will change into 
$\cos(px+\delta_0(p))$.
From the periodic condition, $\psi(L/2)=\psi(-L/2)$
and $\psi'(L/2)=\psi'(-L/2)=0$, we get $\sin(p \frac{L}{2}+\delta_0(p))=0$.
In other words, the phase shift can be a function of momentum as
$\delta_0(p)=-p\frac{L}{2}+n\pi$.

In odd parity case, free wave function is $\sin(px)$ and will be shifted by interaction 
as $\sin(px+\delta_1(p))$. Periodic boundary condition,
$\psi(L/2)=\psi(-L/2)=-\psi(-L/2)=0$. Thus, $\sin(p\frac{L}{2}+\delta_1(p))=0$
and $\delta_1(p)=-p\frac{L}{2}+n\pi$. 
Thus, if one obtain energy eigenvalue, one can obtain phase shifts.

In a similar way, if one introduce wall interaction between two nucleons such that 
$\psi(R_{wall})=0$ (L must be larger than wall radius, $L>2 R_{wall}$).
we can obtain relation with phase shift as
\bea 
\cos(pR_{wall}+\delta_0(p))=0\to \delta_0(p)=-pR_{wall}+(n+\frac{1}{2})\pi, \quad \mbox{even parity},\no 
\sin(pR_{wall}+\delta_1(p))=0\to \delta_1(p)=-pR_{wall}+n\pi, \quad \mbox{odd parity}.
\eea 
Note that $p$ will be a function of $L$ or $R_{wall}$. 

This is the basic of Lusher's method and wall boundary method. 

\chapter{Pinhole algorithm} 


\end{document}


